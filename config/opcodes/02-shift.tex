
%===============================================================================
\section{ALU barrel shifter instructions}
%===============================================================================
\datapath{gpRegWE}{'1'}
\datapath{isNOP}{'0'}
\alu{op1Mux}{EXTEND32}
\alu{op2Mux}{EXTEND32}
\alu{opBrMux}{FALSE}
\alu{intResultMux}{SHIFTER}

\datapath{brFmt}{'1'} % Only because legacy decoding did this.
\alu{bitwiseOp}{BITW_AND} % Only because legacy decoding did this.

The \rvex{} ALU includes a barrel shifter. It should be noted that the shift
amount input to the barrel shifter is 8-bit unsigned, not 32-bit as one might
expect. That is, the upper 24 bits of the shift amount are discarded, and for
instance a left shift by a negative amount will \emph{not} simply result in a
right shift.

%-------------------------------------------------------------------------------
\syllable{01101111-}{SHL}{\rd = \rx, \ry}
\alu{unsignedOp}{'1'}
\alu{shiftLeft}{'1'}
Performs a left-shift operation. Zeros are shifted in from the right.

\begin{lstlisting}[numbers=none, basicstyle=\ttfamily\footnotesize, language=C++]
\rd = \rx << \ry;
\end{lstlisting}

%-------------------------------------------------------------------------------
\syllable{00011000-}{SHR}{\rd = \rx, \ry}
Performs a signed right-shift operation. That is, the sign bit of \code{\rx} is
shifted in from the left.

\begin{lstlisting}[numbers=none, basicstyle=\ttfamily\footnotesize, language=C++]
\rd = \rx >> \ry;
\end{lstlisting}

%-------------------------------------------------------------------------------
\syllable{00011001-}{SHRU}{\rd = \rx, \ry}
\alu{unsignedOp}{'1'}
Performs an unsigned right-shift operation. That is, zeros are shifted in from
the left.

\begin{lstlisting}[numbers=none, basicstyle=\ttfamily\footnotesize, language=C++]
\rd = (unsigned int)\rx >> \ry;
\end{lstlisting}
