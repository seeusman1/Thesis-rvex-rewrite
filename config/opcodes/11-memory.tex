
%===============================================================================
\section{Memory instructions}
%===============================================================================
\class{MEM}
\datapath{isNOP}{'0'}
\datapath{enableALU}{'1'}
\datapath{enableMem}{'1'}
\alu{op1Mux}{EXTEND32}
\alu{op2Mux}{EXTEND32}
\alu{opBrMux}{FALSE}
\alu{intResultMux}{ADDER}
\memory{isMemoryInstruction}{'1'}

\alu{bitwiseOp}{BITW_AND} % Only because legacy decoding did this.

Some \rvex{} pipelanes have a memory unit. The memory unit supports byte,
halfword and word operations. Sign or zero extension is part of the byte and
halfword load instructions.

The addressing mode is always register $+$ immediate. Note that attempts to read
misaligned memory locations will fail with a \trap{MISALIGNED_ACCESS} trap.

In the default pipeline configuration, these instructions are pipelined by
two cycles. That is, the result of a memory load instruction is not available
yet in the subsequent instruction. However, the current cache and core guarantee
that a memory write to address $x$ immediately followed by a memory read from
address $x$ returns the newly written value.

%-------------------------------------------------------------------------------
\syllable{000100001}{LDW}{\rd = \ry[\rx]}
\datapath{funcSel}{MEM}
\datapath{gpRegWE}{'1'}
\memory{readEnable}{'1'}
\memory{accessSizeBLog2}{ACCESS_SIZE_WORD}
Loads a 32-bit word from memory.

\begin{lstlisting}[numbers=none, basicstyle=\ttfamily\footnotesize, language=C++]
\rd = *(int*)(\rx + \ry);
\end{lstlisting}

%-------------------------------------------------------------------------------
\syllable{000100011}{LDH}{\rd = \ry[\rx]}
\datapath{funcSel}{MEM}
\datapath{gpRegWE}{'1'}
\memory{readEnable}{'1'}
\memory{accessSizeBLog2}{ACCESS_SIZE_HALFWORD}
Loads a 16-bit halfword from memory and sign-extends it.

\begin{lstlisting}[numbers=none, basicstyle=\ttfamily\footnotesize, language=C++]
\rd = *(short*)(\rx + \ry);
\end{lstlisting}

%-------------------------------------------------------------------------------
\syllable{000100101}{LDHU}{\rd = \ry[\rx]}
\datapath{funcSel}{MEM}
\datapath{gpRegWE}{'1'}
\memory{readEnable}{'1'}
\memory{unsignedOp}{'1'}
\memory{accessSizeBLog2}{ACCESS_SIZE_HALFWORD}
Loads a 16-bit halfword from memory and zero-extends it.

\begin{lstlisting}[numbers=none, basicstyle=\ttfamily\footnotesize, language=C++]
\rd = *(unsigned short*)(\rx + \ry);
\end{lstlisting}

%-------------------------------------------------------------------------------
\syllable{000100111}{LDB}{\rd = \ry[\rx]}
\datapath{funcSel}{MEM}
\datapath{gpRegWE}{'1'}
\memory{readEnable}{'1'}
\memory{accessSizeBLog2}{ACCESS_SIZE_BYTE}
Loads a byte from memory and sign-extends it.

\begin{lstlisting}[numbers=none, basicstyle=\ttfamily\footnotesize, language=C++]
\rd = *(char*)(\rx + \ry);
\end{lstlisting}

%-------------------------------------------------------------------------------
\syllable{000101001}{LDBU}{\rd = \ry[\rx]}
\datapath{funcSel}{MEM}
\datapath{gpRegWE}{'1'}
\memory{readEnable}{'1'}
\memory{unsignedOp}{'1'}
\memory{accessSizeBLog2}{ACCESS_SIZE_BYTE}
Loads a byte from memory and zero-extends it.

\begin{lstlisting}[numbers=none, basicstyle=\ttfamily\footnotesize, language=C++]
\rd = *(unsigned char*)(\rx + \ry);
\end{lstlisting}

%-------------------------------------------------------------------------------
\syllable{000011011}{LDW}{\lr = \ry[\rx]}
\datapath{funcSel}{MEM}
\datapath{linkWE}{'1'}
\memory{readEnable}{'1'}
\memory{accessSizeBLog2}{ACCESS_SIZE_WORD}
Loads a word from memory. The result is written to the link register.

\begin{lstlisting}[numbers=none, basicstyle=\ttfamily\footnotesize, language=C++]
\lr = *(int*)(\rx + \ry);
\end{lstlisting}

%-------------------------------------------------------------------------------
\syllable{001011101}{LDBR}{\ry[\rx]}
\datapath{funcSel}{MEM}
\datapath{allBrRegsWE}{'1'}
\memory{readEnable}{'1'}
\memory{unsignedOp}{'1'}
\memory{accessSizeBLog2}{ACCESS_SIZE_BYTE}
Loads a byte from memory. The result is written to the entire branch register
file at once. This is intended to improve context switching performance
somewhat.

\begin{lstlisting}[numbers=none, basicstyle=\ttfamily\footnotesize, language=C++]
char tmp = *(char*)(\rx + \ry);
$b0.0 = tmp & 1;
$b0.1 = tmp & 2;
$b0.2 = tmp & 4;
$b0.3 = tmp & 8;
$b0.4 = tmp & 16;
$b0.5 = tmp & 32;
$b0.6 = tmp & 64;
$b0.7 = tmp & 128;
\end{lstlisting}

%-------------------------------------------------------------------------------
\syllable{000101011}{STW}{\ry[\rx] = \rd}
\memory{writeEnable}{'1'}
\memory{accessSizeBLog2}{ACCESS_SIZE_WORD}
Stores a 32-bit word into memory.

\begin{lstlisting}[numbers=none, basicstyle=\ttfamily\footnotesize, language=C++]
*(int*)(\rx + \ry) = \rd;
\end{lstlisting}

%-------------------------------------------------------------------------------
\syllable{000101101}{STH}{\ry[\rx] = \rd}
\memory{writeEnable}{'1'}
\memory{accessSizeBLog2}{ACCESS_SIZE_HALFWORD}
Stores a 16-bit halfword into memory.

\begin{lstlisting}[numbers=none, basicstyle=\ttfamily\footnotesize, language=C++]
*(short*)(\rx + \ry) = \rd;
\end{lstlisting}

%-------------------------------------------------------------------------------
\syllable{000101111}{STB}{\ry[\rx] = \rd}
\memory{writeEnable}{'1'}
\memory{accessSizeBLog2}{ACCESS_SIZE_BYTE}
Stores a byte into memory.

\begin{lstlisting}[numbers=none, basicstyle=\ttfamily\footnotesize, language=C++]
*(char*)(\rx + \ry) = \rd;
\end{lstlisting}

%-------------------------------------------------------------------------------
\syllable{000011101}{STW}{\ry[\rx] = \lr}
\datapath{op3LinkReg}{'1'}
\memory{writeEnable}{'1'}
\memory{accessSizeBLog2}{ACCESS_SIZE_WORD}
Store word in memory, from link register.

\begin{lstlisting}[numbers=none, basicstyle=\ttfamily\footnotesize, language=C++]
*(int*)(\rx + \ry) = \lr;
\end{lstlisting}

%-------------------------------------------------------------------------------
\syllable{001011111}{STBR}{\ry[\rx]}
\datapath{op3BranchRegs}{'1'}
\memory{writeEnable}{'1'}
\memory{accessSizeBLog2}{ACCESS_SIZE_BYTE}
Store byte in memory, from branch register file.

\begin{lstlisting}[numbers=none, basicstyle=\ttfamily\footnotesize, language=C++]
char tmp = $b0.0;
tmp |= $b0.1 << 1
tmp |= $b0.2 << 2
tmp |= $b0.3 << 3
tmp |= $b0.4 << 4
tmp |= $b0.5 << 5
tmp |= $b0.6 << 6
tmp |= $b0.7 << 7
*(char*)(\rx + \ry) = tmp
\end{lstlisting}
