
%===============================================================================
\section{ALU single-bit instructions}
%===============================================================================
\datapath{isNOP}{'0'}
\alu{op1Mux}{EXTEND32}
\alu{op2Mux}{EXTEND32}
\alu{opBrMux}{FALSE}

\alu{bitwiseOp}{BITW_AND} % Only because legacy decoding did this.

The \rvex{} ALU supports several bitfield operations in a single cycle. Note
that the bit selection logic follows the same rules as the shift amount in the
barrel shifter. That is, only the least significant byte of the bit selection
operand is used to select the bit, the rest is ignored.

%-------------------------------------------------------------------------------
\syllable{00101100-}{SBIT}{\rd = \rx, \ry}
\datapath{gpRegWE}{'1'}
\datapath{brFmt}{'1'} % Only because legacy decoding did this.
\alu{opBrMux}{TRUE}
\alu{bitwiseOp}{SET_BIT}
\alu{intResultMux}{BITWISE}
Sets a given bit in a 32-bit integer.

\begin{lstlisting}[numbers=none, basicstyle=\ttfamily\footnotesize, language=C++]
\rd = \rx | (1 << \ry);
\end{lstlisting}

%-------------------------------------------------------------------------------
\syllable{00101101-}{SBITF}{\rd = \rx, \ry}
\datapath{gpRegWE}{'1'}
\datapath{brFmt}{'1'} % Only because legacy decoding did this.
\alu{opBrMux}{FALSE}
\alu{bitwiseOp}{SET_BIT}
\alu{intResultMux}{BITWISE}
Clears a given bit in a 32-bit integer.

\begin{lstlisting}[numbers=none, basicstyle=\ttfamily\footnotesize, language=C++]
\rd = \rx & ~(1 << \ry);
\end{lstlisting}

%-------------------------------------------------------------------------------
\syllable{01011100-}{TBIT}{\rd = \rx, \ry}
\datapath{gpRegWE}{'1'}
\datapath{brFmt}{'1'} % Only because legacy decoding did this.
\alu{intResultMux}{BOOL}
\alu{brResultMux}{TBIT}
Copies a given bit to an integer register.

\begin{lstlisting}[numbers=none, basicstyle=\ttfamily\footnotesize, language=C++]
\rd = (\rx & (1 << \ry)) != 0;
\end{lstlisting}

%-------------------------------------------------------------------------------
\syllable{01011101-}{TBIT}{\bd = \rx, \ry}
\datapath{brRegWE}{'1'}
\alu{intResultMux}{BOOL}
\alu{brResultMux}{TBIT}
Copies a given bit to a branch register.

\begin{lstlisting}[numbers=none, basicstyle=\ttfamily\footnotesize, language=C++]
\bd = (\rx & (1 << \ry)) != 0;
\end{lstlisting}

%-------------------------------------------------------------------------------
\syllable{01011110-}{TBITF}{\rd = \rx, \ry}
\datapath{gpRegWE}{'1'}
\datapath{brFmt}{'1'} % Only because legacy decoding did this.
\alu{intResultMux}{BOOL}
\alu{brResultMux}{TBITF}
Copies the complement of a given bit to an integer register.

\begin{lstlisting}[numbers=none, basicstyle=\ttfamily\footnotesize, language=C++]
\rd = (\rx & (1 << \ry)) == 0;
\end{lstlisting}

%-------------------------------------------------------------------------------
\syllable{01011111-}{TBITF}{\bd = \rx, \ry}
\datapath{brRegWE}{'1'}
\alu{intResultMux}{BOOL}
\alu{brResultMux}{TBITF}
Copies the complement of a given bit to a branch register.

\begin{lstlisting}[numbers=none, basicstyle=\ttfamily\footnotesize, language=C++]
\bd = (\rx & (1 << \ry)) == 0;
\end{lstlisting}
