
%===============================================================================
\section{ALU arithmetic instructions}
%===============================================================================
\datapath{gpRegWE}{'1'}
\datapath{isNOP}{'0'}
\alu{intResultMux}{ADDER}
\alu{op1Mux}{EXTEND32}
\alu{op2Mux}{EXTEND32}
\alu{opBrMux}{FALSE}

\datapath{brFmt}{'1'} % Only because legacy decoding did this.
\alu{bitwiseOp}{BITW_AND} % Only because legacy decoding did this.

The \rvex{} ALU has a 32-bit adder for arithmetic. Some exotic instructions are
available to make efficient multiplications by small constants and to speed up
software divisions.

%-------------------------------------------------------------------------------
\syllable{01100010-}{ADD}{\rd = \rx, \ry}
Performs a 32-bit addition. Notice that \insn{ADD} instructions may be used as
move or load immediate operations when \code{x} is set to 0. While the \insn{OR}
instruction is often used instead, there is no functional difference between the
two when used inthis way.

\begin{lstlisting}[numbers=none, basicstyle=\ttfamily\footnotesize, language=C++]
\rd = \rx + \ry;
\end{lstlisting}

%-------------------------------------------------------------------------------
\syllable{01101011-}{SH1ADD}{\rd = \rx, \ry}
\alu{op1Mux}{SHL1}
Performs a 32-bit addition. \code{\rx} is first left-shifted by one.

\begin{lstlisting}[numbers=none, basicstyle=\ttfamily\footnotesize, language=C++]
\rd = (\rx << 1) + \ry;
\end{lstlisting}

%-------------------------------------------------------------------------------
\syllable{01101100-}{SH2ADD}{\rd = \rx, \ry}
\alu{op1Mux}{SHL2}
Performs a 32-bit addition. \code{\rx} is first left-shifted by two.

\begin{lstlisting}[numbers=none, basicstyle=\ttfamily\footnotesize, language=C++]
\rd = (\rx << 2) + \ry;
\end{lstlisting}

%-------------------------------------------------------------------------------
\syllable{01101101-}{SH3ADD}{\rd = \rx, \ry}
\alu{op1Mux}{SHL3}
Performs a 32-bit addition. \code{\rx} is first left-shifted by three.

\begin{lstlisting}[numbers=none, basicstyle=\ttfamily\footnotesize, language=C++]
\rd = (\rx << 3) + \ry;
\end{lstlisting}

%-------------------------------------------------------------------------------
\syllable{01101110-}{SH4ADD}{\rd = \rx, \ry}
\alu{op1Mux}{SHL4}
Performs a 32-bit addition. \code{\rx} is first left-shifted by four.

\begin{lstlisting}[numbers=none, basicstyle=\ttfamily\footnotesize, language=C++]
\rd = (\rx << 4) + \ry;
\end{lstlisting}

%-------------------------------------------------------------------------------
\syllable{00011010-}{SUB}{\rd = \ry, \rx}
\alu{op1Mux}{EXTEND32INV}
\alu{opBrMux}{TRUE}
Performs a 32-bit subtraction. Note that, unlike all other instructions, the
immediate must be specified first. This allows \insn{SUB} to be used to subtract
a register from an immediate.

Notice that \insn{SUB} reduces to two's complement negation when \code{x} or
\code{imm} equal zero.

\begin{lstlisting}[numbers=none, basicstyle=\ttfamily\footnotesize, language=C++]
\rd = \ry + \rx;
\end{lstlisting}

%-------------------------------------------------------------------------------
\syllable{01111---0}{ADDCG}{\rd, \bd = \bs, \rx, \ry}
\datapath{brFmt}{'1'}
\datapath{brRegWE}{'1'}
\alu{opBrMux}{PASS}
\alu{unsignedOp}{'1'}
\alu{brResultMux}{CARRY_OUT}
Primitive for additions of integers wider than 32 bits. Addition is performed by
first setting a scratch branch register to false using \insn{NEQ} for the carry
input. Then \insn{ADDCG} can be used to add up words together one by one with
increasing significance, using the scratch branch register for the carry chain.

Subtractions can be performed by setting the carry input to 1 using \insn{EQ}
and ones-complementing one of the inputs using \insn{XOR}.

\begin{lstlisting}[numbers=none, basicstyle=\ttfamily\footnotesize, language=C++]
long long tmp = \rx + \ry + (\bs ? 1 : 0);
\rd = (int)tmp;
\bd = (tmp & 0x100000000) != 0;
\end{lstlisting}

%-------------------------------------------------------------------------------
\syllable{01110---0}{DIVS}{\rd, \bd = \bs, \rx, \ry}
\datapath{brFmt}{'1'}
\datapath{brRegWE}{'1'}
\alu{op1Mux}{SHL1}
\alu{opBrMux}{PASS}
\alu{divs}{'1'}
\alu{brResultMux}{DIVS}
Primitive for integer divisions, used in conjunction with \insn{ADDCG}.
\todo[inline]{Figure out how divisions work so it can be explained here.}

Notice that \insn{DIVS} reduces to rotate left by one through a branch register
when \code{y} is zero. This may be used for for shift left by one
operations on integers wider than 32 bits.

\begin{lstlisting}[numbers=none, basicstyle=\ttfamily\footnotesize, language=C++]
int tmp = (\rx << 1) | (\bs ? 1 : 0);
bool flag = (\rx & 0x80000000) != 0;
\rd = flag ? (tmp + \ry) : (tmp - \ry);
\bd = flag;
\end{lstlisting}
