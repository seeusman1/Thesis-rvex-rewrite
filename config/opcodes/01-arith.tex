
%===============================================================================
\section{ALU arithmetic instructions}
%===============================================================================
\datapath{gpRegWE}{'1'}
\datapath{isNOP}{'0'}
\alu{intResultMux}{ADDER}
\alu{op1Mux}{EXTEND32}
\alu{op2Mux}{EXTEND32}
\alu{opBrMux}{FALSE}

\datapath{brFmt}{'1'} % Only because legacy decoding did this.
\alu{bitwiseOp}{BITW_AND} % Only because legacy decoding did this.

The \rvex{} ALU has a 32-bit adder for arithmetic. Some exotic instructions are
available to make efficient multiplications by small constants and to speed up
software divisions.

%-------------------------------------------------------------------------------
\syllable{01100010-}{ADD}{\rd = \rx, \ry}
Performs a 32-bit addition. Notice that \insn{ADD} instructions may be used as
move or load immediate operations when \code{x} is set to 0. While the \insn{OR}
instruction is often used instead, there is no functional difference between the
two when used inthis way.

\begin{lstlisting}[numbers=none, basicstyle=\ttfamily\footnotesize, language=C++]
\rd = \rx + \ry;
\end{lstlisting}

%-------------------------------------------------------------------------------
\syllable{01101011-}{SH1ADD}{\rd = \rx, \ry}
\alu{op1Mux}{SHL1}
Performs a 32-bit addition. \code{\rx} is first left-shifted by one.

\begin{lstlisting}[numbers=none, basicstyle=\ttfamily\footnotesize, language=C++]
\rd = (\rx << 1) + \ry;
\end{lstlisting}

%-------------------------------------------------------------------------------
\syllable{01101100-}{SH2ADD}{\rd = \rx, \ry}
\alu{op1Mux}{SHL2}
Performs a 32-bit addition. \code{\rx} is first left-shifted by two.

\begin{lstlisting}[numbers=none, basicstyle=\ttfamily\footnotesize, language=C++]
\rd = (\rx << 2) + \ry;
\end{lstlisting}

%-------------------------------------------------------------------------------
\syllable{01101101-}{SH3ADD}{\rd = \rx, \ry}
\alu{op1Mux}{SHL3}
Performs a 32-bit addition. \code{\rx} is first left-shifted by three.

\begin{lstlisting}[numbers=none, basicstyle=\ttfamily\footnotesize, language=C++]
\rd = (\rx << 3) + \ry;
\end{lstlisting}

%-------------------------------------------------------------------------------
\syllable{01101110-}{SH4ADD}{\rd = \rx, \ry}
\alu{op1Mux}{SHL4}
Performs a 32-bit addition. \code{\rx} is first left-shifted by four.

\begin{lstlisting}[numbers=none, basicstyle=\ttfamily\footnotesize, language=C++]
\rd = (\rx << 4) + \ry;
\end{lstlisting}

%-------------------------------------------------------------------------------
\syllable{00011010-}{SUB}{\rd = \ry, \rx}
\alu{op1Mux}{EXTEND32INV}
\alu{opBrMux}{TRUE}
Performs a 32-bit subtraction. Note that, unlike all other instructions, the
immediate must be specified first. This allows \insn{SUB} to be used to subtract
a register from an immediate.

Notice that \insn{SUB} reduces to two's complement negation when \code{x} or
\code{imm} equal zero.

\begin{lstlisting}[numbers=none, basicstyle=\ttfamily\footnotesize, language=C++]
\rd = \ry + \rx;
\end{lstlisting}

%-------------------------------------------------------------------------------
\syllable{01111---0}{ADDCG}{\rd, \bd = \bs, \rx, \ry}
\datapath{brFmt}{'1'}
\datapath{brRegWE}{'1'}
\alu{opBrMux}{PASS}
\alu{unsignedOp}{'1'}
\alu{brResultMux}{CARRY_OUT}
Primitive for additions of integers wider than 32 bits. Addition is performed by 
first setting a scratch branch register to false using \insn{CMPNE} for the 
carry input. Then \insn{ADDCG} can be used to add up words together one by one 
with increasing significance, using the scratch branch register for the carry 
chain.

Subtractions can be performed by setting the carry input to 1 using \insn{CMPEQ} 
and ones-complementing one of the inputs using \insn{XOR}.

Notice that \insn{ADDCG} reduces to rotate left by one through a branch register
when \code{x} equals \code{y}. This may be used for for shift left by one
operations on integers wider than 32 bits.

\begin{lstlisting}[numbers=none, basicstyle=\ttfamily\footnotesize, language=C++]
long long tmp = \rx + \ry + (\bs ? 1 : 0);
\rd = (int)tmp;
\bd = (tmp & 0x100000000) != 0;
\end{lstlisting}

%-------------------------------------------------------------------------------
\syllable{01110---0}{DIVS}{\rd, \bd = \bs, \rx, \ry}
\datapath{brFmt}{'1'}
\datapath{brRegWE}{'1'}
\alu{op1Mux}{SHL1}
\alu{opBrMux}{PASS}
\alu{divs}{'1'}
\alu{brResultMux}{DIVS}
Primitive for integer divisions. The following assembly code performs a single 
nonrestoring division step.

\begin{lstlisting}[numbers=none, basicstyle=\ttfamily\footnotesize, language=vexasm]
addcg  s_lo, shift_bit    = s_lo, s_lo, <zero>
;;
divs   s_hi, quotient_bit = s_hi, divisor, shift_bit
;;
addcg  quotient, <unused> = quotient, quotient, quotient_bit
;;
\end{lstlisting}

\noindent Here, \code{s_lo} and \code{s_hi} represent the partial remainder,
initialized to the dividend before the first division step. The first
\insn{ADDCG} and the \insn{DIVS} instruction together perform a 64-bit
shift-left-by-1 operation. Depending on the sign of the partial remainder before
the shift (i.e., the MSB which was shifted out), the dividend is added to or
subtracted from the partial remainder. If an addition was performed,
\code{quotient_bit} is set to 1, representing that the current quotient bit is
-1 in binary signed digit representation. Otherwise, the current quotient bit is
1. The final \insn{ADDCG} stores the result by left-shifting the bit into
\code{quotient}. Post-processing is required to convert the quotient from binary
signed digit representation to two's complement.

It should be noted that a division step can be done in a single cycle using just 
two syllables. This division step has to be applied many times in a loop, and 
benefits from modulo-scheduling (also known as software pipelining), allowing 
each step to be performed in a single cycle. Furthermore, the
\code{quotient_bit}s can be shifted into \code{s_lo} instead of a different
register, as the zero bits shifted into \code{s_lo} are unused. This eliminates
the need for the second \insn{ADDCG}, as well as the zero and scratch branch
registers.

The prologue and epilogue code for various divisions are beyond the scope of
this manual, as the compilers take care of expanding divisions.

\begin{lstlisting}[numbers=none, basicstyle=\ttfamily\footnotesize, language=C++]
int tmp = (\rx << 1) | (\bs ? 1 : 0);
bool flag = (\rx & 0x80000000) != 0;
\rd = flag ? (tmp + \ry) : (tmp - \ry);
\bd = flag;
\end{lstlisting}
