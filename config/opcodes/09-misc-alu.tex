
%===============================================================================
\section{ALU miscellaneous instructions}
%===============================================================================
\datapath{isNOP}{'0'}

%-------------------------------------------------------------------------------
\syllable{01100000-}{NOP}{}
\datapath{isNOP}{'1'}
Performs no operation.

%-------------------------------------------------------------------------------
\syllable{100100010}{CLZ}{\rd = \rx}
\datapath{gpRegWE}{'1'}
\datapath{brFmt}{'1'} % Only because legacy decoding did this.
\alu{op1Mux}{EXTEND32}
\alu{op2Mux}{EXTEND32}
\alu{opBrMux}{FALSE} % Only because legacy decoding did this.
\alu{bitwiseOp}{BITW_AND} % Only because legacy decoding did this.
\alu{intResultMux}{CLZ}
This operations counts the number of leading zeros in the operand. That is, the
value 0x80000000 returns 0 and the value 0 returns 32.

\begin{lstlisting}[numbers=none, basicstyle=\ttfamily\footnotesize, language=C++]
unsigned int in = \rx;
int out = 32;
while (in) {
  in >>= 1;
  out--;
}
\rd = out;
\end{lstlisting}

%-------------------------------------------------------------------------------
\syllable{00001011-}{MOVTL}{\lr = \ry}
\datapath{linkWE}{'1'}
\alu{op2Mux}{EXTEND32}
\alu{opBrMux}{FALSE}
\alu{intResultMux}{OP_SEL}
Copies a general purpose register or immediate to the link register.

\begin{lstlisting}[numbers=none, basicstyle=\ttfamily\footnotesize, language=C++]
\lr = \ry;
\end{lstlisting}

%-------------------------------------------------------------------------------
\syllable{000011000}{MOVFL}{\rd = \lr}
\datapath{gpRegWE}{'1'}
\datapath{op1LinkReg}{'1'}
\alu{op1Mux}{EXTEND32},
\alu{op2Mux}{ZERO},
\alu{bitwiseOp}{BITW_OR},
\alu{intResultMux}{BITWISE},
Copies the link register to a general purpose register.

\begin{lstlisting}[numbers=none, basicstyle=\ttfamily\footnotesize, language=C++]
\rd = \lr;
\end{lstlisting}

%-------------------------------------------------------------------------------
\syllable{10010000-}{TRAP}{\rx, \ry}
\datapath{isTrap}{'1'}
Software trap. The first parameter is the trap argument, while the second
parameter is the trap cause byte.
