
%===============================================================================
\section{Multiply instructions}
%===============================================================================
\class{MUL}
\datapath{funcSel}{MUL}
\datapath{gpRegWE}{'1'}
\datapath{isNOP}{'0'}
\multiplier{isMultiplyInstruction}{'1'}

\rvex{} pipelanes may be design-time configured to contain a multiplication
unit. This unit supports 16x16 and 16x32 multiplications.

In the default pipeline configuration, these instructions are pipelined by
two cycles. That is, the result of a multiply instruction is not available yet
in the subsequent instruction.

%-------------------------------------------------------------------------------
\syllable{00000000-}{MPYLL}{\rd = \rx, \ry}
\multiplier{op1sel}{LOW_HALF}
\multiplier{op2sel}{LOW_HALF}
\multiplier{resultSel}{PASS}
Signed 16x16-bit to 32-bit multiplication.

\begin{lstlisting}[numbers=none, basicstyle=\ttfamily\footnotesize, language=C++]
\rd = (short)\rx * (short)\ry;
\end{lstlisting}

%-------------------------------------------------------------------------------
\syllable{00000001-}{MPYLLU}{\rd = \rx, \ry}
\multiplier{op1sel}{LOW_HALF}
\multiplier{op1unsigned}{'1'}
\multiplier{op2sel}{LOW_HALF}
\multiplier{op2unsigned}{'1'}
\multiplier{resultSel}{PASS}
Unsigned 16x16-bit to 32-bit multiplication.

\begin{lstlisting}[numbers=none, basicstyle=\ttfamily\footnotesize, language=C++]
\rd = (unsigned short)\rx * (unsigned short)\ry;
\end{lstlisting}

%-------------------------------------------------------------------------------
\syllable{00000010-}{MPYLH}{\rd = \rx, \ry}
\multiplier{op1sel}{LOW_HALF}
\multiplier{op2sel}{HIGH_HALF}
\multiplier{resultSel}{PASS}
Signed 16x16-bit to 32-bit multiplication, using the high halfword of
\code{\ry}.

\begin{lstlisting}[numbers=none, basicstyle=\ttfamily\footnotesize, language=C++]
\rd = (short)\rx * (short)(\ry >> 16);
\end{lstlisting}

%-------------------------------------------------------------------------------
\syllable{00000011-}{MPYLHU}{\rd = \rx, \ry}
\multiplier{op1sel}{LOW_HALF}
\multiplier{op1unsigned}{'1'}
\multiplier{op2sel}{HIGH_HALF}
\multiplier{op2unsigned}{'1'}
\multiplier{resultSel}{PASS}
Unsigned 16x16-bit to 32-bit multiplication, using the high halfword of
\code{\ry}.

\begin{lstlisting}[numbers=none, basicstyle=\ttfamily\footnotesize, language=C++]
\rd = (unsigned short)\rx * (unsigned short)(\ry >> 16);
\end{lstlisting}

%-------------------------------------------------------------------------------
\syllable{00000100-}{MPYHH}{\rd = \rx, \ry}
\multiplier{op1sel}{HIGH_HALF}
\multiplier{op2sel}{HIGH_HALF}
\multiplier{resultSel}{PASS}
Signed 16x16-bit to 32-bit multiplication, using the high halfword of both
operands.

\begin{lstlisting}[numbers=none, basicstyle=\ttfamily\footnotesize, language=C++]
\rd = (short)(\rx >> 16) * (short)(\ry >> 16);
\end{lstlisting}

%-------------------------------------------------------------------------------
\syllable{00000101-}{MPYHHU}{\rd = \rx, \ry}
\multiplier{op1sel}{HIGH_HALF}
\multiplier{op1unsigned}{'1'}
\multiplier{op2sel}{HIGH_HALF}
\multiplier{op2unsigned}{'1'}
\multiplier{resultSel}{PASS}
Unsigned 16x16-bit to 32-bit multiplication, using the high halfword of both
operands.

\begin{lstlisting}[numbers=none, basicstyle=\ttfamily\footnotesize, language=C++]
\rd = (unsigned short)(\rx >> 16) * (unsigned short)(\ry >> 16);
\end{lstlisting}

%-------------------------------------------------------------------------------
\syllable{00000110-}{MPYL}{\rd = \rx, \ry}
\multiplier{op1sel}{WORD}
\multiplier{op2sel}{LOW_HALF}
\multiplier{resultSel}{PASS}
Signed 16x32-bit to 32-bit multiplication. \code{\rx} is the 32-bit operand,
\code{\ry} is the 16-bit operand. The upper 16 bits of the multiplication result
are discarded.

\begin{lstlisting}[numbers=none, basicstyle=\ttfamily\footnotesize, language=C++]
\rd = (int)\rx * (short)\ry;
\end{lstlisting}

%-------------------------------------------------------------------------------
\syllable{00000111-}{MPYLU}{\rd = \rx, \ry}
\multiplier{op1sel}{WORD}
\multiplier{op1unsigned}{'1'}
\multiplier{op2sel}{LOW_HALF}
\multiplier{op2unsigned}{'1'}
\multiplier{resultSel}{PASS}
Unsigned 16x32-bit to 32-bit multiplication. \code{\rx} is the 32-bit operand,
\code{\ry} is the 16-bit operand. The upper 16 bits of the multiplication result
are discarded.

This operation may be used as a primitive for 32x32-bit to 32-bit
multiplication, computing the partial product of \code{\rx} and the lower half
of \code{\ry}. \insn{MPYHS} is then used to compute the other partial product. A
final \insn{ADD} instruction combines the partial products.

\begin{lstlisting}[numbers=none, basicstyle=\ttfamily\footnotesize, language=C++]
\rd = (unsigned int)\rx * (unsigned short)\ry;
\end{lstlisting}

%-------------------------------------------------------------------------------
\syllable{00001000-}{MPYH}{\rd = \rx, \ry}
\multiplier{op1sel}{WORD}
\multiplier{op2sel}{HIGH_HALF}
\multiplier{resultSel}{PASS}
Signed 16x32-bit to 32-bit multiplication. \code{\rx} is the 32-bit operand, the 
upper halfword of \code{\ry} is the 16-bit operand. The upper 16 bits of the 
multiplication result are discarded.

\begin{lstlisting}[numbers=none, basicstyle=\ttfamily\footnotesize, language=C++]
\rd = (int)\rx * (short)(\ry >> 16);
\end{lstlisting}

%-------------------------------------------------------------------------------
\syllable{00001001-}{MPYHU}{\rd = \rx, \ry}
\multiplier{op1sel}{WORD}
\multiplier{op1unsigned}{'1'}
\multiplier{op2sel}{HIGH_HALF}
\multiplier{op2unsigned}{'1'}
\multiplier{resultSel}{PASS}
Unsigned 16x32-bit to 32-bit multiplication. \code{\rx} is the 32-bit operand, 
the upper halfword of \code{\ry} is the 16-bit operand. The upper 16 bits of the 
multiplication result are discarded.

\begin{lstlisting}[numbers=none, basicstyle=\ttfamily\footnotesize, language=C++]
\rd = (unsigned int)\rx * (unsigned short)(\ry >> 16);
\end{lstlisting}

%-------------------------------------------------------------------------------
\syllable{00001010-}{MPYHS}{\rd = \rx, \ry}
\multiplier{op1sel}{WORD}
\multiplier{op2sel}{HIGH_HALF}
\multiplier{resultSel}{SHL16}
Signed 16x32-bit to 32-bit multiplication. \code{\rx} is the 32-bit operand, the 
upper halfword of \code{\ry} is the 16-bit operand. The result is shifted left
by 16, discarding the upper 32 bits of the 48-bit result.

This operation may be used as a primitive for 32x32-bit to 32-bit
multiplication, computing the partial product of \code{\rx} and the upper half
of \code{\ry}. \insn{MPYLU} is then used to compute the other partial product. A
final \insn{ADD} instruction combines the partial products.

\begin{lstlisting}[numbers=none, basicstyle=\ttfamily\footnotesize, language=C++]
\rd = ((int)\rx * (short)\ry) << 16;
\end{lstlisting}

%-------------------------------------------------------------------------------
\syllable{10010010-}{MPYLHUS}{\rd = \rx, \ry}
\multiplier{op1sel}{WORD}
\multiplier{op2sel}{LOW_HALF}
\multiplier{op2unsigned}{'1'}
\multiplier{resultSel}{SHR32}
Mixed 16x32-bit to 32-bit multiplication. \code{\rx} is the 32-bit operand,
interpreted as a signed number. \code{\ry} is the 16-bit operand, interpreted as
an unsigned number. The 48-bit result is shifted right by 32 bits.

Together with \insn{MPYHS}, \insn{MPYLU}, \insn{MPYHHS}, \insn{ADD} and
\insn{ADDCG}, a full signed 32x32-bit to 64-bit multiplication may be realized
as follows.

\begin{lstlisting}[numbers=none, basicstyle=\ttfamily\footnotesize, language=vexasm]
mpylu   low1  = op1, op2
mpyhs   low2  = op1, op2
;;
mpylhus high1 = op1, op2
mpyhhs  high2 = op1, op2
;;
addcg   low, carry = <zero>, low1, low2
;;
addcg   high, carry = carry, high1, high2
;;
\end{lstlisting}

\noindent The first two multiply instructions and the first \insn{ADDCG} compute
the low word of the multiplication and carry bit for the high word. The
remaining instructions do the same for the high part of the result.

\begin{lstlisting}[numbers=none, basicstyle=\ttfamily\footnotesize, language=C++]
\rd = ((long long)\rx * (short)\ry) >> 32;
\end{lstlisting}

%-------------------------------------------------------------------------------
\syllable{10010011-}{MPYHHS}{\rd = \rx, \ry}
\multiplier{op1sel}{WORD}
\multiplier{op2sel}{HIGH_HALF}
\multiplier{resultSel}{SHR16}
Signed 16x32-bit to 32-bit multiplication. \code{\rx} is the 32-bit operand,
the upper halfword of \code{\ry} is the 16-bit operand. The 48-bit result is
shifted right by 16 bits.

This syllable is used in conjunction with other multiplication syllables for a
32x32-bit to 64-bit signed multiplication. Refer to \insn{MPYLHUS} for more
information.

\begin{lstlisting}[numbers=none, basicstyle=\ttfamily\footnotesize, language=C++]
\rd = ((long long)\rx * (short)\ry) >> 32;
\end{lstlisting}
