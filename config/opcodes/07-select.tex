
%===============================================================================
\section{ALU selection instructions}
%===============================================================================
\datapath{gpRegWE}{'1'}
\datapath{brFmt}{'1'}
\datapath{isNOP}{'0'}
\alu{op1Mux}{EXTEND32INV}
\alu{op2Mux}{EXTEND32}
\alu{opBrMux}{TRUE}
\alu{intResultMux}{OP_SEL}

\datapath{brFmt}{'1'} % Only because legacy decoding did this.
\alu{bitwiseOp}{BITW_AND} % Only because legacy decoding did this.

The \rvex{} ALU has single-cycle instructions for conditional moves and
computation of the minimum and maximum of two integer values.

%-------------------------------------------------------------------------------
\syllable{00111----}{SLCT}{\rd = \bs, \rx, \ry}
\alu{op1Mux}{EXTEND32}
\alu{op2Mux}{EXTEND32}
\alu{opBrMux}{PASS}
Conditional move.

\begin{lstlisting}[numbers=none, basicstyle=\ttfamily\footnotesize, language=C++]
\rd = \bs ? \rx : \ry;
\end{lstlisting}

%-------------------------------------------------------------------------------
\syllable{00110----}{SLCTF}{\rd = \bs, \rx, \ry}
\alu{op1Mux}{EXTEND32}
\alu{op2Mux}{EXTEND32}
\alu{opBrMux}{INVERT}
Conditional move, with operands swapped with respect to \insn{SLCT}.

Notice that the immediate version of \insn{SLCTF} reduces to a move from a
branch register to an integer register when \code{x} is 0 and \code{y} is 1.

\begin{lstlisting}[numbers=none, basicstyle=\ttfamily\footnotesize, language=C++]
\rd = \bs ? \ry : \rx;
\end{lstlisting}

%-------------------------------------------------------------------------------
\syllable{01100101-}{MAX}{\rd = \rx, \ry}
\alu{compare}{'1'}
\alu{brResultMux}{CMP_GE}
Computes maximum of the input operands using signed arithmetic.

\begin{lstlisting}[numbers=none, basicstyle=\ttfamily\footnotesize, language=C++]
\rd = (\rx >= \ry) : \rx ? \ry;
\end{lstlisting}

%-------------------------------------------------------------------------------
\syllable{01100110-}{MAXU}{\rd = \rx, \ry}
\alu{compare}{'1'}
\alu{unsignedOp}{'1'}
\alu{brResultMux}{CMP_GE}
Computes maximum of the input operands using unsigned arithmetic.

\begin{lstlisting}[numbers=none, basicstyle=\ttfamily\footnotesize, language=C++]
\rd = ((unsigned int)\rx >= (unsigned int)\ry) : \rx ? \ry;
\end{lstlisting}

%-------------------------------------------------------------------------------
\syllable{01100111-}{MIN}{\rd = \rx, \ry}
\alu{compare}{'1'}
\alu{brResultMux}{CMP_LE}
Computes minimum of the input operands using signed arithmetic.

\begin{lstlisting}[numbers=none, basicstyle=\ttfamily\footnotesize, language=C++]
\rd = (\rx <= \ry) : \rx ? \ry;
\end{lstlisting}

%-------------------------------------------------------------------------------
\syllable{01101000-}{MINU}{\rd = \rx, \ry}
\alu{compare}{'1'}
\alu{unsignedOp}{'1'}
\alu{brResultMux}{CMP_LE}
Computes minimum of the input operands using unsigned arithmetic.

\begin{lstlisting}[numbers=none, basicstyle=\ttfamily\footnotesize, language=C++]
\rd = ((unsigned int)\rx <= (unsigned int)\ry) : \rx ? \ry;
\end{lstlisting}
