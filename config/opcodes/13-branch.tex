
%===============================================================================
\section{Branch instructions}
%===============================================================================
\class{BR}
\datapath{isNOP}{'0'}
\branch{isBranchInstruction}{'1'}
\branch{branchIfTrue}{'1'}
\branch{branchIfFalse}{'1'}

The highest-indexed pipelane in every \rvex{} system (i.e., the pipelane that
the last syllable in a bundle maps to) contains a branch unit. This unit
supports the flow control operations outlined below.

Branch offsets are signed immediates relative to the next program counter
(\code{PC+1}). Because there are certain alignment requirements to program
counters, the lower two or three bits of the offset are not actually included
in the bitfield. Whether this value is two or three depends on the value of the
\code{BRANCH_OFFS_SHIFT} constant defined in \code{core_intIface_pkg.vhd}; it
is three by default. It must be set to two to support branching to the start of
any bundle when stop bits are fully enabled. This must then also be updated in
the assembler.

Note that branch offsets and the stack adjust immediate are not eligible for 
long immediate instructions.

%-------------------------------------------------------------------------------
\syllable{00100000-}{GOTO}{\of}
Branches to \code{PC+1} + \code{\of} unconditionally.

\begin{lstlisting}[numbers=none, basicstyle=\ttfamily\footnotesize, language=C++]
PCP1 += \of;
\end{lstlisting}

%-------------------------------------------------------------------------------
\syllable{00100001-}{IGOTO}{\lr}
\branch{branchToLink}{'1'}
Branches to the address in \code{\lr} unconditionally. This is used for branches 
to code that cannot be reached using the branch offset immediate.

\begin{lstlisting}[numbers=none, basicstyle=\ttfamily\footnotesize, language=C++]
PCP1 = \lr;
\end{lstlisting}

%-------------------------------------------------------------------------------
\syllable{00100010-}{CALL}{\lr = \of}
\datapath{funcSel}{PCP1}
\datapath{linkWE}{'1'}
\branch{link}{'1'}
Branches to \code{PC+1} + \code{\of} unconditionally, while storing \code{PC+1} 
in the link register. This is used for function calls.

\begin{lstlisting}[numbers=none, basicstyle=\ttfamily\footnotesize, language=C++]
\lr = PCP1;
PCP1 += \of;
\end{lstlisting}

%-------------------------------------------------------------------------------
\syllable{00100011-}{ICALL}{\lr = \lr}
\datapath{funcSel}{PCP1}
\datapath{linkWE}{'1'}
\branch{branchToLink}{'1'}
\branch{link}{'1'}
Branches to the address in \code{\lr} unconditionally, while storing \code{PC+1} 
in the link register. In other words, it essentially swaps \code{PC+1} and 
\code{\lr}. This is used for dynamic function calls or calls to functions that 
cannot be reached using the branch offset immediate method.

\begin{lstlisting}[numbers=none, basicstyle=\ttfamily\footnotesize, language=C++]
unsigned int tmp = \lr;
\lr = PCP1;
PCP1 = tmp;
\end{lstlisting}

%-------------------------------------------------------------------------------
\syllable{00100100-}{BR}{\bs, \of}
\branch{branchIfFalse}{'0'}
Branches to \code{PC+1} + \code{\of} only if \code{\bs} is true. This 
instruction performs no operation if \code{\bs} is false.

\begin{lstlisting}[numbers=none, basicstyle=\ttfamily\footnotesize, language=C++]
PCP1 += \bs ? \of : 0;
\end{lstlisting}

%-------------------------------------------------------------------------------
\syllable{00100101-}{BRF}{\bs, \of}
\branch{branchIfTrue}{'0'}
Branches to \code{PC+1} + \code{\of} only if \code{\bs} is false. This 
instruction performs no operation if \code{\bs} is true.

\begin{lstlisting}[numbers=none, basicstyle=\ttfamily\footnotesize, language=C++]
PCP1 += \bs ? 0 : \of;
\end{lstlisting}

%-------------------------------------------------------------------------------
\syllable{00100110-}{RETURN}{\rs = \rs, \sa, \lr}
\datapath{gpRegWE}{'1'}
\datapath{stackOp}{'1'}
\datapath{gpRegRdEnaA}{'1'}
\alu{op1Mux}{EXTEND32}
\alu{op2Mux}{EXTEND32}
\alu{opBrMux}{FALSE}
\alu{bitwiseOp}{BITW_AND} % Only because legacy decoding did this.
\alu{intResultMux}{ADDER}
\branch{branchToLink}{'1'}
Returns from a function by branching to \code{\lr} unconditionally, while adding 
\code{\sa} to \code{\rs}. \code{\sa} is interpreted as a signed immediate. This
allows final stack pointer adjustment and returning to be done with a single
syllable.

Notice that this instruction is identical to \insn{IGOTO}, except for the fact 
that \insn{IGOTO} does not access \code{\rs}.

\begin{lstlisting}[numbers=none, basicstyle=\ttfamily\footnotesize, language=C++]
\rs += \sa;
PCP1 = \lr;
\end{lstlisting}

%-------------------------------------------------------------------------------
\syllable{00100111-}{RFI}{\rs = \rs, \sa}
\datapath{gpRegWE}{'1'}
\datapath{stackOp}{'1'}
\datapath{gpRegRdEnaA}{'1'}
\alu{op1Mux}{EXTEND32}
\alu{op2Mux}{EXTEND32}
\alu{opBrMux}{FALSE}
\alu{bitwiseOp}{BITW_AND} % Only because legacy decoding did this.
\alu{intResultMux}{ADDER}
\branch{branchIfTrue}{'0'}
\branch{branchIfFalse}{'0'}
\branch{RFI}{'1'}
Returns from a trap service routine by branching to \creg{TP} unconditionally 
and restoring \creg{SCCR} to \creg{CCR}, while adding \code{\sa} to \code{\rs}. 
\code{\sa} is interpreted as a signed immediate. This allows final stack pointer 
adjustment and returning to be done with a single syllable.

\begin{lstlisting}[numbers=none, basicstyle=\ttfamily\footnotesize, language=C++]
\rs += \sa;
CR_CCR = CR_SCCR;
PCP1 = CR_TP;
\end{lstlisting}

%-------------------------------------------------------------------------------
\syllable{00101000-}{STOP}{}
\branch{branchIfTrue}{'0'}
\branch{branchIfFalse}{'0'}
\branch{stop}{'1'}
Causes a \trap{STOP} trap to occur during execution of the next instruction.
The \trap{STOP} trap will cause the B flag in \creg{DCR} to be set, which will
stop execution. Thus, the processor will be stopped after the bundle in which
the \insn{STOP} instruction resides is executed.
