
%===============================================================================
\register{CMC1}{Cache/MMU control 1}{0x2A0}
%===============================================================================

This register contains various cache and MMU control flags. Note that not all
platforms include the cache and MMU, in which case the respective registers are
no-operation.

%-------------------------------------------------------------------------------
\field{31..24}{BLP}
%-------------------------------------------------------------------------------
This field allows the application to prioritize updating certain cache blocks or 
TLBs when a miss occurs. This potentially decreases the cache penalty of 
reconfiguration if the next configuration is known in advance, as it allows 
directing new data to the blocks that are known to be shared in the next 
configuration.

Each bit in this field maps to a lane group/cache block/TLB. Cache blocks/TLBs
for which this bit is set have a higher priority for servicing misses than
those for which the bit is cleared.

% - - - - - - - - - - - - - - - - - - - - - - - - - - - - - - - - - - - - - - -
\declaration{}
\declRegister{_r}{byte}{0}
\implementation{}
\begin{lstlisting}

// Handle bus interface.
_r = (_r & ~_wmask) | (_write & _wmask);
_read = _r;

\end{lstlisting}

%-------------------------------------------------------------------------------
\field{19..18}{WTC}
%-------------------------------------------------------------------------------
This field selects whether the write-to-clean trap is enabled or disabled. This
trap may be used to implement copy-on-write behavior.

When \trap{WRITE_TO_CLEAN_PAGE} is enabled, this register reads as \code{01}. 
When it is disabled, it reads as \code{10}. Writing \code{00} has no effect, 
writing \code{01} enables the trap, writing \code{10} disables it and writing 
\code{11} toggles the state. This prevents the need for read-modify-write 
operations.

% - - - - - - - - - - - - - - - - - - - - - - - - - - - - - - - - - - - - - - -
\reset{10}
\declaration{}
\declVariable{_v}{bit}{0}
\declRegister{_r}{bit}{0}
\declVariable{_cmd}{bitvec2}{0}
\implementation{}
\begin{lstlisting}
_v = _r;

// Handle bus interface.
_cmd = _wmask & _write;
if (_cmd == "11") _v = ~_v; else _v = (_v | _cmd[0]) & ~_cmd[1];
_read = ~_r $ _r;

_r = _v;
\end{lstlisting}

%-------------------------------------------------------------------------------
\field{16}{T}
%-------------------------------------------------------------------------------
This flag controls TLB flushing. Writing a \code{1} to this register initiates a
flush, writing \code{0} has no effect. This register reads as \code{1} while the
flush is pending or in progress. \creg{TFHI}, \creg{TFLO}, and \creg{TFID} may
be used to restrict the TLB flush to only entries that meet certain criteria.

% - - - - - - - - - - - - - - - - - - - - - - - - - - - - - - - - - - - - - - -
\declaration{}
\declRegister{_r}{bit}{0}
\declRegister{_s}{bit}{0}
\declRegister{_br}{bit}{0}
\implementation{}
\begin{lstlisting}

if (_write & _wmask) {
  _r = 1;
  _s = 1;
} else {
  _r = mem2cxreg_mmuStatus.flush_busy;
  _s = 0;
}
_br = mem2cxreg_mmuStatus.flush_busy;
_read = _r;

\end{lstlisting}

%-------------------------------------------------------------------------------
\field{11..10}{BYP}
%-------------------------------------------------------------------------------
This field may be used to force all data accesses to bypass the cache. When this 
is the case, this register reads as \code{01}. Otherwise, it reads as \code{10}. 
Writing \code{00} has no effect, writing \code{01} enables the bypass, writing 
\code{10} disables it and writing \code{11} toggles the state. This prevents the 
need for read-modify-write operations.

% - - - - - - - - - - - - - - - - - - - - - - - - - - - - - - - - - - - - - - -
\reset{10}
\declaration{}
\declVariable{_v}{bit}{0}
\declRegister{_r}{bit}{0}
\declVariable{_cmd}{bitvec2}{0}
\implementation{}
\begin{lstlisting}
_v = _r;

// Handle bus interface.
_cmd = _wmask & _write;
if (_cmd == "11") _v = ~_v; else _v = (_v | _cmd[0]) & ~_cmd[1];
_read = ~_r $ _r;

_r = _v;
\end{lstlisting}

%-------------------------------------------------------------------------------
\field{8}{D}
%-------------------------------------------------------------------------------
This flag controls data cache flushing. Writing a \code{1} to this register 
initiates a flush, writing \code{0} has no effect. This register reads as 
\code{1} while the flush is pending or in progress.

% - - - - - - - - - - - - - - - - - - - - - - - - - - - - - - - - - - - - - - -
\declaration{}
\declRegister{_r}{bit}{0}
\declRegister{_s}{bit}{0}
\implementation{}
\begin{lstlisting}

if (_write & _wmask) {
  _r = 1;
  _s = 1;
} else {
  _r = mem2cxreg_cacheStatus.data_flushBusy;
  _s = 0;
}
_read = _r;

\end{lstlisting}

%-------------------------------------------------------------------------------
\field{0}{I}
%-------------------------------------------------------------------------------
This flag controls instruction cache flushing. Writing a \code{1} to this 
register initiates a flush, writing \code{0} has no effect. This register reads 
as \code{1} while the flush is pending or in progress.

% - - - - - - - - - - - - - - - - - - - - - - - - - - - - - - - - - - - - - - -
\declaration{}
\declRegister{_r}{bit}{0}
\declRegister{_s}{bit}{0}
\implementation{}
\begin{lstlisting}

if (_write & _wmask) {
  _r = 1;
  _s = 1;
} else {
  _r = mem2cxreg_cacheStatus.instr_flushBusy;
  _s = 0;
}
_read = _r;

\end{lstlisting}

%===============================================================================
\register{CMC2}{Cache/MMU control 2}{0x2A4}
%===============================================================================

This register is reserved for future expansions.

%===============================================================================
\register{PTP}{Page table pointer}{0x2A8}
%===============================================================================

This register holds the base address to the page table as used by the MMU.
Writing to this register while the MMU is enabled leads to undefined behavior.
In any case, a full TLB flush is needed after writing to this register.

% - - - - - - - - - - - - - - - - - - - - - - - - - - - - - - - - - - - - - - -
\field{31..0}{PTP}
\declaration{}
\declRegister{_r}{address}{0}
\implementation{}
\begin{lstlisting}

// Handle bus interface.
_r = (_r & ~_wmask) | (_write & _wmask);
_read = _r;

\end{lstlisting}

%===============================================================================
\register{ASID}{Address space ID}{0x2AC}
%===============================================================================

This register holds the address space ID for the currently running task as used
by the MMU. The address space ID is used by the TLB to distinguish between
entries from different tasks, preventing the need to flush the TLB after every
context switch. Only the $n$ LSBs of this register are actually used. The value
of $n$ is design-time configurable and may be read from the TODO field of TODO
\todo{ASID width version register}. It is 10 by default.

% - - - - - - - - - - - - - - - - - - - - - - - - - - - - - - - - - - - - - - -
\field{31..0}{ASID}
\declaration{}
\declRegister{_r}{data}{0}
\implementation{}
\begin{lstlisting}

// Handle bus interface.
_r = (_r & ~_wmask) | (_write & _wmask);
_read = _r;

\end{lstlisting}

%===============================================================================
\register{TFLO}{TLB flush tag low}{0x2B0}
%===============================================================================

This register specifies the inclusive lower limit for the range of virtual tags 
to match when flushing the TLB. When the processor is reset or a flush 
completes, this register is reset to zero.

% - - - - - - - - - - - - - - - - - - - - - - - - - - - - - - - - - - - - - - -
\field{31..0}{TFLO}
\reset{00000000000000000000000000000000}
\declaration{}
\declRegister{_r}{address}{0}
\implementation{}
\begin{lstlisting}

// Handle bus interface.
_r = (_r & ~_wmask) | (_write & _wmask);
_read = _r;

// Reset the register when a flush completes.
if (CR_CMC1_T_br && !mem2cxreg_mmuStatus.flush_busy) {
  _r = 0;
}

\end{lstlisting}

%===============================================================================
\register{TFHI}{TLB flush tag high}{0x2B4}
%===============================================================================

This register specifies the inclusive upper limit for the range of virtual tags 
to match when flushing the TLB. When the processor is reset or a flush
completes, this register is reset to \code{0xFFFFFFFF}.

% - - - - - - - - - - - - - - - - - - - - - - - - - - - - - - - - - - - - - - -
\field{31..0}{TFHI}
\reset{11111111111111111111111111111111}
\declaration{}
\declRegister{_r}{address}{X"FFFFFFFF"}
\implementation{}
\begin{lstlisting}

// Handle bus interface.
_r = (_r & ~_wmask) | (_write & _wmask);
_read = _r;

// Reset the register when a flush completes.
if (CR_CMC1_T_br && !mem2cxreg_mmuStatus.flush_busy) {
  _r = X"FFFFFFFF";
}

\end{lstlisting}

%===============================================================================
\register{TFID}{TLB flush address space ID}{0x2B8}
%===============================================================================

This register allows TLB flushes to be restricted to a single address space ID.

%-------------------------------------------------------------------------------
\field{31}{E}
%-------------------------------------------------------------------------------
This flag selects whether address space ID matching is performed during a TLB 
flush or not. When this bit is set to 1, the address space ID of a TLB entry is 
ignored. When it is set to 0, only TLB entries with an address space ID matching 
the value stored in the ID field are flushed. When the processor is reset or a
flush completes, this flag is reset to 1.

% - - - - - - - - - - - - - - - - - - - - - - - - - - - - - - - - - - - - - - -
\reset{1}
\declaration{}
\declRegister{_r}{bit}{1}
\implementation{}
\begin{lstlisting}

// Handle bus interface.
_r = (_r & ~_wmask) | (_write & _wmask);
_read = _r;

// Reset the register when a flush completes.
if (CR_CMC1_T_br && !mem2cxreg_mmuStatus.flush_busy) {
  _r = 1;
}

\end{lstlisting}

%-------------------------------------------------------------------------------
\field{30..0}{ASID}
%-------------------------------------------------------------------------------
This field stores the address space ID that TLB entries should be matched
against during a TLB flush.

% - - - - - - - - - - - - - - - - - - - - - - - - - - - - - - - - - - - - - - -
\declaration{}
\declRegister{_r}{data}{0}
\implementation{}
\begin{lstlisting}

// Handle bus interface.
_r = '0' $ ((_r[0, 31] & ~_wmask) | (_write & _wmask));
_read = _r[0, 31];

\end{lstlisting}

%===============================================================================
% Connect the cache and MMU control records
%===============================================================================

% These must unfortunately be one-liners. LAC cannot deal with writing to these
% entries separately.

\connect{cxreg2mem_cacheControl}{{ instr_flushStart=>CR_CMC1_I_s, data_flushStart=>CR_CMC1_D_s, data_bypass=>CR_CMC1_BYP_r, blockPrio=>CR_CMC1_BLP_r }}
\connect{cxreg2mem_mmuControl}{{ enable=>CR_CCR_M_r, kernelMode=>CR_CCR_K_r, writeToCleanEna=>CR_CMC1_WTC_r, pageTablePtr=>CR_PTP_PTP_r, asid=>CR_ASID_ASID_r, flush_start=>CR_CMC1_T_s, flush_asid=>CR_TFID_ASID_r, flush_asidEna=>CR_TFID_E_r, flush_tagLow=>CR_TFLO_TFLO_r, flush_tagHigh=>CR_TFHI_TFHI_r, blockPrio=>CR_CMC1_BLP_r }}
