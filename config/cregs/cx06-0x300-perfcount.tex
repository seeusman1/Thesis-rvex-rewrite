%===============================================================================
\register{CYC}{Cycle counter}{0x300}
%===============================================================================

This register will automatically increment every cycle while an instruction from
this context is in the pipeline, even when the context is stalled.

When the debug bus writes to this register, the register is cleared. When it
writes 1 to it, \creg{STALL}, \creg{BUN}, \creg{SYL} and \creg{NOP} will also be
cleared, simultaneously.

\field{31..0}{Cycle counter}
\debugCanWrite{}

%===============================================================================
\register{STALL}{Stall cycle counter}{0x308}
%===============================================================================

This register will automatically increment every cycle while an instruction from 
this context is in the pipeline and the context is stalled. When the debug bus 
writes to this register, the register is cleared. It will also be cleared if the 
debug bus writes a 1 to \creg{CYC}.

As long as neither \creg{CYC} nor \creg{STALL} have overflowed,
\creg{CYC}~-~\creg{STALL} represents the number of active cycles.

\field{31..0}{Stall cycle counter}
\debugCanWrite{}

%===============================================================================
\register{BUN}{Committed bundle counter}{0x310}
%===============================================================================

This register will automatically increment whenever the results of executing a 
bundle are committed. When the debug bus writes to this register, the register 
is cleared. It will also be cleared if the debug bus writes a 1 to \creg{CYC}.

As long as neither \creg{CYC}, \creg{STALL} nor \creg{BUN} have overflowed,
\creg{CYC}~-~\creg{STALL}~-~\creg{BUN} represents the number of cycles spent
doing pipeline flushes, for example due to traps or the branch delay slot.

\field{31..0}{Committed bundle counter}
\debugCanWrite{}

%===============================================================================
\register{SYL}{Committed syllable counter}{0x318}
%===============================================================================

This register will automatically increment whenever the results of executing a 
non-\insn{NOP} syllable are committed. When the debug bus writes to this 
register, the register is cleared. It will also be cleared if the debug bus 
writes a 1 to \creg{CYC}.

As long as neither \creg{BUN} nor \creg{SYL} have overflowed,
\creg{SYL}~/~\creg{BUN} represents average instruction-level parallelism since
the registers were cleared.

\field{31..0}{Committed syllable counter}
\debugCanWrite{}

%===============================================================================
\register{NOP}{Committed NOP counter}{0x320}
%===============================================================================

This register will automatically increment whenever a \insn{NOP} syllable is 
committed. When the debug bus writes to this register, the register is cleared. 
It will also be cleared if the debug bus writes a 1 to \creg{CYC}.

As long as neither \creg{SYL} nor \creg{NOP} have overflowed,
\creg{SYL}~/~(\creg{SYL} + \creg{NOP}) represents average fraction of syllables
that are \insn{NOP}, i.e. the compression efficiency of the binary.

\field{31..0}{Committed NOP counter}
\debugCanWrite{}

%===============================================================================
\register{IACC}{Instruction cache access counter}{0x328}
%===============================================================================

This register increments for every instruction cache access.

\field{31..0}{Instruction cache access counter}
\debugCanWrite{}

%===============================================================================
\register{IMISS}{Instruction cache miss counter}{0x330}
%===============================================================================

This register increments every time there is a miss in the instruction cache.

\field{31..0}{Instruction cache miss counter}
\debugCanWrite{}

%===============================================================================
\register{DRACC}{Data cache read access counter}{0x338}
%===============================================================================

This register increments every time there is a read access to the data cache.

\field{31..0}{Data cache read access counter}
\debugCanWrite{}

%===============================================================================
\register{DRMISS}{Data cache read miss counter}{0x340}
%===============================================================================

This register increments every time there is a read miss in the data cache.

\field{31..0}{Data cache read miss counter}
\debugCanWrite{}

%===============================================================================
\register{DWACC}{Data cache write access counter}{0x348}
%===============================================================================

This register increments every time there is a write access to the data cache.

\field{31..0}{Data cache write access counter}
\debugCanWrite{}

%===============================================================================
\register{DWMISS}{Data cache write miss counter}{0x350}
%===============================================================================

This register increments every time there is a write miss in the data cache.

\field{31..0}{Data cache write miss counter}
\debugCanWrite{}

%===============================================================================
\register{DBYPASS}{Data cache bypass counter}{0x358}
%===============================================================================

This register increments every time there is a bypassed access to the data cache.

\field{31..0}{Data cache bypass counter}
\debugCanWrite{}

%===============================================================================
\register{DWBUF}{Data cache write buffer counter}{0x360}
%===============================================================================

This register increments every time there is a write to the data cache while
the write buffer is full.

\field{31..0}{Data cache write buffer counter}
\debugCanWrite{}
