
\contextInterface{}

%===============================================================================
\ifaceGroup{Run control interface}
%===============================================================================
\ifaceSubGroup{}
\ifaceOutCtxt{rv}{reset}{bit}{0}
Active high context reset output. This should be wired to the ctxtReset signal
externally.
    
\ifaceSubGroup{}
\ifaceInCtxt{rctrl}{resetVect}{address}
Reset vector. When the context or the entire core is reset, the PC register will 
be set to this value.
    
\ifaceSubGroup{}
\ifaceOutCtxt{rctrl}{done}{bit}{0}
Active high done output. This is asserted when the context encounters a stop 
syllable. Processing a stop signal also sets the BRK control register, which 
stops the core. This bit can be reset by issuing a core reset or by means of the 
debug interface.

%===============================================================================
\ifaceGroup{Memory interface: trace information}
%===============================================================================
\ifaceSubGroup{}
\ifaceInCtxt{imem}{access}{bit}
\ifaceInCtxt{imem}{miss}{bit}
\ifaceInCtxt{imem}{tlbAccess}{bit}
\ifaceInCtxt{imem}{tlbMiss}{bit}
\ifaceInCtxt{imem}{tlbMispredict}{bit}
\ifaceInCtxt{dmem}{accessType}{twobit}
\ifaceInCtxt{dmem}{bypass}{bit}
\ifaceInCtxt{dmem}{miss}{bit}
\ifaceInCtxt{dmem}{writePending}{bit}
\ifaceInCtxt{dmem}{tlbAccess}{bit}
\ifaceInCtxt{dmem}{tlbMiss}{bit}
\ifaceInCtxt{dmem}{tlbMispredict}{bit}
Cache/MMU performance signals. These don't really belong here because they're
per lane group, but the cache does not have a register interface yet.

%===============================================================================
\ifaceGroup{Memory interface: status/control}
%===============================================================================

\ifaceSubGroup{}
\ifaceOutCtxt{imem}{flushStart}{bit}{0}
\ifaceInCtxt{imem}{flushBusy}{bit}
Instruction cache flush control. When flushStart is asserted high, a flush
should be performed. flushBusy should be high while a flush is in progress. If
a flush is single-cycle, it may stay low.

\ifaceSubGroup{}
\ifaceOutCtxt{dmem}{flushStart}{bit}{0}
\ifaceInCtxt{dmem}{flushBusy}{bit}
Data cache flush control. When flushStart is asserted high, a flush should be 
performed. flushBusy should be high while a flush is in progress. If a flush is 
single-cycle, it may stay low.

\ifaceSubGroup{}
\ifaceOutCtxt{dmem}{bypass}{bit}{0}
Data cache bypass override. When high, all accesses should bypass the cache.

\ifaceSubGroup{}
\ifaceOutCtxt{mem}{mmuEnable}{bit}{0}
This signal controls whether address translation is active or not.

\ifaceSubGroup{}
\ifaceOutCtxt{mem}{kernelMode}{bit}{0}
This signal represents the current privilege level of processor. It is
high for kernel mode and low for application mode.

\ifaceSubGroup{}
\ifaceOutCtxt{mem}{writeToCleanEna}{bit}{0}
This signal controls whether a trap is generated when a write to a clean
page is attempted.

\ifaceSubGroup{}
\ifaceOutCtxt{mem}{writeProtect}{bit}{0}
This signal controls whether kernel threads can write to read-only pages. This 
is the case when this signal is low.
    
\ifaceSubGroup{}
\ifaceOutCtxt{mem}{globalPageEna}{bit}{0}
This signal specifies whether the global page bit in the page table is enabled 
or ignored.
    
\ifaceSubGroup{}
\ifaceOutCtxt{mem}{execPageEna}{bit}{0}
This signal specifies whether the executable page bit in the page table is 
enabled or ignored.

\ifaceSubGroup{}
\ifaceOutCtxt{mem}{pageTablePtr}{address}{0}
This signal specifies the page table pointer for the current thread.

\ifaceSubGroup{}
\ifaceOutCtxt{mem}{asid}{data}{0}
This signal specifies the address space ID for the current thread.

\ifaceSubGroup{}
\ifaceOutCtxt{mem}{tlbFlushStart}{bit}{0}
\ifaceInCtxt{mem}{tlbFlushBusy}{bit}
TLB flush control. When flushStart is asserted high, a flush should be 
performed. flushBusy should be high while a flush is in progress. If a flush is 
single-cycle, it may stay low.

\ifaceSubGroup{}
\ifaceOutCtxt{mem}{tlbFlushAsidEna}{bit}{0}
\ifaceOutCtxt{mem}{tlbFlushAsid}{data}{0}
When tlbFlushAsidEna is high, tlbFlushAsid specifies a specific ASID that must 
be flushed during a TLB flush. Entries with other ASIDs are then unaffected.

\ifaceSubGroup{}
\ifaceOutCtxt{mem}{tlbFlushTagLow}{address}{0}
\ifaceOutCtxt{mem}{tlbFlushTagHigh}{address}{0}
These two signals specify a lower and upper limit for the virtual page
addresses that are to be flushed. Both are inclusive.

\ifaceSubGroup{}
\ifaceOutCtxt{mem}{blockPrio}{byte}{0}
This signal is used to assign a priority to certain blocks in the cache/TLB
replacement policy. The vector is indexed by lane group; when a bit is high, the
associated cache/TLB blocks should get a higher update priority than those for
which the bit is low.

%===============================================================================
\ifaceGroup{Pipelane interface: misc}
%===============================================================================
\ifaceSubGroup{}
\ifaceInCtxt{cxplif}{stall}{bit}
When high, the context registers must maintain their current value.

\ifaceSubGroup{}
\ifaceInCtxt{cxplif}{idle}{bit}
Idle flag, as reported to the external run control interface. Used for the 
performance counters.

\ifaceSubGroup{}
\ifaceInCtxt{cxplif}{sylCommit}{sylStatus}
Syllable committed flag for each lane, used for the performance counters.
    
\ifaceSubGroup{}
\ifaceInCtxt{cxplif}{sylNop}{sylStatus}
NOP flag for each lane with the same timing as sylCommit, used for the 
performance counters.

\ifaceSubGroup{}
\ifaceInCtxt{cxplif}{stop}{bit}
Stop flag. When high, the BRK and done flags in the debug control
register should be set.
    
%===============================================================================
\ifaceGroup{Pipelane interface: branch/link registers}
%===============================================================================
\ifaceSubGroup{}
\ifaceInCtxt{cxplif}{brWriteData}{brRegData}
\ifaceInCtxt{cxplif}{brWriteEnable}{brRegData}
Write data and enable signal for each branch register.
    
\ifaceSubGroup{}
\ifaceOutCtxt{cxplif}{brReadData}{brRegData}{0}
Current state of the branch registers.
    
\ifaceSubGroup{}
\ifaceInCtxt{cxplif}{linkWriteData}{data}
\ifaceInCtxt{cxplif}{linkWriteEnable}{bit}
Write data and enable signal for the link register.
    
\ifaceSubGroup{}
\ifaceOutCtxt{cxplif}{linkReadData}{data}{0}
Current state of the link register.

%===============================================================================
\ifaceGroup{Pipelane interface: program counter}
%===============================================================================
\ifaceSubGroup{}
\ifaceInCtxt{cxplif}{nextPC}{address}
Next value for the PC register. This is written when stall is low and overridePC 
is not asserted.

\ifaceSubGroup{}
\ifaceOutCtxt{cxplif}{currentPC}{address}{rctrl2cxreg_resetVect}
Current value of the PC register.

\ifaceSubGroup{}
\ifaceOutCtxt{cxplif}{overridePC}{bit}{0}
\ifaceInCtxt{cxplif}{overridePC_ack}{bit}
The overridePC flag is set when the debug bus writes to the context registers or 
when the context or processor is reset. This is reset when overridePC_ack is 
asserted while stall is low. It indicates to the branch unit that it should 
inject a branch to the current PC register regardless of the current instruction 
or state.

%===============================================================================
\ifaceGroup{Pipelane interface: trap handling}
%===============================================================================
\ifaceSubGroup{}
\ifaceOutCtxt{cxplif}{trapHandler}{address}{0}
Current trap handler. When the application has marked that it is not currently 
capable of accepting a trap, this is set to the panic handler register instead.

\ifaceSubGroup{}
\ifaceInCtxt{cxplif}{trapInfo}{trapInfo}
\ifaceInCtxt{cxplif}{trapPoint}{address}
\ifaceInCtxt{cxplif}{trapIsDebug}{bit}
Regular trap information. When the trap in trapInfo is active, the trap 
information should be stored in the trap cause/arg registers. In addition, the 
register hardware should save the current value of the control register and 
should clear the ready-for-trap and interrupt- enable bits, as well as the 
debug-trap-enable bit if the trap cause maps to a debug trap.

\ifaceSubGroup{}
\ifaceOutCtxt{cxplif}{trapReturn}{address}{0}
Connected to the current value of the trap point register. Used by the branch 
unit as the return address for the RFI instruction.

\ifaceSubGroup{}
\ifaceInCtxt{cxplif}{rfi}{bit}
RFI flag. When high, the saved control register value should be restored and the 
trap cause field should be set to 0.

\ifaceSubGroup{}
\ifaceOutCtxt{cxplif}{handlingDebugTrap}{bit}{0}
Set when the current value of the trap cause register maps to a debug trap. This 
is used by the branch unit to disable breakpoints for the first instruction 
executed after the debug trap returns.

\ifaceSubGroup{}
\ifaceOutCtxt{cxplif}{interruptEnable}{bit}{0}
Current value of the interrupt-enable flag in the control register.

\ifaceSubGroup{}
\ifaceOutCtxt{cxplif}{debugTrapEnable}{bit}{0}
Current value of the debug-trap-enable flag in the control register.

\ifaceSubGroup{}
\ifaceOutCtxt{cxplif}{softCtxtSwitch}{bit}{0}
This signal is activated when RSC does not equal CSC and the software context
switch trap is enabled in CCR. It should trigger an RVEX_TRAP_SOFT_CTXT_SWITCH
in the S_MEM stage when high.

%===============================================================================
\ifaceGroup{Pipelane interface: external debug control signals}
%===============================================================================
\ifaceSubGroup{}
\ifaceOutCtxt{cxplif}{breakpoints}{breakpointInfo}{{others => 0}}
Current hardware breakpoint configuration. The cfg fields have the following
encoding:
 - 00 -> breakpoint disabled.
 - 01 -> instruction breakpoint.
 - 10 -> data write breakpoint.
 - 11 -> data access breakpoint.
All these signals map to a single cregl2pl_breakpoint_info_type, but the
control register generator does not support structures.

\ifaceSubGroup{}
\ifaceOutCtxt{cxplif}{extDebug}{bit}{0}
Whether debug traps are to be handled normally or by halting execution for 
debugging through the external bebug bus.

\ifaceSubGroup{}
\ifaceInCtxt{cxplif}{exDbgTrapInfo}{trapInfo}
External debug trap information. When the trap in exDbgTrapInfo is active, the 
trap cause should be stored in the debug control register and the BRK flag in 
the debug control register should be set.

\ifaceSubGroup{}
\ifaceOutCtxt{cxplif}{brk}{bit}{0}
BRK flag from the debug control register. When high, the core should be halted.

\ifaceSubGroup{}
\ifaceOutCtxt{cxplif}{stepping}{bit}{0}
Stepping mode flag from the debug control register. When high, executing any 
instruction which has the brkValid flag set should cause a step trap.

\ifaceSubGroup{}
\ifaceOutCtxt{cxplif}{resuming}{bit}{0}
\ifaceInCtxt{cxplif}{resuming_ack}{bit}
Resuming flag. This is set when the BRK flag is cleared by the debug bus. It is 
cleared when the resumed bit is high while stall is low. While high, issued 
instructions should have the brkValid flag cleared, so breakpoints and step 
traps are ignored.
    
%===============================================================================
\ifaceGroup{Interface with configuration logic}
%===============================================================================
\ifaceSubGroup{}
\ifaceIn{cfg}{currentConfig}{data}
Current configuration. Each nibble in the data word corresponds to a pipelane 
group, of which bit 3 specifies whether the pipelane group should be disabled 
(high) or enabled (low) and, if low, bit 2..0 specify the context it should run 
on.

\ifaceSubGroup{}
\ifaceOutCtxt{cfg}{requestData}{data}{0}
\ifaceOutCtxt{cfg}{requestEnable}{bit}{0}
Manual configuration request signals. The data signal has the same encoding as 
currentConfig. Bits which are not supported by the core (as specified in the CFG 
generic) should be written zero or the request will be ignored (as specified by 
the error flag in the global control register file). The enable signal is active 
high.

\ifaceSubGroup{}
\ifaceOut{cfg}{wakeupConfig}{data}{0}
\ifaceOut{cfg}{wakeupEnable}{bit}{0}
\ifaceIn{cfg}{wakeupAck}{bit}
Wakeup system configuration signals. wakeupConfig has the same encoding as
currentConfig and requestData. When wakeupEnable is high, the wakeup system will
request that configuration when the interrupt request line for context 0 is high
and context 0 is not already running. When it does, wakeupAck will be high.

%===============================================================================
\ifaceGroup{Trace control unit interface}
%===============================================================================
\ifaceSubGroup{}
\ifaceOutCtxt{trace}{enable}{bit}{0}
Whether tracing should be enabled or not for each context. Active high.
    
\ifaceSubGroup{}
\ifaceOutCtxt{trace}{trapEn}{bit}{0}
Whether trap information should be traced. Active high.
    
\ifaceSubGroup{}
\ifaceOutCtxt{trace}{memEn}{bit}{0}
Whether memory operations should be traced. Active high.
    
\ifaceSubGroup{}
\ifaceOutCtxt{trace}{regEn}{bit}{0}
Whether register writes should be traced. Active high.
    
\ifaceSubGroup{}
\ifaceOutCtxt{trace}{cacheEn}{bit}{0}
Whether cache performance information should be traced. Active high.
    
\ifaceSubGroup{}
\ifaceOutCtxt{trace}{instrEn}{bit}{0}
Whether instructions (the raw syllables) should be traced. Active high.
