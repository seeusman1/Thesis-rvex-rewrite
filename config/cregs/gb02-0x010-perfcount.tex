
%===============================================================================
\register{CNT}{Cycle counter register}{0x010}
%===============================================================================

Cycle counter. This register is simply always incremented by one in hardware.
Simply overflows when it reaches 0xFFFFFFFF. Its intended use is to monitor
real time. As an indication, this register overflows approximately every 85
seconds at 50 MHz.

% - - - - - - - - - - - - - - - - - - - - - - - - - - - - - - - - - - - - - - -
\field{31..0}{CNT}
\declaration{}
\declRegister{_r}{unsigned56}{0}
\implementation{}
\begin{lstlisting}
_r = _r + 1;
_read = _r[0, 32];
\end{lstlisting}


%===============================================================================
\register{CNTH}{Cycle counter register high}{0x014}
%===============================================================================

This register extends the \creg{CNT} register by 24 bits. The low byte is equal
to the high byte of \creg{CNT}, similar to the performance counters, which
allows the same algorithm to be used in order to read the value. Refer to
Section~\ref{sec:core-ug-creg-perf} for more information. Note however, that
unlike the other performance counters, this register always exists, regardless
of the design-time configured performance counter width.

% - - - - - - - - - - - - - - - - - - - - - - - - - - - - - - - - - - - - - - -
\field{31..8}{CNTH}
\implementation{}
\begin{lstlisting}
_read = cr_cnt_cnt_r[32, 24];
\end{lstlisting}
\field{7..0}{CNT}
\implementation{}
\begin{lstlisting}
_read = cr_cnt_cnt_r[24, 8];
\end{lstlisting}

