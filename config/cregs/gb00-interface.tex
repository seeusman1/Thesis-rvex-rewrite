
\globalInterface{}

%===============================================================================
\ifaceGroup{Interface with configuration logic}
%===============================================================================
\ifaceSubGroup{}
\ifaceOut{cfg}{requestData_r}{data}{0}
\ifaceOut{cfg}{requestEnable}{bit}{0}
Each nibble in the data word corresponds to a pipelane group, of which bit 3 
specifies whether the pipelane group should be disabled (high) or enabled (low) 
and, if low, bit 2..0 specify the context it should run on. Bits which are not 
supported by the core (as specified in the CFG generic) should be written zero 
or the request will be ignored (as specified by the error flag in the global 
control register file). The enable signal is active high, and is valid one 
clkEn'd cycle BEFORE the data vector is. This is because the enable signal is 
connected to the bus write enable signal for the register and the data is 
connected to the register output.

\ifaceSubGroup{}
\ifaceIn{cfg}{currentCfg}{data}
Current configuration, using the same encoding as the request data.
    
\ifaceSubGroup{}
\ifaceIn{cfg}{busy}{bit}
Configuration busy signal. When set, new configuration requests are not 
accepted.
    
\ifaceSubGroup{}
\ifaceIn{cfg}{error}{bit}
Configuration error signal. This is set when the last configuration request was 
erroneous.
    
\ifaceSubGroup{}
\ifaceIn{cfg}{requesterID}{bit(3..0)}
When reconfiguration is requested, this field is set to the index of the context 
which requested the configuration, or all ones if the source was the debug bus.

%===============================================================================
\ifaceGroup{Interface with memory}
%===============================================================================
\ifaceSubGroup{}
\ifaceIn{imem}{affinity}{data}
Affinity signal from the memory.

