%===============================================================================
\register{RSC}{Requested software context}{0x260}
%===============================================================================

This register is intended to hold an identifier for the software context that is 
requested to be run on this hardware context. When this value does not equal the 
value in \code{CSC} and context switching is enabled in \creg{CCR}, the 
\trap{SOFT_CTXT_SWITCH} trap is caused. Refer to its documentation in 
Section~\ref{sec:core-ug-traps} for more information.

The reset value for this register is -1, which would normally indicate that no
software context is loaded.

% - - - - - - - - - - - - - - - - - - - - - - - - - - - - - - - - - - - - - - -
\field{31..0}{RSC}
\reset{--------------------------------}
\declaration{}
\declRegister{_r}{data}{X"FFFFFFFF"}
\declRegister{_c0data}{data}{0}
\declRegister{_c0mask}{data}{0}
\implementation{}
\begin{lstlisting}
if (ctxt == 0) {
    
    // Handle bus access.
    _r = (_r & ~_wmask) | (_write & _wmask);
    _read = _r;
    
} else {
    
    // Handle bus access. The request from context 0 is taken from _c0data and
    // _c0mask, which are driven by the RSCn implementation. For some reason,
    // the placer did not appreciate doing this single-cycle.
    _r = (((_r & ~_wmask) | (_write & _wmask)) & ~_c0mask) | (_c0data & _c0mask);
    _read = _r;
    
}
\end{lstlisting}

%===============================================================================
\register{CSC}{Current software context}{0x264}
%===============================================================================

This register is intended to hold an identifier for the software context that is 
currently running on this hardware context. When this value does not equal the 
value in \code{RSC} and context switching is enabled in \creg{CCR}, the 
\trap{SOFT_CTXT_SWITCH} trap is caused. For proper operation, the trap handler
for this trap must set \code{CSC} to \code{RSC}. Refer to its documentation in 
Section~\ref{sec:core-ug-traps} for more information.

% - - - - - - - - - - - - - - - - - - - - - - - - - - - - - - - - - - - - - - -
\field{31..0}{CSC}
\reset{11111111111111111111111111111111}
\declaration{}
\declRegister{_r}{data}{X"FFFFFFFF"}
\declRegister{_neq}{bit}{0}
\implementation{}
\begin{lstlisting}

// Handle bus access.
_r = (_r & ~_wmask) | (_write & _wmask);
_read = _r;

// Compare CSC with RSC and register the signal.
_neq = (bit)(cr_csc_csc_r != cr_rsc_rsc_r);

\end{lstlisting} % Request a context switch when CSC != RSC and CCR.C is set:
\connect{cxreg2cxplif_softCtxtSwitch}{_neq & cr_ccr_c_r}

%===============================================================================
\registergen{range(1, 8)}{RSC\n{}}{Requested swctxt on hwctxt \n{}}{0x260}{8}
%===============================================================================

\emph{This register only exists on context 0, and only if the core is
design-time configured to support hardware context \n{}.} This register is 
hardwired to \creg{RSC} in hardware context \n{}. This allows the scheduler
(which is expected to be run on hardware context 0) to request a soft context
switch preemptively. Refer to \creg{RSC} for more information.

% - - - - - - - - - - - - - - - - - - - - - - - - - - - - - - - - - - - - - - -
\field{31..0}{RSC\n{}}
\reset{11111111111111111111111111111111}
\implementation{}
\begin{lstlisting}
if ((ctxt == 0) && (\n{} < (1 << CFG.numContextsLog2))) {
    
    // Handle bus access.
    cr_rsc_rsc_c0data@\n{} = _write;
    cr_rsc_rsc_c0mask@\n{} = _wmask;
    _read = cr_rsc_rsc_r@\n{};
    
}
\end{lstlisting}

%===============================================================================
\registergen{range(1, 8)}{CSC\n{}}{Current swctxt on hwctxt \n{}}{0x264}{8}
%===============================================================================

\emph{This register only exists on context 0, and only if the core is
design-time configured to support hardware context \n{}.} This register is 
hardwired to \creg{CSC} in hardware context \n{}. It allows the scheduler
(which is expected to be run on context 0) to see which software context is
running on the other hardware contexts. It is read-only. Refer to \creg{CSC} for
more information.

% - - - - - - - - - - - - - - - - - - - - - - - - - - - - - - - - - - - - - - -
\field{31..0}{CSC\n{}}
\reset{11111111111111111111111111111111}
\implementation{}
\begin{lstlisting}
if ((ctxt == 0) && (\n{} < (1 << CFG.numContextsLog2))) {
    
    // Handle bus access.
    _read = cr_csc_csc_r@\n{};
    
}
\end{lstlisting}

