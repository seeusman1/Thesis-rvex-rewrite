%===============================================================================
\register{RSC}{Requested software context}{0x260}
%===============================================================================

\emph{This register does not exist on context 0.} It is hardwired to
\code{RSC}$n$ in hardware context 0, and represents the software context that
should be loaded into our hardware context, if it is not already loaded. The
encoding of the register is at the user's discretion, but it is intended that
this points to a memory region that contains the to be loaded context.

The contents of this register are controlled by hardware context 0, which is 
expected to run the scheduler. When this value does not equal the value in 
\code{CSC} and context switching is enabled in \creg{CCR}, the 
\trap{SOFT_CTXT_SWITCH} trap is caused. Refer to its documentation in
Section~\ref{sec:core-ug-traps} for more information.

% - - - - - - - - - - - - - - - - - - - - - - - - - - - - - - - - - - - - - - -
\field{31..0}{RSC}
\reset{11111111111111111111111111111111}
\declaration{}
\declRegister{_r}{data}{X"FFFFFFFF"}
\implementation{}
\begin{lstlisting}
if (ctxt != 0) {
    
    // Handle bus access.
    _read = _r;
    
}
\end{lstlisting}

%===============================================================================
\register{CSC}{Current software context}{0x264}
%===============================================================================

\emph{This register does not exist on context 0.} It is hardwired to
\code{CSC}$n$ in hardware context 0. The value in this register should be set
to the value in \creg{RSC} by the \trap{SOFT_CTXT_SWITCH} trap.

% - - - - - - - - - - - - - - - - - - - - - - - - - - - - - - - - - - - - - - -
\field{31..0}{CSC}
\reset{11111111111111111111111111111111}
\declaration{}
\declRegister{_r}{data}{X"FFFFFFFF"}
\declRegister{_neq}{bit}{0}
\implementation{}
\begin{lstlisting}
if (ctxt != 0) {
    
    // Handle bus access.
    _r = (_r & ~_wmask) | (_write & _wmask);
    _read = _r;
    
    // Compare CSC with RSC and register the signal.
    _neq = (bit)(cr_csc_csc_r != cr_rsc_rsc_r);
    
}
\end{lstlisting} % Request a context switch when CSC != RSC and CCR.C is set:
\connect{cxreg2cxplif_softCtxtSwitch}{_neq & cr_ccr_c_r}

%===============================================================================
\registergen{range(1, 8)}{RSC\n{}}{Requested swctxt on hwctxt \n{}}{0x260}{8}
%===============================================================================

\emph{This register only exists on context 0, and only if the core is
design-time configured to support hardware context \n{}.} This register is 
hardwired to \creg{RSC} in hardware context \n{}. Refer to \creg{RSC} for more
information.

% - - - - - - - - - - - - - - - - - - - - - - - - - - - - - - - - - - - - - - -
\field{31..0}{RSC\n{}}
\reset{11111111111111111111111111111111}
\implementation{}
\begin{lstlisting}
if ((ctxt == 0) && (\n{} < (1 << CFG.numContextsLog2))) {
    
    // Handle bus access.
    cr_rsc_rsc_r@\n{} = (cr_rsc_rsc_r@\n{} & ~_wmask) | (_write & _wmask);
    _read = cr_rsc_rsc_r@\n{};
    
}
\end{lstlisting}

%===============================================================================
\registergen{range(1, 8)}{CSC\n{}}{Current swctxt on hwctxt \n{}}{0x264}{8}
%===============================================================================

\emph{This register only exists on context 0, and only if the core is
design-time configured to support hardware context \n{}.} This register is 
hardwired to \creg{CSC} in hardware context \n{}. Refer to \creg{CSC} for more
information.

% - - - - - - - - - - - - - - - - - - - - - - - - - - - - - - - - - - - - - - -
\field{31..0}{CSC\n{}}
\reset{11111111111111111111111111111111}
\implementation{}
\begin{lstlisting}
if ((ctxt == 0) && (\n{} < (1 << CFG.numContextsLog2))) {
    
    // Handle bus access.
    _read = cr_csc_csc_r@\n{};
    
}
\end{lstlisting}

