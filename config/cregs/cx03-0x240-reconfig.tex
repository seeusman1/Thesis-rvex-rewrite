
%===============================================================================
\register{CRR}{Context reconfiguration request register}{0x240}
%===============================================================================

This register may be written to by the core only. When it is written, a
reconfiguration is requested. Refer to Section~\ref{sec:core-ug-reconf} for more
information.

\field{31..0}{Context reconfiguration request}
\coreCanWrite{}

%===============================================================================
%\register{...}{...}{0x244}
%===============================================================================
% Reserved for a control register for reconfiguration flags if it's ever needed.

%===============================================================================
\register{WCFG}{Wakeup configuration}{0x248}
%===============================================================================

\todo{The WCFG register doesn't exist yet.}
\emph{This register only exists on context 0.} This configuration register is 
used in conjunction with the S flag in \creg{SAWC}. Refer to
Section~\ref{sec:core-ug-reconf-saw} for more information.

\field{31..0}{Wakeup configuration}
\coreCanWrite{}
\debugCanWrite{}

%===============================================================================
\register{SAWC}{Sleep and wake-up control register}{0x24C}
%===============================================================================

\todo{The SAWC register doesn't exist yet.}
\emph{This register only exists on context 0.} This register contains special 
control features for sleeping (reconfiguring to a configuration with all lane 
groups disabled) and waking up other hardware contexts.

\field{7..1}{Running}
\debugCanWrite{}
\coreCanWrite{}
This field contains a bit for every other context, i.e., not all of these bits
will be available if the core is not configured to support all eight hardware
contexts. When reading this register, each bit represents the ones complement
of the B flag in \creg{DCR} for each other context. Writing a one to a bit is
equivalent to writing a one to the R flag in \creg{DCR} for each other context.

A scheduler running on context 0 may use this feature, combined with an
interrupt controller that triggers an interrupt when the done output for any
other context has a rising edge, to support task yielding for cooperative
scheduling. A yield will then be equivalent to a \insn{STOP} instruction, which
will thus trigger an interrupt for the scheduler. The scheduler may then
switch out the software context and subsequently restart the hardware context
using these flags.

\field{0}{S}
\debugCanWrite{}
\coreCanWrite{}
Sleep flag. This enables or disables the sleep and wake-up system. Refer to
Section~\ref{sec:core-ug-reconf-saw} for more information.

