
%===============================================================================
\register{CCR}{Main context control register}{0x200}
%===============================================================================

The primary purpose of the context control register is to store the primary
control flags of the processor, for example whether interrupts are enabled. In
addition, it also stores the trap cause and exposes the branch register file to
the debug bus.

%-------------------------------------------------------------------------------
\field{31..24}{CAUSE}
%-------------------------------------------------------------------------------
\id{TC}
Trap cause. Set to the trap cause by hardware when the trap handler is called. 
Reset to 0 by hardware when an \insn{RFI} instruction is encountered. Read-write 
by the debug bus, but the processor cannot write to this register.

% - - - - - - - - - - - - - - - - - - - - - - - - - - - - - - - - - - - - - - -
\declaration{}
\declVariable{_v}{trapCause}{0}
\declRegister{_r}{trapCause}{0}
\implementation{}
\begin{lstlisting}
_v = _r;

// Handle bus interface.
_v = (_v & ~_wmask_dbg) | (_write & _wmask_dbg);
_read = _r;

// Handle entering/leaving trap.
if (!cxplif2cxreg_stall) {
    if (cxplif2cxreg_trapInfo.active) {
        _v = cxplif2cxreg_trapInfo.cause;
    }
    if (cxplif2cxreg_rfi) {
        _v = 0;
    }
}

// Output the cxreg2cxplif_handlingDebugTrap signal.
<?vhdl
@lvalue cxreg2cxplif_handlingDebugTrap <= TRAP_TABLE(vect2uint(@read _v)).isDebugTrap;
?><?c
#error "Need to implement this!"
?>

_r = _v;
\end{lstlisting}

%-------------------------------------------------------------------------------
\field{23..16}{BRANCH}
%-------------------------------------------------------------------------------
\id{BR}
Branch register file. Contains the current state of the branch registers. Only
intended for use by the debug bus to see and modify the state of the branch
register file. While the core is running, accessing this register is undefined
due to it being dependent on the pipeline and forwarding state.

% - - - - - - - - - - - - - - - - - - - - - - - - - - - - - - - - - - - - - - -
\declaration{}
\declVariable{_v}{brRegData}{0}
\declRegister{_r}{brRegData}{0}
\implementation{}
\begin{lstlisting}
_v = _r;

// Handle bus interface.
_v = (_v & ~_wmask_dbg) | (_write & _wmask_dbg);
_read = _r;

// Handle core interface.
if (!cxplif2cxreg_stall) {
    _v = (_v                       & ~cxplif2cxreg_brWriteEnable)
       | (cxplif2cxreg_brWriteData &  cxplif2cxreg_brWriteEnable);
}

_r = _v;
\end{lstlisting}
\connect{cxreg2cxplif_brReadData}{_r}

%-------------------------------------------------------------------------------
\field{9..8}{K}
%-------------------------------------------------------------------------------
\todo{The kernel mode/MMU enable flag exists, but the MMU is not in the design
yet.}
This register selects between kernel mode and user mode. Currently, this flag 
only controls whether the MMU is enabled; the \rvex{} processor does not have 
any security features yet. In kernel mode, the MMU is bypassed; in user mode, it 
is activated. Kernel mode is activated when the core is reset and when entering 
the trap or panic handlers. These must thus always point to code in hardware 
memory space. When \insn{RFI} is executed, the state is restored from
\creg{SCCR}.

In kernel mode, the register reads as \code{01}, while in user mode, it reads as 
\code{10}. The only way to enter user mode is by writing the user mode command 
to \creg{SCCR} and subsequently executing \insn{RFI}. Neither the core nor the 
debug bus can write to this field directly.

% - - - - - - - - - - - - - - - - - - - - - - - - - - - - - - - - - - - - - - -
\reset{01}
\declaration{}
\declVariable{_v}{bit}{0}
\declRegister{_r}{bit}{1}
\implementation{}
\begin{lstlisting}
_v = _r;

// Handle bus interface.
_read = ~_r $ _r;

// Handle restoring and trap entry.
if (!cxplif2cxreg_stall) {
    if (cxplif2cxreg_trapInfo.active) {
        _v = 1;
    }
    if (cxplif2cxreg_rfi) {
        _v = cr_sccr_k_r;
    }
}

_r = _v;
\end{lstlisting}

%-------------------------------------------------------------------------------
\field{7..6}{C}
%-------------------------------------------------------------------------------
This register controls whether the context switch trap is enabled. It does not 
exist on hardware context 0. When the core is reset or the trap service routine 
is entered, the context switch trap is disabled. When \insn{RFI} is executed, 
the state is restored from \creg{SCCR}.

When the context switch trap is enabled, this register reads as \code{01}. When 
it is disabled, it reads as \code{10}. Both the core and the debug bus can write 
to this register. Writing \code{00} has no effect, writing \code{01} enables the 
context switching trap, writing \code{10} disables it and writing \code{11} 
toggles the state. This prevents the need for read-modify-write operations.

Refer to \creg{RSC} for more information.

% - - - - - - - - - - - - - - - - - - - - - - - - - - - - - - - - - - - - - - -
\reset{10}
\declaration{}
\declVariable{_v}{bit}{0}
\declRegister{_r}{bit}{0}
\declVariable{_cmd}{bitvec2}{0}
\implementation{}
\begin{lstlisting}
_v = _r;

// Handle bus interface.
_cmd = _wmask & _write;
if (_cmd == "11") _v = ~_v; else _v = (_v | _cmd[0]) & ~_cmd[1];
_read = ~_r $ _r;

// Handle restoring and trap entry.
if (!cxplif2cxreg_stall) {
    if (cxplif2cxreg_trapInfo.active) {
        _v = 0;
    }
    if (cxplif2cxreg_rfi) {
        _v = cr_sccr_c_r;
    }
}

_r = _v;
\end{lstlisting}

%-------------------------------------------------------------------------------
\field{5..4}{B}
%-------------------------------------------------------------------------------
This register controls whether breakpoints are enabled in self-hosted debug 
mode. Its value is ignored in external debug mode. When the core is reset or the 
trap service routine is entered due to a debug trap in self-hosted debug mode, 
breakpoints are disabled. When \insn{RFI} is executed, the state is restored 
from \creg{SCCR}.

When breakpoints are enabled, this register reads as \code{01}. When they are 
disabled, it reads as \code{10}. Both the core and the debug bus can write to 
this register. Writing \code{00} has no effect, writing \code{01} enables debug 
traps, writing \code{10} disables them and writing \code{11} toggles the state. 
This prevents the need for read-modify-write operations.

% - - - - - - - - - - - - - - - - - - - - - - - - - - - - - - - - - - - - - - -
\reset{10}
\declaration{}
\declVariable{_v}{bit}{0}
\declRegister{_r}{bit}{0}
\declVariable{_cmd}{bitvec2}{0}
\implementation{}
\begin{lstlisting}
_v = _r;

// Handle bus interface.
_cmd = _wmask & _write;
if (_cmd == "11") _v = ~_v; else _v = (_v | _cmd[0]) & ~_cmd[1];
_read = ~_r $ _r;

// Handle restoring and trap entry.
if (!cxplif2cxreg_stall) {
    if (cxplif2cxreg_trapInfo.active & cxplif2cxreg_trapIsDebug) {
        _v = 0;
    }
    if (cxplif2cxreg_rfi) {
        _v = cr_sccr_b_r;
    }
}

_r = _v;
\end{lstlisting}
\connect{cxreg2cxplif_debugTrapEnable}{_r}

%-------------------------------------------------------------------------------
\field{3..2}{R}
%-------------------------------------------------------------------------------
This register, named ready-for-trap, tentatively specifies if the processor is
currently capable of servicing traps. However, since traps cannot be masked,
any trap that occurs while ready-for-trap is cleared will cause a panic.
Therefore, the only thing this register does in hardware is switch between the
trap handler and panic handler address. When the core is reset or the trap
service routine is entered, ready-for-trap is cleared. When \insn{RFI} is 
executed, the state is restored from \creg{SCCR}.

When ready-for-trap is set (trap handler selected), this register reads as 
\code{01}. When it is cleared (panic handler selected), it reads as \code{10}. 
Both the core and the debug bus can write to this register. Writing \code{00} 
has no effect, writing \code{01} sets ready-for-trap, writing \code{10} clears 
it and writing \code{11} toggles the state. This prevents the need for 
read-modify-write operations.

% - - - - - - - - - - - - - - - - - - - - - - - - - - - - - - - - - - - - - - -
\reset{10}
\declaration{}
\declVariable{_v}{bit}{0}
\declRegister{_r}{bit}{0}
\declVariable{_cmd}{bitvec2}{0}
\implementation{}
\begin{lstlisting}
_v = _r;

// Handle bus interface.
_cmd = _wmask & _write;
if (_cmd == "11") _v = ~_v; else _v = (_v | _cmd[0]) & ~_cmd[1];
_read = ~_r $ _r;

// Handle restoring and trap entry.
if (!cxplif2cxreg_stall) {
    if (cxplif2cxreg_trapInfo.active) {
        _v = 0;
    }
    if (cxplif2cxreg_rfi) {
        _v = cr_sccr_r_r;
    }
}

_r = _v;
\end{lstlisting}

%-------------------------------------------------------------------------------
\field{1..0}{I}
%-------------------------------------------------------------------------------
This register selects whether external interrupts are enabled or not. When the 
core is reset or the trap service routine is entered, external interrupts are 
disabled. When \insn{RFI} is executed, the state is restored from \creg{SCCR}.

When interrupts are enabled, this register reads as \code{01}. When they are 
disabled, it reads as \code{10}. Both the core and the debug bus can write to 
this register. Writing \code{00} has no effect, writing \code{01} enables 
external interrupts, writing \code{10} disables them and writing \code{11} 
toggles the state. This prevents the need for read-modify-write operations.

% - - - - - - - - - - - - - - - - - - - - - - - - - - - - - - - - - - - - - - -
\reset{10}
\declaration{}
\declVariable{_v}{bit}{0}
\declRegister{_r}{bit}{0}
\declVariable{_cmd}{bitvec2}{0}
\implementation{}
\begin{lstlisting}
_v = _r;

// Handle bus interface.
_cmd = _wmask & _write;
if (_cmd == "11") _v = ~_v; else _v = (_v | _cmd[0]) & ~_cmd[1];
_read = ~_r $ _r;

// Handle restoring and trap entry.
if (!cxplif2cxreg_stall) {
    if (cxplif2cxreg_trapInfo.active) {
        _v = 0;
    }
    if (cxplif2cxreg_rfi) {
        _v = cr_sccr_i_r;
    }
}

_r = _v;
\end{lstlisting}
\connect{cxreg2cxplif_interruptEnable}{_r}

%===============================================================================
\register{SCCR}{Saved context control register}{0x204}
%===============================================================================

This register saves the state of the primary control flags of the processor when
entering the trap service routine. When \insn{RFI} is executed, the state is
restored from this register. In addition, this register contains the context ID,
which contexts may read to identify themselves.

%-------------------------------------------------------------------------------
\field{31..24}{ID}
%-------------------------------------------------------------------------------
\id{CID}
This field is hardwired to the context index. Programs running on the \rvex{}
processor may use this field to determine which hardware context they are
running on.

Note that \creg{CID} is not unique in a multi-processor system. If a unique
processor ID is needed in such a case, \creg{COID} should be used as well.

% - - - - - - - - - - - - - - - - - - - - - - - - - - - - - - - - - - - - - - -
\reset{********}
\implementation{}
\begin{lstlisting}
_read = ctxt;
\end{lstlisting}

%-------------------------------------------------------------------------------
\field{9..8}{K}
%-------------------------------------------------------------------------------
When the trap service routine is entered, this register stores whether kernel
the processor was in kernel mode or user mode. When \insn{RFI} is executed,
the state is set to this value.

Unlike the kernal mode field in \creg{CCR}, this field can be written. Writing 
\code{00} has no effect, writing \code{01} enables external interrupts, writing 
\code{10} disables them and writing \code{11} toggles the state. This prevents 
the need for read-modify-write operations. Read behavior is identical to the
K field in \creg{CCR}.

% - - - - - - - - - - - - - - - - - - - - - - - - - - - - - - - - - - - - - - -
\reset{10}
\declaration{}
\declVariable{_v}{bit}{0}
\declRegister{_r}{bit}{0}
\declVariable{_cmd}{bitvec2}{0}
\implementation{}
\begin{lstlisting}
_v = _r;

// Handle bus interface. The core can only go to application mode.
_cmd = (_wmask_dbg | (_wmask_core & "10")) & _write;
if (_cmd == "11") _v = ~_v; else _v = (_v | _cmd[0]) & ~_cmd[1];
_read = ~_r $ _r;

// Handle saving.
if (!cxplif2cxreg_stall) {
    if (cxplif2cxreg_trapInfo.active) {
        _v = cr_ccr_k_r;
    }
}

_r = _v;
\end{lstlisting}

%-------------------------------------------------------------------------------
\field{7..6}{C}
%-------------------------------------------------------------------------------
When the trap service routine is entered, this register stores whether the 
context switching trap was enabled. When \insn{RFI} is executed, the state is 
set to this value.

Core and debug bus access behavior is identical to the C field in \creg{CCR}.

% - - - - - - - - - - - - - - - - - - - - - - - - - - - - - - - - - - - - - - -
\reset{10}
\declaration{}
\declVariable{_v}{bit}{0}
\declRegister{_r}{bit}{0}
\declVariable{_cmd}{bitvec2}{0}
\implementation{}
\begin{lstlisting}
_v = _r;

// Handle bus interface.
_cmd = _wmask & _write;
if (_cmd == "11") _v = ~_v; else _v = (_v | _cmd[0]) & ~_cmd[1];
_read = ~_r $ _r;

// Handle saving.
if (!cxplif2cxreg_stall) {
    if (cxplif2cxreg_trapInfo.active) {
        _v = cr_ccr_c_r;
    }
}

_r = _v;
\end{lstlisting}

%-------------------------------------------------------------------------------
\field{5..4}{B}
%-------------------------------------------------------------------------------
When the trap service routine is entered, this register stores whether 
self-hosted debug breakpoints were enabled. When \insn{RFI} is executed, the 
state is set to this value.

Core and debug bus access behavior is identical to the B field in \creg{CCR}.

% - - - - - - - - - - - - - - - - - - - - - - - - - - - - - - - - - - - - - - -
\reset{10}
\declaration{}
\declVariable{_v}{bit}{0}
\declRegister{_r}{bit}{0}
\declVariable{_cmd}{bitvec2}{0}
\implementation{}
\begin{lstlisting}
_v = _r;

// Handle bus interface.
_cmd = _wmask & _write;
if (_cmd == "11") _v = ~_v; else _v = (_v | _cmd[0]) & ~_cmd[1];
_read = ~_r $ _r;

// Handle saving.
if (!cxplif2cxreg_stall) {
    if (cxplif2cxreg_trapInfo.active) {
        _v = cr_ccr_b_r;
    }
}

_r = _v;
\end{lstlisting}

%-------------------------------------------------------------------------------
\field{3..2}{R}
%-------------------------------------------------------------------------------
When the trap service routine is entered, this register stores whether 
ready-for-trap was set. When \insn{RFI} is executed, the state is set to this
value.

Core and debug bus access behavior is identical to the R field in \creg{CCR}.

% - - - - - - - - - - - - - - - - - - - - - - - - - - - - - - - - - - - - - - -
\reset{10}
\declaration{}
\declVariable{_v}{bit}{0}
\declRegister{_r}{bit}{0}
\declVariable{_cmd}{bitvec2}{0}
\implementation{}
\begin{lstlisting}
_v = _r;

// Handle bus interface.
_cmd = _wmask & _write;
if (_cmd == "11") _v = ~_v; else _v = (_v | _cmd[0]) & ~_cmd[1];
_read = ~_r $ _r;

// Handle saving.
if (!cxplif2cxreg_stall) {
    if (cxplif2cxreg_trapInfo.active) {
        _v = cr_ccr_r_r;
    }
}

_r = _v;
\end{lstlisting}

%-------------------------------------------------------------------------------
\field{1..0}{I}
%-------------------------------------------------------------------------------
When the trap service routine is entered, this register stores whether 
interrupts were enabled. When \insn{RFI} is executed, the state is set to this
value.

Core and debug bus access behavior is identical to the I field in \creg{CCR}.

% - - - - - - - - - - - - - - - - - - - - - - - - - - - - - - - - - - - - - - -
\reset{10}
\declaration{}
\declVariable{_v}{bit}{0}
\declRegister{_r}{bit}{0}
\declVariable{_cmd}{bitvec2}{0}
\implementation{}
\begin{lstlisting}
_v = _r;

// Handle bus interface.
_cmd = _wmask & _write;
if (_cmd == "11") _v = ~_v; else _v = (_v | _cmd[0]) & ~_cmd[1];
_read = ~_r $ _r;

// Handle saving.
if (!cxplif2cxreg_stall) {
    if (cxplif2cxreg_trapInfo.active) {
        _v = cr_ccr_i_r;
    }
}

_r = _v;
\end{lstlisting}

%===============================================================================
\register{LR}{Link register}{0x208}
%===============================================================================

Contains the current link register (\texttt{\$l0.0}) value. Only intended for
use by the debug bus. While the core is running, accessing this register is
undefined due to it being dependent on the pipeline and forwarding state.

% - - - - - - - - - - - - - - - - - - - - - - - - - - - - - - - - - - - - - - -
\field{31..0}{LR}
\declaration{}
\declRegister{_r}{address}{0}
\implementation{}
\begin{lstlisting}

// Handle bus interface.
_r = (_r & ~_wmask_dbg) | (_write & _wmask_dbg);
_read = _r;

// Handle core interface.
if (!cxplif2cxreg_stall && cxplif2cxreg_linkWriteEnable) {
    _r = cxplif2cxreg_linkWriteData;
}

\end{lstlisting}
\connect{cxreg2cxplif_linkReadData}{_r}

%===============================================================================
\register{PC}{Program counter}{0x20C}
%===============================================================================

Contains the current program counter. Only intended for use by the debug bus.
When the register is written by the debug bus, the jump flag in \creg{DCR} is
set, to ensure that the branch unit properly jumps to the new PC. This works
even if the processor is running.

% - - - - - - - - - - - - - - - - - - - - - - - - - - - - - - - - - - - - - - -
\field{31..0}{PC}
\declaration{}
\declVariable{_v}{address}{0}
\declRegister{_r}{address}{0}
\implementation{}
\begin{lstlisting}
_v = _r;

// The bus must be able to override the PC, so we need to execute the write from
// the pipelane first. Notice that writes are ignored when the debugger is
// trying to jump.
if (!cxplif2cxreg_stall && !cr_dcr_j_r) {
    _v = cxplif2cxreg_nextPC;
}

// Handle bus interface. Set the jump flag when the register is written to by
// the debug bus.
_v = (_v & ~_wmask_dbg) | (_write & _wmask_dbg);
if (_wmask_dbg[0]) {
    cr_dcr_j_r = '1';
}

_r = _v;
_read = _r;

\end{lstlisting}
\connect{cxreg2cxplif_currentPC}{_r}

%===============================================================================
\register{TH}{Trap handler}{0x210}
%===============================================================================

Contains the address of the trap service routine. This is where the processor
will jump to if a trap occurs while ready-for-trap in \creg{CCR} is set. Even if
the design contains an MMU, this should be a hardware address, as the MMU is
disabled when a trap occurs.

% - - - - - - - - - - - - - - - - - - - - - - - - - - - - - - - - - - - - - - -
\field{31..0}{TH}
\declaration{}
\declRegister{_r}{address}{0}
\implementation{}
\begin{lstlisting}
_r = (_r & ~_wmask) | (_write & _wmask);
_read = _r;
\end{lstlisting}

%===============================================================================
\register{PH}{Panic handler}{0x214}
%===============================================================================

Contains the address of the panic service routine. This is where the processor
will jump to if a trap occurs while ready-for-trap in \creg{CCR} is NOT set.
Even if the design contains an MMU, this should be a hardware address, as the
MMU is disabled when a trap occurs.

The difference between the trap and panic service routines, is that the trap
service routine has all state information of the processor at its disposal. That
is, if the trap is recoverable, the program can continue after the trap service
routine completes. The panic service routine, however, should assume that the
state information of the processor is incomplete. Refer to
Section~\ref{sec:core-ug-traps} for more information.

% - - - - - - - - - - - - - - - - - - - - - - - - - - - - - - - - - - - - - - -
\field{31..0}{PH}
\declaration{}
\declRegister{_r}{address}{0}
\implementation{}
\begin{lstlisting}
_r = (_r & ~_wmask) | (_write & _wmask);
_read = _r;
\end{lstlisting}

%===============================================================================
% Trap handler output
%===============================================================================

% Output the currently active trap handler when all variables have been set.
\connect{cxreg2cxplif_trapHandler}{(CR_TH_TH_r & CR_CCR_R_r) | (CR_PH_PH_r & ~CR_CCR_R_r)}

%===============================================================================
\register{TP}{Trap point}{0x218}
%===============================================================================

When a trap occurs, this register is set to the address of the start of the
offending bundle. The address is in user space if the MMU was enabled when the
trap occured. In addition, when \insn{RFI} is executed, the processor will jump
back to this address to resume execution. This is the correct behavior for both
external interrupts and traps that, after servicing, should return to the
previously offending instruction, such as a page fault.

To support software context switching, the processor may write to this register
to change the resumption address. \insn{RFI} will then cause execution to be
resumed in the new software context, assuming the rest of the processor state
has been swapped in as well.

% - - - - - - - - - - - - - - - - - - - - - - - - - - - - - - - - - - - - - - -
\field{31..0}{TP}
\declaration{}
\declRegister{_r}{address}{0}
\implementation{}
\begin{lstlisting}

// Handle bus interface.
_r = (_r & ~_wmask) | (_write & _wmask);
_read = _r;

// Handle entering/leaving trap.
if (!cxplif2cxreg_stall && cxplif2cxreg_trapInfo.active) {
    _r = cxplif2cxreg_trapPoint;
}

\end{lstlisting}
\connect{cxreg2cxplif_trapReturn}{_r}

%===============================================================================
\register{TA}{Trap argument}{0x21C}
%===============================================================================

When a trap occurs, this register is set to the trap argument. The significance
of this value depends on the trap, which can be identified from the trap cause
field in \creg{CCR}. Refer to Section~\ref{sec:core-ug-traps} for more
information.

% - - - - - - - - - - - - - - - - - - - - - - - - - - - - - - - - - - - - - - -
\field{31..0}{TA}
\declaration{}
\declRegister{_r}{address}{0}
\implementation{}
\begin{lstlisting}

// Handle bus interface.
_r = (_r & ~_wmask_dbg) | (_write & _wmask_dbg);
_read = _r;

// Handle entering/leaving trap.
if (!cxplif2cxreg_stall && cxplif2cxreg_trapInfo.active) {
    _r = cxplif2cxreg_trapInfo.arg;
}

\end{lstlisting}

