%===============================================================================
\registergen{range(4)}{BR\n{}}{Breakpoint \n{}}{0x220}{4}
%===============================================================================

These registers hold the addresses for the hardware breakpoints and/or
watchpoints. These registers only exist up to how many break-/watchpoints are
design-time configured to be supported by the processor. Breakpoints are
configured in \creg{DCR}.

% - - - - - - - - - - - - - - - - - - - - - - - - - - - - - - - - - - - - - - -
\field{31..0}{BR\n{}}
\implementation{}

%===============================================================================
\register{DCR}{Debug control register 1}{0x230}
%===============================================================================

This register controls the debugging system of the \rvex{} processor.

%-------------------------------------------------------------------------------
\field{31}{D}
%-------------------------------------------------------------------------------
Done/reset flag. This bit is set by hardware when a \insn{STOP} instruction is 
encountered. It is cleared when a one is written to the R or S flags.

In addition, when a one is written to this flag, the control register file for 
this context is completely reset, as if the external context reset signal was 
asserted. Writing a zero has no effect. When combined with writing a one to the 
external debug flag, the core starts in external debug mode, and when combined 
with writing a one to B or the S flag, the core will stop execution before any 
instruction is executed, allowing the user to single-step from the start of the 
program. This works because I, E, S and B are not affected by a context reset.

Note that breakpoint information will have to be reloaded when the context is 
reset using this method.

% - - - - - - - - - - - - - - - - - - - - - - - - - - - - - - - - - - - - - - -
\implementation{}

%-------------------------------------------------------------------------------
\field{30}{J}
%-------------------------------------------------------------------------------
This bit is set by hardware when the debug bus writes to the PC register and is 
cleared when the processor jumps to it. It can thus be used as an acknowledgement
flag for jumping. The flag is read only.

% - - - - - - - - - - - - - - - - - - - - - - - - - - - - - - - - - - - - - - -
\implementation{}

%-------------------------------------------------------------------------------
\field{28}{I}
%-------------------------------------------------------------------------------
\reset{1}
Internal debug flag. Complement of the external debug flag. When the debug bus 
writes a one to this flag, the external debug flag is cleared, giving the
processor control over debugging. Writing a zero has no effect. This flag is not
affected by a context reset; it is only reset when the entire core is reset.

% - - - - - - - - - - - - - - - - - - - - - - - - - - - - - - - - - - - - - - -
\implementation{}

%-------------------------------------------------------------------------------
\field{27}{E}
%-------------------------------------------------------------------------------
External debug flag. Complement of the internal debug flag. When the debug bus 
writes a one to this flag, the external debug flag is set, enabling external 
debug mode. Writing a zero has no effect. This flag is not affected by a context 
reset; it is only reset when the entire core is reset.

While in external debug mode, debug traps cause the B flag to be set and the 
trap cause to be recorded in \creg{DCR} instead of the normal registers. This 
thus allows an external debugger to handle the debug traps instead, even if the 
processor is in the middle of a trap service routine and is not even ready for a 
trap. Writing a one to the R or the S flag is the equivalent of \insn{RFI} for 
the external debugging system. 

% - - - - - - - - - - - - - - - - - - - - - - - - - - - - - - - - - - - - - - -
\implementation{}

%-------------------------------------------------------------------------------
\field{26}{R}
%-------------------------------------------------------------------------------
Resume flag. When the debug bus writes a one to this flag, the B flag is
cleared, causing the processor to resume execution if it was halted. Writing a
zero has no effect; this flag is cleared by hardware when the first instruction
is successfully fetched. It can thus be used as an acknowledgement flag for
resuming execution.

In addition, debug traps are disabled for instructions that were fetched while
this flag was set. This behavior allows the processor to step beyond the
breakpoint that caused the processor to break, so there is no need to disable
the triggered breakpoint in order to resume. This behavior is also used for
single stepping; see below.

% - - - - - - - - - - - - - - - - - - - - - - - - - - - - - - - - - - - - - - -
\implementation{}

%-------------------------------------------------------------------------------
\field{25}{S}
%-------------------------------------------------------------------------------
Step flag. This flag may be set by the debug bus by writing a one to it. Doing
so will also cause the R flag to be set and the B flag to be cleared, causing
the processor to resume execution if it was halted. Writing a zero has no
effect. The processor can also set this flag, but only if the E flag is cleared,
i.e., if the processor is in self-hosted debug mode. This flag is not affected 
by a context reset; it is only reset when the entire core is reset.

While set, any instruction will cause a step debug trap. However, as noted
above, all debug traps are disabled for the first instruction fetched after
execution resumes. They should also be disabled while in the trap service
routine through the breakpoint enable field in \creg{CCR}. This allows both an
external debugger and the self-hosted debug system to single-step.

% - - - - - - - - - - - - - - - - - - - - - - - - - - - - - - - - - - - - - - -
\implementation{}

%-------------------------------------------------------------------------------
\field{24}{B}
%-------------------------------------------------------------------------------
Break flag. When this flag is set, the context stops fetching instructions and
flushes the pipeline, as it would if the external run signal is low or if a
reconfiguration is pending. It effectively halts execution. This flag is not 
affected by a context reset; it is only reset when the entire core is reset.

This flag may be set by the debug bus by writing a one to it, in order to pause
execution. Writing a zero has no effect. In addition, the flag is set by
hardware when a debug trap occurs while the E flag is set and when a \insn{STOP}
instruction is executed.

% - - - - - - - - - - - - - - - - - - - - - - - - - - - - - - - - - - - - - - -
\implementation{}

%-------------------------------------------------------------------------------
\field{23..16}{CAUSE}
%-------------------------------------------------------------------------------
Trap cause for debug traps that should be handled by the external debug system.
This is set to the debug trap cause by hardware when the B flag is set due to a
debug trap.

% - - - - - - - - - - - - - - - - - - - - - - - - - - - - - - - - - - - - - - -
\implementation{}

%-------------------------------------------------------------------------------
\field{13..12}{BR3}
%-------------------------------------------------------------------------------
Breakpoint 3 control field. This field only exists if the core is design-time
configured to support all four hardware breakpoints. See also BR0.

% - - - - - - - - - - - - - - - - - - - - - - - - - - - - - - - - - - - - - - -
\implementation{}

%-------------------------------------------------------------------------------
\field{9..8}{BR2}
%-------------------------------------------------------------------------------
Breakpoint 2 control field. This field only exists if the core is design-time
configured to support at least three hardware breakpoints. See also BR0.

% - - - - - - - - - - - - - - - - - - - - - - - - - - - - - - - - - - - - - - -
\implementation{}

%-------------------------------------------------------------------------------
\field{5..4}{BR1}
%-------------------------------------------------------------------------------
Breakpoint 1 control field. This field only exists if the core is design-time
configured to support at least two hardware breakpoints. See also BR0.

% - - - - - - - - - - - - - - - - - - - - - - - - - - - - - - - - - - - - - - -
\implementation{}

%-------------------------------------------------------------------------------
\field{1..0}{BR0}
%-------------------------------------------------------------------------------
Breakpoint 0 control field. This field only exists if the core is design-time
configured to support at least one hardware breakpoint.

The core can only write to BR$n$ fields when the E flag is cleared, i.e. when
self-hosted debug mode is selected. The encoding for the fields is as follows.

\vskip 10pt\noindent\begin{tabularx}{\textwidth}{@{}l@{}X@{}}
BR$n$ = \code{00}: & \ breakpoint/watchpoint disabled. \\
BR$n$ = \code{01}: & \ breakpoint enabled. \\
BR$n$ = \code{10}: & \ data write watchpoint enabled. \\
BR$n$ = \code{11}: & \ data read/write watchpoint enabled. \\
\end{tabularx}

% - - - - - - - - - - - - - - - - - - - - - - - - - - - - - - - - - - - - - - -
\implementation{}

%===============================================================================
\register{DCR2}{Debug control register 2}{0x234}
%===============================================================================

This register controls the trace unit, if the core is design-time configured to
support tracing. It also contains an 8-bit scratchpad register for communicating
an execution result to the debug system.

%-------------------------------------------------------------------------------
\field{31..24}{RESULT}
%-------------------------------------------------------------------------------
\signed{}
\id{RET}
This field does not have a hardwired function. It is intended to be used to
communicate the reason for executing a \insn{STOP} instruction to the debug
system. The default \code{_start.s} file will write the \code{main()} return
value to this register before stopping.

% - - - - - - - - - - - - - - - - - - - - - - - - - - - - - - - - - - - - - - -
\implementation{}

%-------------------------------------------------------------------------------
\field{15}{U}
%-------------------------------------------------------------------------------
\reset{*}
Whether the core was design-time configured to be able to trace trap
information.

% - - - - - - - - - - - - - - - - - - - - - - - - - - - - - - - - - - - - - - -
\implementation{}

%-------------------------------------------------------------------------------
\field{14}{N}
%-------------------------------------------------------------------------------
\reset{*}
Whether the core was design-time configured to be able to trace memory/control
register operations.

% - - - - - - - - - - - - - - - - - - - - - - - - - - - - - - - - - - - - - - -
\implementation{}

%-------------------------------------------------------------------------------
\field{13}{S}
%-------------------------------------------------------------------------------
\reset{*}
Whether the core was design-time configured to be able to trace register writes.

% - - - - - - - - - - - - - - - - - - - - - - - - - - - - - - - - - - - - - - -
\implementation{}

%-------------------------------------------------------------------------------
\field{12}{D}
%-------------------------------------------------------------------------------
\reset{*}
Whether the core was design-time configured to be able to trace cache
performance information.

% - - - - - - - - - - - - - - - - - - - - - - - - - - - - - - - - - - - - - - -
\implementation{}

%-------------------------------------------------------------------------------
\field{11}{J}
%-------------------------------------------------------------------------------
\reset{*}
Whether the core was design-time configured to be able to trace all fetched
instructions.

% - - - - - - - - - - - - - - - - - - - - - - - - - - - - - - - - - - - - - - -
\implementation{}

%-------------------------------------------------------------------------------
\field{8}{F}
%-------------------------------------------------------------------------------
\reset{*}
Whether the core was design-time configured to support tracing.

% - - - - - - - - - - - - - - - - - - - - - - - - - - - - - - - - - - - - - - -
\implementation{}

%-------------------------------------------------------------------------------
\field{7}{T}
%-------------------------------------------------------------------------------
Setting this bit enables trap tracing if the E flag is set and the core is 
design-time configured to support it.

% - - - - - - - - - - - - - - - - - - - - - - - - - - - - - - - - - - - - - - -
\implementation{}

%-------------------------------------------------------------------------------
\field{6}{M}
%-------------------------------------------------------------------------------
Setting this bit enables memory/control register tracing if the E flag is set 
and the core is design-time configured to support it.

% - - - - - - - - - - - - - - - - - - - - - - - - - - - - - - - - - - - - - - -
\implementation{}

%-------------------------------------------------------------------------------
\field{5}{R}
%-------------------------------------------------------------------------------
Setting this bit enables register write tracing if the E flag is set and the 
core is design-time configured to support it.

% - - - - - - - - - - - - - - - - - - - - - - - - - - - - - - - - - - - - - - -
\implementation{}

%-------------------------------------------------------------------------------
\field{4}{C}
%-------------------------------------------------------------------------------
Setting this bit enables cache performance tracing if the E flag is set and the 
core is design-time configured to support it.

% - - - - - - - - - - - - - - - - - - - - - - - - - - - - - - - - - - - - - - -
\implementation{}

%-------------------------------------------------------------------------------
\field{3}{I}
%-------------------------------------------------------------------------------
Setting this bit causes all fetched instructions to be traced if the E flag is 
set and the core is design-time configured to support it.

% - - - - - - - - - - - - - - - - - - - - - - - - - - - - - - - - - - - - - - -
\implementation{}

%-------------------------------------------------------------------------------
\field{0}{E}
%-------------------------------------------------------------------------------
Setting this bit enables tracing if the core is design-time configured to 
support it. If no other bits are set, only branch origins and destinations are 
traced.

% - - - - - - - - - - - - - - - - - - - - - - - - - - - - - - - - - - - - - - -
\implementation{}

