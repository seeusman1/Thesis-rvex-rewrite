\chapter{\rvex{} core user guide}
\label{sec:core-ug}

This chapter is intended as a user guide for using the \rvex{} processor core.
For documentation about how the core works internally, refer to
Chapter~\ref{sec:core-int} instead.

The \hyperref[sec:core-ug-intro]{first section} of this chapter gives a general 
introduction on VLIW processor architecture and the parts that make the \rvex{} 
processor special. The \hyperref[sec:core-ug-isa]{next section} describes the 
instruction set architecture (ISA) in detail, including a list of all 
instructions and their encodings. The \hyperref[sec:core-ug-creg]{third section} 
lists and documents all the control registers. The 
\hyperref[sec:core-ug-traps]{fourth section} describes the trap and interrupt 
model of the core, and subsequently lists all currently defined traps. The 
\hyperref[sec:core-ug-reconf]{fifth section} briefly documents the 
reconfiguration system. The \hyperref[sec:core-ug-cfg]{sixth section} documents 
the design-time configuration system of the core. Finally, the 
\hyperref[sec:core-ug-inst]{last section} documents how the core is instantiated 
in an HDL design.

\clearpage
\section{Introduction to the \rvex{} processor}
\subimport{1-intro/}{core-ug-intro.tex}

\clearpage
\section{Instruction set architecture}
\subimport{2-isa/}{core-ug-isa.tex}

\clearpage
\section{Control registers}
\subimport{3-creg/}{core-ug-creg.tex}

\clearpage
\section{Traps and interrupts}
\subimport{4-traps/}{core-ug-traps.tex}

\clearpage
\section{Reconfiguration and sleeping}
\subimport{5-reconf/}{core-ug-reconf.tex}

\clearpage
\section{Design-time configuration}
\subimport{6-cfg/}{core-ug-cfg.tex}

\clearpage
\section{Instantiation}
\subimport{7-inst/}{core-ug-inst.tex}
