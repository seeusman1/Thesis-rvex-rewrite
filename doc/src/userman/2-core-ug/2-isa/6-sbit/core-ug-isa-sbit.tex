
\subsection{Stop bits}
\label{sec:core-ug-isa-sbit}

The stop bit system is the colloquial name for the binary compression algorithm
that the core may be design-time configured to support. It refers to a bit
present in every syllable, which, if set, marks the syllable as the last syllable
in the current bundle. In contrast, when the stop bit system is not used, bundle
boundaries are based on alignment; each bundle is expected to start on an
alignment boundary of the maximum size of a bundle. \insn{NOP} instructions are
then used to fill the unused words. The stop bit should then still be set in the
last syllable, as failing to do so will cause a trap if the bundle contains a
branch syllable.

The major advantage of stop bits is the decreased size of the binary. This does
not only mean that the memory footprint of a program will be smaller; memory is
cheap, so this is usually not an issue. More importantly, it means that the
processor will need to do less instruction memory accesses for the same amount
of computation; memory bandwidth and caches \emph{are} expensive.

There is an additional benefit when combined with generic binaries. When a
generic binary without stop bits runs in 8-way mode, the \insn{NOP} instructions
needed for bundle alignment do not cause any delays in execution, aside from the
implicit delays due to the strain on the instruction memory system. However,
when the binary is run in 2-way mode, these alignment \insn{NOP}s may actually
cost cycles. To illustrate, imagine an 8-way generic binary bundle with only
two syllables used. When this bundle is executed in 2-way mode, execution will
necessarily still take four cycles, because the processor still needs to work
through eight syllables.\footnote{It is certainly possible to avoid this without
a complete stop bit system. For example, for the previous version of the \rvex{}
processor, it was proposed to use the stop bits to mark the end of the useful
part of a bundle, instead of the actual boundaries. In the case of our 8-way
bundle with only two syllables used, assuming the two syllables can be placed
in the first two slots, the stop bit would be set in the second syllable instead
of the eighth. When this code is executed in 2-way mode, the \rvex{} processor
would recognize that it can jump to the next 8-way bundle alignment boundary,
thus skipping the six \insn{NOP} syllables.} When stop bits are enabled, such
alignment \insn{NOP}s do not exist, so they will naturally never waste cycles.

The major disadvantage of using stop bits is its hardware complexity. Without 
stop bits, the core naturally always fetches a nicely aligned block of 
instruction memory to process. Each 32-bit word in this block can be wired 
directly to the syllable input of each lane. In contrast, when stop bits are 
fully enabled, a bundle may start on any 32-bit word boundary. Thus, a new 
module is needed between the instruction memory (which expects accesses aligned 
to its access size) and the pipelanes. This module must then be capable of 
routing any incoming 32-bit word to any pipelane, based on the lower bits of the 
current program counter and even the syllable type, as branch syllables always 
need to be routed to the last pipelane. It must also store the previous fetch to 
handle misaligned bundles, and when a branch to a misaligned address occurs, it 
must stall execution for an additional cycle, as it will have to fetch both the 
memory block before and after the crossed alignment boundary.

On the plus side, the large multiplexers involved in this instruction buffer do
not increase in size when adding reconfiguration capabilities to an 8-way core
with stop bits. Some additional control logic is obviously required, but nothing
more.

\subsubsection{Design-time configuration}
\label{ref:core-ug-isa-sbit-cfg}

The \rvex{} processor core allows the designer to make a compromise between the
large binary size without stop bits and the additional hardware needed with stop
bits. Instead of simply supporting stop bits or not, the stop bit system is
configured by specifying the bundle alignment boundaries that the core may
expect. When the bundle alignment boundaries equal the size of the maximum
bundle size, stop bits are effectively disabled. When the alignment boundary is
set to 32-bit words, stop bits are fully enabled. Midway configuration are
supported equally well.

Every time the bundle alignment boundary is halved, the multiplexers in the
syllable dispatch logic double in size. The complexity of program counter
generation increases with each step as well, as does the instruction fetch
buffer size. Meanwhile, the number of alignment \insn{NOP}s required in the
binary decreases with each step.

The default 8-way reconfigurable core with stop bits enabled have the bundle
alignment boundary set to 64-bit. Going all the way to 32-bit boundaries does
not increase 2-way execution performance of an 8-way generic binary further, and
most \insn{NOP}s have already been eliminated, so doubling the hardware
complexity once more is generally not justifiable.
