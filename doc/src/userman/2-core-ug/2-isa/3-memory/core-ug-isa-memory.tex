
\subsection{Memory}
\label{sec:core-ug-isa-memory}

Each lane group of the \rvex{} processor currently has exactly one memory unit.
The configurability of this may be extended in the future, as memory operations
commonly end up being the critical path when extracting instruction-level
parallelism. However, doing so would require significant modifications to the
\rvex{} core and the reconfigurable cache.

The \rvex{} processor is big endian. This means that when accessing a 32-bit or 
16-bit word, the most significant byte will reside in the lowest address. This
is the opposite of what you may be used to coming from x86.

The \rvex{} processor is capable of reading and writing 32-bit, 16-bit and 8-bit
words. Seperate read instructions exist for reading 16-bit and 8-bit words in
signed or unsigned mode. All $n$-bit accesses must be $n$-bit aligned. If an
access is improperly aligned, a \code{MISALIGNED_ACCESS} trap will be caused.

Note that a 1 kiB block of the external memory space must be selected to be 
remapped to the control register file internally. This prevents the processor 
from being able to access the block. Refer to 
Section~\ref{sec:core-ug-isa-regs-creg} for more information.
