
\subsubsection{Trace interface}
\label{sec:core-ug-cfg-inst-iface-trace}

The trace interface provides an optional write-only bus to some memory system or 
peripheral, which the core may send trace information to. The trace system is 
disabled by default and must be enabled in the \creg{DCR2} control register. In 
addition, the trace unit hardware is only instantiated when \code{traceEnable} 
is set in the \code{CFG} vector.

\begin{itemize}

\item \code{rv2trsink_push : out std_logic}

When high, \code{rv2trsink_data} and \code{rv2trsink_end} are valid and should 
be registered in the next cycle where \code{clkEn} is high.

\vspace{1em}
\item \code{rv2trsink_data : out rvex_byte_type}

Trace data signal. Valid when \code{rv2trsink_push} is high.

\vspace{1em}
\item \code{rv2trsink_end : out std_logic}

When high, this is the last byte of this trace packet. May be used to flush 
buffers downstream, or may be ignored.

\vspace{1em}
\item \code{trsink2rv_busy : in  std_logic}

When high while \code{rv2trsink_push} is high, the trace unit is stalled. While 
stalled, \code{rv2trsink_push} will stay high and \code{rv2trsink_data} and 
\code{rv2trsink_end} will remain stable.

\end{itemize}

