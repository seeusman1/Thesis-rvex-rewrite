
\subsubsection{Core configuration}
\label{sec:core-ug-cfg-inst-iface-generics}

These generics parameterize the core.

\begin{itemize}

\item \code{CFG : rvex_generic_config_type}

This generic contains the configuration parameters for the core.
\code{rvex_generic_config_type} is a \code{record} type with the following
members.

\begin{itemize}

\item \code{numLanesLog2 : natural}

This parameter specifies the binary logarithm of the number of lanes to
instantiate. The range of acceptable values is 0 through 4, although only 1, 2
and 3 are tested. The default is 3, which specifies an 8-way \rvex{} processor.

\item \code{numLaneGroupsLog2 : natural}

This parameter specifies the binary logarithm of the number of lane groups to
instantiate. Each lane group can be disabled individually to save power, operate
on its own, or work together on a single thread with other lane groups. May not
be greater than 3 (due to configuration register size limits) or
\code{numLanesLog2}. It is only tested up to \code{numLanesLog2}-1. The default
is 2, specifying 4 lane groups.

\item \code{numContextsLog2 : natural}

This parameter specifies the binary logarithm of the number of hardware contexts
in the core. May not be greater than 3 due to configuration register size limits.
The default is 2, specifying 4 hardware contexts.

\item \code{genBundleSizeLog2 : natural}

This parameter specifies the binary logarithm of the number of syllables in a
generic binary bundle. When a branch address is not aligned to this and
\code{limmhFromPreviousPair} is set, then special actions will be taken to
ensure that the relevant syllables preceding the trap point are fetched before
operation resumes. The default is 3, specifying 8-way generic binary bundles.

\item \code{bundleAlignLog2 : natural}

The \rvex{} processor will assume (and enforce) that the start addresses of
bundles are aligned to the specified amount of syllables. When this is less than
\code{numLanesLog2}, the stop bit system is enabled. The value may not be
greater than \code{numLanesLog2}. The default is 3, disabling the stop bit
system.

\item \code{multiplierLanes : natural}

This parameter defines what lanes have a multiplier. Bit 0 of this number maps
to the first lane, bit 1 to the second lane, etc. The default is \code{0xFF},
specifying that each lane has a multiplier.

\item \code{memLaneRevIndex : natural}

This parameter specifies the lane index for the memory unit, counting down from
the last lane in each lane group. So \code{memLaneRevIndex} = 0 results in the
memory unit being in the last lane in each group, \code{memLaneRevIndex} = 1
results in it being in the second to last lane, etc. The default is 1.

\item \code{numBreakpoints : natural}

This parameter specifies how many hardware breakpoints are instantiated. The
maximum is 4 due to the register map only having space for 4. The default is
also 4.

\item \code{forwarding : boolean}

This parameter specifies whether or not register forwarding logic should be
instantiated. With forwarding disabled, the core will use less area and might
run at higher frequencies, but much more NOPs are necessary between
data-dependent instructions. The forwarding logic is enabled by default.

\item \code{limmhFromNeighbor : boolean}

When this parameter is true, syllables can borrow long immediates from the
neighboring syllable in a syllable pair. This is enabled by default.

\item \code{limmhFromPreviousPair : boolean}

When this parameter is true, syllables can borrow long immediates from the
previous syllable pair. This is enabled by default. This is not supported when
stop bits are enabled, i.e. when \code{bundleAlignLog2} $<$ \code{numLanesLog2}.
Therefore, when stop bits are enabled, this should be disabled.

\item \code{reg63isLink : boolean}

When this parameter is true, general purpose register 63 maps directly to the
link register. When false, \insn{MOVTL}, \insn{MOVFL}, \insn{STW} and \insn{LDW}
must be used to access the link register, but an additional general purpose
register is available. This exists for compatibility with the ST200 series
processors. It is disabled by default.

\item \code{cregStartAddress : rvex_address_type}

This paramater specifies the start address of the 1kiB control register file as
seen from the processor. It must be aligned to a 1kiB boundary. The core is not
able to access data memory in the specified region. The default value is
\code{0xFFFFFC00}, i.e. the block from \code{0xFFFFFC00} to \code{0xFFFFFFFF}.

\item \code{resetVectors : rvex_address_array(7 downto 0)}

This parameter specifies the reset address for each context, if not overruled at
runtime by connecting the optional \code{rctrl2rv_resetVect} signal. When less
than eight contexts are instantiated, the higher indexed values are unused. The
default is 0 for all contexts.

\item \code{unifiedStall : boolean}

When this parameter is true, the stall signals for each group will be connected
to the same signal. That is, if one lane group has to stall, all lane groups
necessarily have to stall. This may be a requirement of the memory subsystem
connected to the core; when this is enabled, the memory architecture can be made
simpler, but cannot make use of the possible performance gain due to being able
to stall only part of the core. This parameter is disabled by default, meaning
that the stall signals are independent.

\item \code{gpRegImpl : natural}

This parameter specifies the general purpose register implementation to use.
The following values are accepted.

\begin{itemize}
\item \code{RVEX_GPREG_IMPL_MEM} (default): block RAM + LVT implementation for
FPGAs.
\item \code{RVEX_GPREG_IMPL_SIMPLE}: behavioral implementation for Synopsis.
\end{itemize}

\item \code{traceEnable : boolean}

This parameter specifies whether the trace unit should be instantiated. It is
disabled by default.

\item \code{perfCountSize : natural}

This parameter specifies the size of the performance counters in bytes. Up to 7 
bytes are supported. The default is 4 bytes.

\item \code{cachePerfCountEnable : boolean}

This parameter enables or disables the cache performance counters. When enabled, 
the number of lane groups must equal the number of contexts, because the signals 
from the cache blocks are mapped to the contexts directly. In the future, the
cache performance counters are to be placed in the cache instead of the core.
This parameter is off by default.

\end{itemize}

\noindent Typically, one will want to use the \code{rvex_cfg} function to
specify this value. This function takes as its arguments values for all the
record members as specified above, but has default values for each of them,
meaning that not all of them have to be specified. In addition, a \code{base}
argument of type \code{rvex_generic_config_type} may be specified, which will be
used as the default value for unspecified parameters. This permits mutation of
the \code{CFG} record as it passes from entity to subentity, which is otherwise
impossible to do with record generics.

\item \code{CORE_ID : natural}

This value is used to uniquely identify this core within a multicore platform. 
It is made available to the programs running on the core and the debug system 
through \creg{COID}.

\item \code{PLATFORM_TAG : std_logic_vector(55 downto 0)}

This value is to uniquely identify the platform as a whole. It is intended that
this value be generated by the toolchain by hashing the source files and
synthesis options. \todo{Reference version system.} It is made available to the
programs running on the core and the debug system through \creg{PTAG}.

\end{itemize}

