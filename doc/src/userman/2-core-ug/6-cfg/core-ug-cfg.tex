
\label{sec:core-ug-cfg}

The \rvex{} core is design-time configured by means of two different systems:
the VHDL generics passed to the toplevel core entity and the configuration
scripts.

\paragraph*{VHDL generics} VHDL generics are used to configure the most 
important metrics of the core, such as the issue width, the degree of 
reconfigurability, the functional unit distribution, and complexity of the debug 
support system. Refer to Section~\ref{sec:core-ug-cfg-inst-iface-generics} for 
more information about what exactly the generics control.

Because the generics are specified per instantiation of the core, it is possible
to have differently configured \rvex{} cores in a single design. This allows for
heterogeneous multicore systems.

The values of the generics are represented as read-only registers in the global
control register file in a generic way. The registers are designed such that
future additions to the core are unlikely to require restructuring the existing
registers, making them forward compatible. In addition, they are structured such
that it is easy to extract information from the data, usually even by visually
inspecting the hexadecimal values. The global control registers are described in
detail in Section~\ref{sec:core-ug-creg-gb}.

\todo[inline]{Uncomment stuff about version system here and write about it elsewhere in
the manual.}
% It should be noted that the value of the generics do not affect the core version
% tag, as they do not modify the source files of the core. However, since the
% generics must be specified in the platform VHDL sources, the platform tag will
% be affected. Refer to Section~\ref{sec:version-system} for more information
% about the versioning system.

\paragraph*{Configuration scripts} The `configuration scripts' refer to a set of 
Python scripts residing in the \code{config} directory in the root of the 
\rvex{} repository. When run by calling \code{make} in the root of the 
\code{config} directory, these scripts read a set of configuration and template 
files, to generate various sources in the repository. These sources vary from 
key VHDL sources for the \rvex{} core, to memory map headers for \code{rvd} and 
the build system, to the LaTeX source files for this very document. The 
philosophy is that this not only makes it easier to change key components of the 
core, but that it should also stimulate developers to keep the documentation 
up-to-date, without the primary source for documentation needing to be comments 
in the VHDL sources.

The configuration scripts control the following processor features.

\begin{itemize}

\item Global and context control register file functionality, memory map and
      documentation (Section~\ref{sec:core-ug-cfg-cregs}).

\item Instruction set encoding and documentation, as well as assembly syntax
      (Section~\ref{sec:core-ug-cfg-opc}).

\item Pipeline configuration of the \rvex{} core
      (Section~\ref{sec:core-ug-cfg-pip}).

\item Trap decoding and documentation
      (Section~\ref{sec:core-ug-cfg-traps}).

\end{itemize}

\noindent Each feature is controlled by a set of LaTeX-like files and/or 
key-value configuration files. These LaTeX-like files are not intended to be 
processed by anything other than the scripts --- they cannot be processed by 
LaTeX directly. The only reason for their syntax to be derived from LaTeX is 
because it allows the documentation sections to be properly syntax-highlighted.

Each class of configuration files supports a set of non-standard commands that 
define the configuration. These commands are described in the following
sections, as referenced in the list.

As stated, the configuration needs to be manually committed to the repository
by calling \code{make} in the \code{config} directory. Changing the
configuration files without doing this has no effect. This command also
regenerates this PDF file, but it does not rebuild or test anything else. It is
highly advised to run the conformance test suite in \code{platform/core-tests}
after changing the configuration.


\subimport{1-cregs/}{core-ug-cfg-cregs.tex}
\subimport{2-opc/}{core-ug-cfg-opc.tex}
\subimport{3-pip/}{core-ug-cfg-pip.tex}
\subimport{4-traps/}{core-ug-cfg-traps.tex}
