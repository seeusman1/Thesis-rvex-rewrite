
\subsubsection{LAC type system}
\label{sec:core-ug-cfg-cregs-types}

The primitive types supported by LAC and their VHDL equivalents are shown in
Table~\ref{tbl:core-ug-cfg-cregs-types-primitive}. The C equivalents of the types
range from \code{uint8_t} to \code{uint64_t}. The smallest available C type
that the LAC/VHDL type fits in is used. Note that this means that
\code{bitvec} and \code{unsigned} types of more than 64 bits are not supported.

\begin{table}[h]
\centering
\caption{LAC primitive types.}
\label{tbl:core-ug-cfg-cregs-types-primitive}
\newcommand{\lactypentry}[3]{\texttt{\detokenize{#1}} & #2 & \texttt{\detokenize{#3}}\\ \hline}
\footnotesize\begin{tabular}{|l|l|l|} \hline
\textbf{LAC type name} & \textbf{Supported values} & \textbf{VHDL type name} \\ \hline
\lactypentry{boolean}{\texttt{true} or \texttt{false}}{boolean}
\lactypentry{natural}{0..2147483647 (31 bit)}{natural}
\lactypentry{bit}{\texttt{'0'} or \texttt{'1'}}{std_logic}
\lactypentry{bitvec<n>}{$n$ bits of \texttt{'0'} or \texttt{'1'}}{std_logic_vector(n-1 downto 0)}
\lactypentry{unsigned<n>}{$n$ bits of \texttt{'0'} or \texttt{'1'}}{unsigned(n-1 downto 0)}
\end{tabular}\normalsize
\end{table}

In addition to the scalar primitives in the table, LAC also supports hardcoded 
aggregate types, i.e. the equivalent of a VHDL record or C struct. Rudimentary 
support is provided for array-typed aggregate members to be compatible with 
existing VHDL data structures at the time it was developed, though these arrays 
can only be indexed by integer literals.

In addition, objects can be instantiated per hardware context, which also 
results in an array. If per-context objects are used in a context control
register implementation, they are implicitely indexed by the context which the
register belongs to. Otherwise, a context may only be explicitely specified as
an integer literal.

Arrays present a problem in VHDL code output. Of the primitive types, only
\code{bit} actually has a VHDL array type. To get around this, and also to have
the generated code be consistent with the human-written VHDL sources, a number
of derived types are available. These are listed along with the supported
aggregate types in Table~\ref{tbl:core-ug-cfg-cregs-types-derived}.

\begin{table}[h]
\centering
\caption{LAC derived types.}
\label{tbl:core-ug-cfg-cregs-types-derived}
\newcommand{\lactypentry}[5]{
    \texttt{\detokenize{#1}} & \texttt{\detokenize{#3}} & \texttt{\detokenize{#5}} \\
    \texttt{\detokenize{(#2)}} & \texttt{\detokenize{#4}} & \\ \hline
}
\footnotesize\begin{tabular}{|l|l|l|} \hline
\textbf{LAC} & \textbf{VHDL} & \textbf{C} \\ \hline
\lactypentry{byte}          {bitvec8}  {rvex_byte_type}               {rvex_byte_array}               {uint8_t}
\lactypentry{data}          {bitvec32} {rvex_data_type}               {rvex_data_array}               {uint32_t}
\lactypentry{address}       {bitvec32} {rvex_address_type}            {rvex_address_array}            {uint32_t}
\lactypentry{sylstatus}     {bitvec16} {rvex_sylStatus_type}          {rvex_sylStatus_array}          {uint16_t}
\lactypentry{brregdata}     {bitvec8}  {rvex_brRegData_type}          {rvex_brRegData_array}          {uint8_t}
\lactypentry{trapcause}     {bitvec8}  {rvex_trap_type}               {rvex_trap_array}               {uint8_t}
\lactypentry{twobit}        {bitvec2}  {rvex_2bit_type}               {rvex_2bit_array}               {uint8_t}
\lactypentry{threebit}      {bitvec3}  {rvex_3bit_type}               {rvex_3bit_array}               {uint8_t}
\lactypentry{fourbit}       {bitvec4}  {rvex_4bit_type}               {rvex_4bit_array}               {uint8_t}
\lactypentry{sevenByte}     {bitvec56} {rvex_7byte_type}              {rvex_7byte_array}              {uint64_t}
\lactypentry{trapinfo}      {aggregate}{trap_info_type}               {trap_info_array}               {trapInfo_t}
\lactypentry{breakpointinfo}{aggregate}{cxreg2pl_breakpoint_info_type}{cxreg2pl_breakpoint_info_array}{breakpointInfo_t}
\lactypentry{cachestatus}   {aggregate}{rvex_cacheStatus_type}        {rvex_cacheStatus_array}        {cacheStatus_t}
\lactypentry{cfgvect}       {aggregate}{rvex_generic_config_type}     {-}                             {cfgVect_t}
\end{tabular}\normalsize
\end{table}

\FloatBarrier
\paragraph*{Coercion and typecasts} Typically, LAC will take care of typing for
you by coercing one type into another on the fly. If LAC does not know how to do
it or a cast would be ambiguous, you can cast manually using C-style typecasts.
The following rules apply.

\begin{itemize}

\item Conversions between \code{boolean} and \code{natural} works as they do in
C. That is, \code{false} converts to zero and vice versa, \code{true} converts
to one, and nonzero converts to \code{true}.

\item Conversions between \code{boolean} and \code{bit} use positive logic. That
is, \code{'1'} equals \code{true} and \code{'0'} equals \code{false}.

\item \code{bit} and \code{bitvec1} are interchangeable.

\item \code{bitvec<n>} and \code{unsigned<n>} are interchangeable.

\item When a \code{bitvec<n>} is cast to a \code{bitvec} of different size, the
vector is zero-extended or truncated.

\item When a \code{bitvec<n>} is cast to a \code{natural} or vice versa, the
value is zero-extended or truncated, with the \code{natural} behaving as a
31-bit value.

\end{itemize}

\paragraph*{Access types} Aside from having a type that describes what kind of 
values are allowed for an object, LAC objects also have an `access type'. This 
describes the access priviliges, scoping rules and general behavior of an 
object. The following access types are available.

\begin{itemize}

\item \emph{Input}: represents an input port of the VHDL entity. They are 
available everywhere but in constant object initializers. They are read only.

\item \emph{Register output}: represents an output port of the VHDL entity,
driven from the clocked process. They are write only.

\item \emph{Combinatorial output}: represents an output port of the VHDL entity,
driven combinatorially using a \code{\connect} command. These objects may not
be used in LAC sections.

\item \emph{Register}: represents a user defined register, declared using a
\code{\declRegister} command. These may be read and written in any LAC section,
regardless of where they are declared. They behave like VHDL signals; that is,
when they are written to, the read value of a register is not affected until the
next clock cycle.

\item \emph{Variable}: represents a user defined variable, declared using a
\code{\declVariable} command. These may be read and written in the
\code{\implementation} section of only the field to which they belong.
Furthermore, they may be read in any \code{\finally} section.

\item \emph{Constant}: represents a user defined constant, declared using a
\code{\declConstant} command. They are read only and globally scoped.

\item \emph{Predefined constant}: represents a predefined constant, such as a
package constant or the \code{CFG} generic. They are read only and globally
scoped. Section~\ref{sec:core-ug-cfg-cregs-predef} lists the available
predefined constants.

\end{itemize}

