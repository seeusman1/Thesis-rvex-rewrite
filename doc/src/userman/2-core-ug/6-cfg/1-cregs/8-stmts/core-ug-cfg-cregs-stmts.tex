
\subsubsection{LAC statements}
\label{sec:core-ug-cfg-cregs-stmts}

LAC only supports the following statements.

\vskip 6 pt
\codehead{<reference> = <expression>;}

\noindent Assignment statement.

\vskip 6 pt
\codehead{if (<expression>) <statement>}

\noindent Conditional statement without \code{else}.

\vskip 6 pt
\codehead{if (<expression>) <statement> else <statement>}

\noindent Conditional statement with \code{else}.

\vskip 6 pt
\codehead{{ <statement*> }}

\noindent C-style block statement.

\vskip 6 pt
\codehead{<?vhdl ... ?>}

\codehead{<?c ... ?>}

\noindent Verbatim block statements. Anything written in place of the ellipsis 
is in principle outputted straight to the VHDL or C output. This allows the 
usage of constructs unknown to LAC. Even in these sections however, it is 
possible to have the code generator convert LAC-style references to the target 
language. This is particularly useful for C output, where the LAC objects are 
part of special data structures. The syntaxes for such converted references are 
as follows.

\begin{lstlisting}[numbers=none, language=nothing]
@read <name>
@read <name>@<context>
@lvalue <name>
@lvalue <name>@<context>
\end{lstlisting}

\noindent In addition to being convenient syntactic sugar, the LAC generator 
keeps track of which objects are read from and written to. Not using this syntax 
may result in incorrect optimizations.

