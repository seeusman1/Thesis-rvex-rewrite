
\subsection{Traps}
\label{sec:core-ug-cfg-traps}

The trap configuration files reside in the \code{config/traps} directory of the 
\rvex{} repository. The configuration consists of a set of LaTeX-styled files, 
interpreted ordered alphabetically by their filenames. The configuration 
controls roughly the following things.

\begin{itemize}

\item The trap names and numeric identifiers.

\item Decoding signals for each trap; debug and interrupt traps are handled
      differently by the processor.
      
\item A pretty-printing macro for each trap.
      
\item Documentation for each trap, as it appears in
      Section~\ref{sec:core-ug-traps-ids}.

\end{itemize}

\noindent The next section describes the structure of the LaTeX-style
configuration files. The subsequent section provides a command reference.


\subsubsection[.tex file structure]{\texttt{.tex} file structure}
\label{sec:core-ug-cfg-traps-struct}

The \code{\trap} and \code{\trapgen} commands start the definition of a trap or
a number of similar traps respectively. Any unrecognized command or text
following such a command is interpreted as being LaTeX documentation for the
trap. The obligatory \code{\description} command defines the formatting string
used to pretty-print the trap information. The remaining commands are optional
decoding attributes for the traps.


\subsubsection[.tex command reference]{\texttt{.tex} command reference}
\label{sec:core-ug-cfg-traps-cref}

The following LaTeX-like commands are interpreted by the Python scripts to 
define the traps. They must be the only thing on a certain line aside from 
optional LaTeX-style comments at the end of the line, otherwise they are 
interpreted as part of a documentation section.

\vskip 6 pt
\codehead{\trap{<index>}{<mnemonic>}{<name>}}

\noindent The command starts a trap description. \code{<index>} is the trap 
index, which may range from 1 to 255. \code{<mnemonic>} is the trap identifier, 
which must be a valid C and VHDL identifier and should be uppercase. It is 
prefixed with \code{RVEX_TRAP_} in the header files. \code{<name>} is the 
LaTeX-formatted friendly name of the trap, used as the section title in the 
documentation.

\vskip 6 pt
\codehead{\trapgen{<python range>}{<start index>}{<mnemonic>}{<name>}}

\noindent This command works the same as \code{\trap}, but specifies a list of 
traps. \code{<python range>} is executed as a Python expression, expected to 
generate an iterable of integers. A trap specification is generated for each of 
these iterations. The index for each trap is computed as \code{<offset>}$ + 
iter$. \code{\n{}} expands to the iterator value when used inline in 
\code{<mnemonic>} and \code{<name>}, as well as in \code{\description{<desc>}} 
below. In the documentation it expands to \code{$n$}.

\vskip 6 pt
\codehead{\description{<desc>}}

\noindent This command defines a formatting string used to pretty-print the trap 
information. It is used by the debug systems to allow the user to quickly 
identify the trap. In this description, the following commands may be used 
inline.

\begin{itemize}

\item \code{\at{}} expands to ``\code{ at <trap point>}'' if the trap point is 
known, or to nothing if the trap point is not known. The trap point is expressed 
in hexadecimal notation.

\item \code{\arg{u}} expands to the trap argument in unsigned decimal notation.

\item \code{\arg{s}} expands to the trap argument in signed decimal notation.

\item \code{\arg{x}} expands to the trap argument in hexadecimal notation.

\end{itemize}

\vskip 6 pt
\codehead{\debug{}}

\noindent Marks that this trap is a debug trap. The \code{{}} is required.

\vskip 6 pt
\codehead{\interrupt{}}

\noindent Marks that this trap is an interrupt trap. The \code{{}} is required.

