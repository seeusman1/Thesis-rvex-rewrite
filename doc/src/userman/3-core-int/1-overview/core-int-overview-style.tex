
\label{sec:core-int-overview-style}

All code within the \rvex{} packages is wrote using a consistent code style.
Special attention was paid to naming conventions, as VHDL easily becomes
confusing due to the large amount of signals and variables everywhere.

\begin{itemize}

\item All signals, entity names, package names and types use a combination of 
camelCase and underscores. Typically, underscores are used as a form of 
hierarchy separation, where the VHDL language does not otherwise allow it, and 
camelCase is used to indicate word boundaries within one level of hierarchy. For 
example, \code{core_ctrlRegs_bank} refers to a (register) bank for the control 
registers for the \rvex{} core.

\item Most signal names start with an underscore-terminated abbreviation, which
indicates the source and destination entity. This identifier contains two entity 
abbreviation codes separated by a \code{2}. The entity abbreviation codes are 
defined at the top of \code{core.vhd} and are also listed in the previous
section. Sometimes other abbreviations are used for signals local to one entity,
which should be clear from context.

\item All constant names are uppercase with underscores.

\item Labels typically use underscores only, to prevent conflicts between 
similarly named entities and signals.

\item Types use one of the following suffixes to indicate what kind of type they 
represent.
\begin{itemize}
    \item \code{_type}: scalar type.
    \item \code{_array}: array type.
    \item \code{_ptr}: access type for a scalar.
    \item \code{_array_ptr}: access type for an array.
\end{itemize}

\item Enties and packages have the same name as their filename; exactly one 
entity or package is defined per VHDL file. Package names end in \code{_pkg}. 
All \rvex{} core files start with \code{core_} to keep their names unique within 
the \code{rvex} package, which contains a number of supporting packages as well.

\item Entity descriptions must clearly document the function of every port and
generic that passes it by, \emph{especially} when the entity generates or uses
the signal (as opposed to just routing it). All hope of future generations
comprehending the code is lost when interfaces are not clear.

\item Ports should be grouped by function or route. The groups should be made
apparent in the entity descriptions using blocks so they're easy to spot.

\item Entity instantiation code should include the port group names.

\item If words cannot describe how the code works, ASCII art diagrams might. 
This may seem a bit silly, but the only way to maintain up-to-date documentation 
is by having the documentation right in the developer's face. A picture 
somewhere in some documentation folder simply will not do. This manual is 
already a stretch.
\item Indentation is accomplished using two spaces, tabs are not used.

\item The \code{:} symbol in declarations, and the \code{=>} symbol in case
statements, port maps and generic maps, is generally aligned to column 33 using
spaces for aesthetically pleasing code.

\item Comments must wrap at column 80 for easy readability. Code should also not 
be too wide, although the column 80 limit is not strictly adhered to.

\end{itemize}

