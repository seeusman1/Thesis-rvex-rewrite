
\label{sec:rvsyssa}

The \rvex{} standalone processing system has the following features.

\begin{itemize}

\item Single cycle local instruction memory implemented in block RAMs.

\item Local data memory implemented in block RAMs that is single cycle for up to
two accesses at a time.

\item The initial contents of the local memories can be set.

\item Optionally, the cache can be instantiated. In this case, a unified
instruction/data memory is instantiated in block RAMs. The access latency of
this memory is configurable at runtime to mimic a more realistic memory access
latency for cache tests.

\item An external bus for peripherals or other memories may be connected through 
a bus master interface. Without the cache, the \rvex{} cannot read instructions 
from this bus, but it can access it using memory operations.

\item A slave bus interface allows access to the \rvex{} debug port, a trace
buffer, and the local memories, as well as the cache control register if the
cache is instantiated.

\item The cache, if instantiated, is coherent only for accesses made by the
\rvex{} itself. A cache flush is required using the cache control register if
the debug bus is used to write to the local memories.

\end{itemize}

\subsection{Instantiation template}
\label{sec:rvsyssa-inst}

The following listing serves as an instantiation template for the system. The 
code is documented in the following sections.

If you get errors when instantiating the core with this template, the
documentation might be out of date. Fear not, for the signals are also
documented in the entity description in \code{rvsys_standalone.vhd}.

\begin{lstlisting}[numbers=none]
library rvex;
use rvex.common_pkg.all;
use rvex.bus_pkg.all;
use rvex.bus_addrConv_pkg.all;
use rvex.core_pkg.all;
use rvex.cache_pkg.all;
use rvex.rvsys_standalone_pkg.all;

-- ...

rvex_standalone_inst: entity rvex.rvsys_standalone
  generic map (
    
    -- System configuration.
    CFG => rvex_sa_cfg(
      core => rvex_cfg(
        numLanesLog2              => 3,
        numLaneGroupsLog2         => 2,
        numContextsLog2           => 2
        -- ...
      ),
      core_valid => true
      -- ...
    ),
    CORE_ID                     => CORE_ID,
    PLATFORM_TAG                => PLATFORM_TAG,
    MEM_INIT                    => MEM_INIT
    
  )
  port map (
    
    -- System control.
    reset                       => reset,
    clk                         => clk,
    clkEn                       => clkEn,
    
    -- Run control interface.
    rctrl2rv_irq                => rctrl2rv_irq,
    rctrl2rv_irqID              => rctrl2rv_irqID,
    rv2rctrl_irqAck             => rv2rctrl_irqAck,
    rctrl2rv_run                => rctrl2rv_run,
    rv2rctrl_idle               => rv2rctrl_idle,
    rctrl2rv_reset              => rctrl2rv_reset,
    rctrl2rv_resetVect          => rctrl2rv_resetVect,
    rv2rctrl_done               => rv2rctrl_done,
    
    -- Peripheral interface.
    rvsa2bus                    => rvsa2bus,
    bus2rvsa                    => bus2rvsa,
    
    -- Debug interface.
    debug2rvsa                  => debug2rvsa,
    rvsa2debug                  => rvsa2debug
    
  );

\end{lstlisting}

\subsection{Interface description}
\label{sec:rvsyssa-iface}

As you can see in the template, the generics and signals are grouped by their 
function. The following subsections will document each group.

\subsubsection{System configuration}
\label{sec:rvsyssa-iface-generics}

These generics parameterize the system.

\begin{itemize}

\item \code{CFG : rvex_sa_generic_config_type}

This generic contains the configuration parameters for the core.
\code{rvex_sa_generic_config_type} is a \code{record} type with the following
members.

\begin{itemize}

\item \code{core : rvex_generic_config_type}

This parameter specifies the \rvex{} core configuration as passed to the bare 
\rvex{} processor core. Refer to 
Section~\ref{sec:core-ug-cfg-inst-iface-generics} for more information.

\item \code{cache_enable : boolean}

This parameter selects whether or not the cache should be instantiated. This is
false by default.

\item \code{cache_config : cache_generic_config_type}

This parameter specifies the size of the cache blocks. 
\code{cache_generic_config_type} is a record type with two \code{natural}-typed 
members: \code{instrCacheLinesLog2} and \code{dataCacheLinesLog2}. The sizes are
determined as follows.

\vskip -16 pt \begin{flalign*}
\mathrm{Instr.\ cache\ size} = 4 \cdot N_{lanes} \cdot 
2^{instrCacheLinesLog2} \cdot N_{laneGroups} &&
\end{flalign*}

\vskip -12 pt \noindent\begin{flalign*}
\mathrm{Data\ cache\ size} = 4 \cdot 
2^{dataCacheLinesLog2} \cdot N_{laneGroups} &&
\end{flalign*}

\noindent The number of lane groups is part of the equation because the number 
of lines are specified per block, and a different block is instantiated for each 
lane group.

\item \code{cache_bypassRange : addrRange_type}

This parameter specifies the range of addresses for which the cache (if 
instantiated) is bypassed. This range is \code{0x80000000..0xFFFFFFFF} by 
default.\code{addrRange_type} is a record containing four 
\code{rvex_address_type} members: \code{low}, \code{high}, \code{mask}, and 
\code{match}. An address is considered to be part of the range if the following 
VHDL expression is true.

\begin{lstlisting}[numbers=none]
unsigned(addr and mask) >= unsigned(low) and
unsigned(addr and mask) <= unsigned(high) and
std_match(addr, match)
\end{lstlisting}

\noindent This record may be set using the \code{addrRange} function, which
allows parameters to be omitted. The defaults for each parameter specify the
complete 32-bit address range, so it is usually sufficient to only set one or
two of the parameters.

\item \code{imemDepthLog2B : natural}
\item \code{dmemDepthLog2B : natural}

These parameters specify the sizes of the local instruction and data memories 
respectively if the cache is not used. Otherwise, \code{dmemDepthLog2B} 
specifies the size of the unified memory and \code{imemDepthLog2B} is ignored.
The size is specified as the logarithm of the number of bytes. The default value
is 16 for both of these, resulting in 64 kiB memories.

\item \code{traceDepthLog2B : natural}

This parameter specifies the size of the trace buffer in the same way that the
memory sizes are specified. The default value is 13, resulting in a trace buffer
8 kiB in size. This size is required if the serial debug interface is to be
used, due to the way in which bulk data transfers are implemented in the serial
protocol.

\item \code{debugBusMap_imem : addrRangeAndMapping_type}
\item \code{debugBusMap_dmem : addrRangeAndMapping_type}
\item \code{debugBusMap_rvex : addrRangeAndMapping_type}
\item \code{debugBusMap_trace : addrRangeAndMapping_type}

These parameters specify which addresses on the debug bus are mapped to which 
device. These parameters may be specified with the \code{addrRangeAndMap} 
function, which takes the same parameters as the \code{addrRange} function 
discussed for \code{cache_bypassRange}. In addition, it also allows the designer 
to change how the address bits are mapped from source to peripheral address. 
Refer to the comments in \code{bus_addrConv_pkg.vhd} for more information.

By default, the instruction memory is mapped to \code{0x10000000..0x1FFFFFFF} 
and to \code{0x30000000..0x3FFFFFFF}, the data memory is mapped to 
\code{0x20000000..0x3FFFFFFF}, the \rvex{} debug port is mapped to
\code{0xF0000000..0xFFFFFFFF} and the trace buffer is mapped to
\code{0xE0000000..0xEFFFFFFF}. Note that the range \code{0x30000000..0x3FFFFFFF}
maps to both the instruction and data memories. This range allows the
instruction and data memory to be written simultaneously, limiting the upload
time using the debug unit.

\item \code{debugBusMap_mutex : boolean}

This parameter specifies whether logic needed to handle overlaps in the debug 
bus address map is to be instantiated. If it is set to false, this logic is 
instantiated, allowing bus write commands to access multiple memories at the 
same time. This is the default. If it is set to true, overlaps are not
supported, but a some area may be saved.

\item \code{rvexDataMap_dmem : addrRangeAndMapping_type}
\item \code{rvexDataMap_bus : addrRangeAndMapping_type}

These parameters specify where data accesses from the \rvex{} are to be routed.
They work the same way as the \code{debugBusMap} parameters. By default, the
lower half of the address space, \code{0x00000000..0x7FFFFFFF}, is mapped to the
data memory, and the remainder is mapped to the bus. Overlaps are not allowed.
Accesses made to unmapped addresses cause a bus fault.

\end{itemize}

\noindent Typically, one will want to use the \code{rvex_sa_cfg} function to
specify this value. This function takes as its arguments values for all the
record members as specified above, but has default values for each of them,
meaning that not all of them have to be specified. In addition, a \code{base}
argument of type \code{rvex_generic_config_type} may be specified, which will be
used as the default value for unspecified parameters. This permits mutation of
the \code{CFG} record as it passes from entity to subentity, which is otherwise
impossible to do with record generics.

Important note: in order to allow the function to detect whether the \code{core} 
and \code{cache_config} fields are specified, the \code{core_valid} and 
\code{cache_config_valid} parameters must be set to true, or the defaults will 
be substituted!

\item \code{CORE_ID : natural}

This value is used to uniquely identify this core within a multicore platform. 
It is made available to the programs running on the core and the debug system 
through \creg{COID}.

\item \code{PLATFORM_TAG : std_logic_vector(55 downto 0)}

This value is to uniquely identify the platform as a whole. It is intended that
this value be generated by the toolchain by hashing the source files and
synthesis options. \todo{Reference version system.} It is made available to the
programs running on the core and the debug system through \creg{PTAG}.

\item \code{MEM_INIT : rvex_data_array}

This value is used to initialize the instruction and data memories. If left
unspecified, the memories are initialized to zero.

\end{itemize}

\subsubsection{System and run control interfaces}
\label{sec:rvsyssa-iface-sysrctrl}

These interfaces are identical to those specified for the bare \rvex{} core in 
Sections~\ref{sec:core-ug-cfg-inst-iface-syscon} and 
\ref{sec:core-ug-cfg-inst-iface-rctrl}.

\subsubsection{Peripheral and debug interfaces}
\label{sec:rvsyssa-iface-periph}

\begin{itemize}

\item \code{rvsa2bus : out bus_mst2slv_type}
\item \code{bus2rvsa : in bus_slv2mst_type}

These signals form a master \rvex{} bus device, allowing the \rvex{} to access 
memory or peripherals outside the processing system. A number of bus 
interconnection primitives are available in \code{rvex_rewrite/lib/rvex/bus}. 
Instantiation of these primitives is beyond the scope of this manual.

\item \code{debug2rvsa : in bus_mst2slv_type}
\item \code{rvsa2debug : out bus_slv2mst_type}

These signals form a slave \rvex{} bus device, allowing devices outside the 
processing system, such as the debug serial port peripheral, to access the local 
memories, trace buffer and the \rvex{} control registers.

The memory map of the debug interface is specified using generics. If the cache
is instantiated, The cache control register is mapped to the same address as
\creg{AFF}. Because \creg{AFF} is read-only and the cache control register is
write only, this does not cause conflicts. The cache control register has the
following layout.

\noindent\footnotesize
\begin{tabular}{@{}p{12mm}@{}*{32}{p{3.4mm}@{}}p{3mm}@{}}
 & & & & & & & & & & & & & & & & & & & & & & & & & & & & & & & & & \\
\multicolumn{1}{@{}l@{}|}{\scriptsize Offset \ } & \multicolumn{1}{@{}c@{}}{\tiny31} & \multicolumn{1}{@{}c@{}}{\tiny30} & \multicolumn{1}{@{}c@{}}{\tiny29} & \multicolumn{1}{@{}c@{}}{\tiny28} & \multicolumn{1}{@{}c@{}}{\tiny27} & \multicolumn{1}{@{}c@{}}{\tiny26} & \multicolumn{1}{@{}c@{}}{\tiny25} & \multicolumn{1}{@{}c@{}|}{\tiny24} & \multicolumn{1}{@{}c@{}}{\tiny23} & \multicolumn{1}{@{}c@{}}{\tiny22} & \multicolumn{1}{@{}c@{}}{\tiny21} & \multicolumn{1}{@{}c@{}}{\tiny20} & \multicolumn{1}{@{}c@{}}{\tiny19} & \multicolumn{1}{@{}c@{}}{\tiny18} & \multicolumn{1}{@{}c@{}}{\tiny17} & \multicolumn{1}{@{}c@{}|}{\tiny16} & \multicolumn{1}{@{}c@{}}{\tiny15} & \multicolumn{1}{@{}c@{}}{\tiny14} & \multicolumn{1}{@{}c@{}}{\tiny13} & \multicolumn{1}{@{}c@{}}{\tiny12} & \multicolumn{1}{@{}c@{}}{\tiny11} & \multicolumn{1}{@{}c@{}}{\tiny10} & \multicolumn{1}{@{}c@{}}{\tiny9} & \multicolumn{1}{@{}c@{}|}{\tiny8} & \multicolumn{1}{@{}c@{}}{\tiny7} & \multicolumn{1}{@{}c@{}}{\tiny6} & \multicolumn{1}{@{}c@{}}{\tiny5} & \multicolumn{1}{@{}c@{}}{\tiny4} & \multicolumn{1}{@{}c@{}}{\tiny3} & \multicolumn{1}{@{}c@{}}{\tiny2} & \multicolumn{1}{@{}c@{}}{\tiny1} & \multicolumn{1}{@{}c@{}|}{\tiny0} & \\
\cline{2-33}
\multicolumn{1}{@{}l@{}|}{\footnotesize \texttt{0x200}} & \multicolumn{8}{@{}c@{}|}{\tiny LAT} & \multicolumn{8}{@{}c@{}|}{} & \multicolumn{8}{@{}c@{}|}{\tiny DFL} & \multicolumn{8}{@{}c@{}|}{\tiny IFL} & \hspace{0.6 mm} \normalsize\footnotesize \\
\cline{2-33}
\end{tabular}
\normalsize\vskip 6pt

\noindent\textbf{LAT field, bits 31..24}

\noindent Must be written to a value between 1 and 254 inclusive for correct 
operation. That amount of cycles plus one are added to the bus access delay in 
case of a cache bypass, write or miss.

\noindent\textbf{DFL field, bits 15..8}

\noindent Each of these bits corresponds to an \rvex{} lane group. Writing a one 
to a bit causes the data cache block corresponding to the indexed lane group to 
be flushed. Writing a zero has no effect.

\noindent\textbf{IFL field, bits 7..0}

\noindent Each of these bits corresponds to an \rvex{} lane group. Writing a one 
to a bit causes the instruction cache block corresponding to the indexed lane 
group to be flushed. Writing a zero has no effect.

\end{itemize}

