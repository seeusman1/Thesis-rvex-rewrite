
\label{sec:rvsysgr}

The \rvex{} GRLIB-based processing system has the following features.

\begin{itemize}

\item One AHB master interface per \rvex{} lane group.

\item Cache snooping on the AHB bus guarantees cache coherency with other
processors and the debug interface sharing the same bus.

\item A LEON3 interrupt controller compatible interface is exposed. This allows
the \rvex{} to use the interrupt controller that comes with GRLIB.

\item For simulation, an S-record file specifying the expected memory contents
can be specified. Every instruction fetch and data access made by the \rvex{} is
snooped and checked against this memory. The memory automatically updates when
the \rvex{} writes a value. Whenever the cache returns an unexpected or
inconsistent value, a VHDL warning is printed.

\end{itemize}

\subsection{Instantiation template}
\label{sec:rvsysgr-inst}

The following listing serves as an instantiation template for the system. The 
code is documented in the following sections.

If you get errors when instantiating the core with this template, the
documentation might be out of date. Fear not, for the signals are also
documented in the entity description in \code{rvsys_standalone.vhd}.

\begin{lstlisting}[numbers=none]
library rvex;
use rvex.common_pkg.all;
use rvex.bus_pkg.all;
use rvex.bus_addrConv_pkg.all;
use rvex.core_pkg.all;
use rvex.cache_pkg.all;
use rvex.rvsys_grlib_pkg.all;

library grlib;
use grlib.amba.all;
use grlib.devices.all;

library gaisler;
use gaisler.leon3.all;

-- ...

rvex_grlib_inst: entity rvex.rvsys_grlib
  generic map (
    
    -- System configuration.
    CFG => rvex_grlib_cfg(
      core => rvex_cfg(
        numLanesLog2              => 3,
        numLaneGroupsLog2         => 2,
        numContextsLog2           => 2
        -- ...
      ),
      core_valid => true,
      cache => cache_cfg(
        instrCacheLinesLog2       => 18,
        dataCacheLinesLog2        => 18
      ),
      cache_valid => true
    ),
    PLATFORM_TAG                => PLATFORM_TAG,
    AHB_MASTER_INDEX_START      => RVEX_MST_INDEX,
    CHECK_MEM                   => false,
    CHECK_MEM_FILE              => ""
  )
  port map (
    
    -- System control.
    clki                        => clki,
    rstn                        => rstn,
    
    -- AHB interface.
    ahbmi                       => ahbmi,
    ahbmo                       => ahbmo_rvex,
    ahbsi                       => ahbsi,
    
    -- Debug interface.
    bus2dgb                     => bus2dgb,
    dbg2bus                     => dbg2bus,
    
    -- LEON3 compatible interrupt controller interface.
    irqi                        => irqi,
    irqo                        => irqo
    
  );

\end{lstlisting}

\subsection{Interface description}
\label{sec:rvsysgr-iface}

As you can see in the template, the generics and signals are grouped by their 
function. The following subsections will document each group.

\subsubsection{System configuration}
\label{sec:rvsysgr-iface-generics}

These generics parameterize the system.

\begin{itemize}

\item \code{CFG : rvex_grlib_generic_config_type}

This generic contains the configuration parameters for the core.
\code{rvex_grlib_generic_config_type} is a \code{record} type with the following
members.

\begin{itemize}

\item \code{core : rvex_generic_config_type}

This parameter specifies the \rvex{} core configuration as passed to the bare 
\rvex{} processor core. Refer to 
Section~\ref{sec:core-ug-cfg-inst-iface-generics} for more information.

\item \code{cache : cache_generic_config_type}

This parameter specifies the size of the cache blocks. 
\code{cache_generic_config_type} is a record type with two \code{natural}-typed 
members: \code{instrCacheLinesLog2} and \code{dataCacheLinesLog2}. The sizes are
determined as follows.

\vskip -16 pt \begin{flalign*}
\mathrm{Instr.\ cache\ size} = 4 \cdot N_{lanes} \cdot 
2^{instrCacheLinesLog2} \cdot N_{laneGroups} &&
\end{flalign*}

\vskip -12 pt \noindent\begin{flalign*}
\mathrm{Data\ cache\ size} = 4 \cdot 
2^{dataCacheLinesLog2} \cdot N_{laneGroups} &&
\end{flalign*}

\noindent The number of lane groups is part of the equation because the number 
of lines are specified per block, and a different block is instantiated for each 
lane group.

\end{itemize}

\noindent Similar to the bare \rvex{} and the standalone platform, the
\code{rvex_grlib_cfg} function is available to set this record.

Important note: in order to allow the function to detect whether the \code{core} and \code{cache} fields are specified, the \code{core_valid} and \code{cache_valid} parameters must be set to true, or the defaults will be substituted!

\item \code{PLATFORM_TAG : std_logic_vector(55 downto 0)}

This value is to uniquely identify the platform as a whole. It is intended that
this value be generated by the toolchain by hashing the source files and
synthesis options. \todo{Reference version system.} It is made available to the
programs running on the core and the debug system through \creg{PTAG}.

\item \code{AHB_MASTER_INDEX_START : natural}

This value must be set to the AHB master index of the first lane group. The 
remaining lane groups are mapped to subsequent master indices. In addition, this 
value is made available to the programs running on the core and the debug system 
through \creg{COID}, to allow a program to uniquely identify which \rvex{} it is 
running on in a multicore system.

\item \code{CHECK_MEM : boolean}
\item \code{CHECK_MEM_FILE : string}

These parameters configure the simulation-only memory consistency checking
system. \code{CHECK_MEM} enables or disables the system. If the system is
enabled, \code{CHECK_MEM_FILE} must specify the filename of an S-record file
holding the initial memory contents. The filename must be relative to the
simulator search path.

\end{itemize}

\subsubsection{System control interface}
\label{sec:rvsysgr-iface-sysctrl}

The system control signals include the clock source and the reset signal. All
registers are rising-edge triggered. The reset signal is active-low to comply
with the AHB standard. It is inverted in the system before it is passed to the
\rvex{} logic blocks, which assume an active-high reset signal.

\subsubsection{AHB interface}
\label{sec:rvsysgr-iface-ahb}

\begin{itemize}

\item \code{ahbmi : in ahb_mst_in_type}
\item \code{ahbmo : out ahb_mst_out_vector_type(}\textit{number of lane groups - 1}\code{ downto 0)}

These signals represent the AHB master interfaces that the cache blocks use as
their data source. One master interface is required for each \rvex{} lane group.
The master indices must be a contiguous range, starting at the index specified
by the \code{AHB_MASTER_INDEX_START} generic.

\item \code{ahbsi : in ahb_slv_in_type}

This signal must be tied to the signal that the AHB interconnect logic 
broadcasts to all AHB slaves. It is used by the cache to monitor the bus for 
cache coherence purposes.

\end{itemize}

\subsubsection{Debug interface}
\label{sec:rvsysgr-iface-debug}

\begin{itemize}

\item \code{bus2dgb : in bus_mst2slv_type}
\item \code{dbg2bus : out bus_slv2mst_type}

The debug interface is a slave \rvex{} bus device. It allows access to all 
control registers in the system and the trace buffer. It should be connected to 
the AHB bus using the \code{ahb2bus} bridge, possibly through additional \rvex{} 
bus interconnect primitives. This allows a single AHB to \rvex{} bus bridge to 
be used for multiple \rvex{} devices, similar to how an APB bridge allows 
multiple APB peripherals to share a single AHB slave interface.

The \rvex{} bus primitives are available in \code{rvex_rewrite/lib/rvex/bus}. 
Instantiation of these primitives is beyond the scope of this manual.

The debug interface port has a fixed memory map, shown below.

\newcommand{\memoryMapUsedEntry}[3]{\multirow{2}{*}{#3} & \code{#2} \\ & \code{#1} \\ \cline{1-1}}
\newcommand{\memoryMapUnusedEntry}{\emph{Unused} & ... \\ \cline{1-1}}
\vskip 6 pt
\noindent
\begin{tabular}{|l|l}
\cline{1-1}
\memoryMapUsedEntry{0x2000}{0x3FFF}{Trace buffer}
\memoryMapUsedEntry{0x1D00}{0x1FFF}{\rvex{} context 7 registers}
\memoryMapUnusedEntry{}
\memoryMapUsedEntry{0x1900}{0x1BFF}{\rvex{} context 6 registers}
\memoryMapUnusedEntry{}
\memoryMapUsedEntry{0x1500}{0x17FF}{\rvex{} context 5 registers}
\memoryMapUnusedEntry{}
\memoryMapUsedEntry{0x1100}{0x13FF}{\rvex{} context 4 registers}
\memoryMapUnusedEntry{}
\memoryMapUsedEntry{0x0D00}{0x0FFF}{\rvex{} context 3 registers}
\memoryMapUnusedEntry{}
\memoryMapUsedEntry{0x0900}{0x0BFF}{\rvex{} context 2 registers}
\memoryMapUnusedEntry{}
\memoryMapUsedEntry{0x0800}{0x0803}{Cache control register}
\memoryMapUsedEntry{0x0500}{0x07FF}{\rvex{} context 1 registers}
\memoryMapUnusedEntry{}
Reset register & \code{0x0400} \\ \cline{1-1}
\memoryMapUsedEntry{0x0100}{0x03FF}{\rvex{} context 0 registers}
\memoryMapUsedEntry{0x0000}{0x00FF}{\rvex{} global control registers}
\end{tabular}
\vskip 6 pt

\noindent The reset register resets the entire processing system when written.
The only thing that is not reset is the AHB bus bridge to prevent deadlocks due
to the AHB bus interconnect not being reset.

The cache control register has the following layout. Note that it is different
from the standalone system cache control register.

\noindent\footnotesize
\begin{tabular}{@{}p{12mm}@{}*{32}{p{3.4mm}@{}}p{3mm}@{}}
 & & & & & & & & & & & & & & & & & & & & & & & & & & & & & & & & & \\
\multicolumn{1}{@{}l@{}|}{\scriptsize Offset \ } & \multicolumn{1}{@{}c@{}}{\tiny31} & \multicolumn{1}{@{}c@{}}{\tiny30} & \multicolumn{1}{@{}c@{}}{\tiny29} & \multicolumn{1}{@{}c@{}}{\tiny28} & \multicolumn{1}{@{}c@{}}{\tiny27} & \multicolumn{1}{@{}c@{}}{\tiny26} & \multicolumn{1}{@{}c@{}}{\tiny25} & \multicolumn{1}{@{}c@{}|}{\tiny24} & \multicolumn{1}{@{}c@{}}{\tiny23} & \multicolumn{1}{@{}c@{}}{\tiny22} & \multicolumn{1}{@{}c@{}}{\tiny21} & \multicolumn{1}{@{}c@{}}{\tiny20} & \multicolumn{1}{@{}c@{}}{\tiny19} & \multicolumn{1}{@{}c@{}}{\tiny18} & \multicolumn{1}{@{}c@{}}{\tiny17} & \multicolumn{1}{@{}c@{}|}{\tiny16} & \multicolumn{1}{@{}c@{}}{\tiny15} & \multicolumn{1}{@{}c@{}}{\tiny14} & \multicolumn{1}{@{}c@{}}{\tiny13} & \multicolumn{1}{@{}c@{}}{\tiny12} & \multicolumn{1}{@{}c@{}}{\tiny11} & \multicolumn{1}{@{}c@{}}{\tiny10} & \multicolumn{1}{@{}c@{}}{\tiny9} & \multicolumn{1}{@{}c@{}|}{\tiny8} & \multicolumn{1}{@{}c@{}}{\tiny7} & \multicolumn{1}{@{}c@{}}{\tiny6} & \multicolumn{1}{@{}c@{}}{\tiny5} & \multicolumn{1}{@{}c@{}}{\tiny4} & \multicolumn{1}{@{}c@{}}{\tiny3} & \multicolumn{1}{@{}c@{}}{\tiny2} & \multicolumn{1}{@{}c@{}}{\tiny1} & \multicolumn{1}{@{}c@{}|}{\tiny0} & \\
\cline{2-33}
\multicolumn{1}{@{}l@{}|}{\footnotesize \texttt{0x0800}} & \multicolumn{8}{@{}c@{}|}{\tiny IFL} & \multicolumn{8}{@{}c@{}|}{\tiny DFL} & \multicolumn{15}{@{}c@{}|}{} & \multicolumn{1}{@{}c@{}|}{\tiny B} & \hspace{0.6 mm} \normalsize\footnotesize \\
\cline{2-33}
\end{tabular}
\normalsize\vskip 6pt

\noindent\textbf{IFL field, bits 7..0}

\noindent Each of these bits corresponds to an \rvex{} lane group. Writing a one 
to a bit causes the instruction cache block corresponding to the indexed lane 
group to be flushed. Writing a zero has no effect. The register always reads as
0.

\noindent\textbf{DFL field, bits 15..8}

\noindent Each of these bits corresponds to an \rvex{} lane group. Writing a one 
to a bit causes the data cache block corresponding to the indexed lane group to 
be flushed. Writing a zero has no effect. The register always reads as 0.

\noindent\textbf{B flag, bit 0}

\noindent When this bit is set, the data cache is always bypassed. When it is 
cleared, the cache is only bypassed for memory accesses to the upper half of the 
address space, i.e. \code{0x80000000..0xFFFFFFFF}. The flag resets to 0.

\end{itemize}

\subsubsection{Interrupt controller interface}
\label{sec:rvsysgr-iface-irq}

\begin{itemize}

\item \code{irqi : in  irq_in_vector(0 to }\textit{number of contexts - 1}\code{)}
\item \code{irqo : out irq_out_vector(0 to }\textit{number of contexts - 1}\code{)}

These signals should be tied to a GRLIB \code{irqmp} interrupt controller, with 
the number of processors set to the number of \rvex{} contexts.

\end{itemize}
