\label{sec:rvd}

There are two main approaches to debugging the \rvex{} processor. This chapter 
documents the external debugger approach. In this approach, a computer is 
connected to the \rvex{} is used to debug the processor and the software running 
on it. The computer is connected to the \rvex{} using some interface, usually a
serial port in the case of the \rvex{}. The alternative approach is called
self-hosted debug, where the debugger runs on the \rvex{} itself in order to
debug another thread. However, this approach requires a sophisticated
multithreading operating system, such as a Linux kernel with the \code{ptrace}
system call implemented for the \rvex{}. Although the hardware should be ready
for such a system, the software for it has not yet been implemented.

\section{Setting up}
\label{sec:rvd-start}

The connection between the computer and the \rvex{} is called the debug link.
Currently, the following options exist.

\begin{itemize}
\item A serial port, through the \rvex{} debug support peripheral.
\item PCI express, developed in \cite{vanderwijst2015}.
\item Memory mapped on a Zynq FPGA, running Linaro Linux with \code{rvsrv} on
the embedded ARM processor. The debug commands may be given in Linaro, or
\code{rvd} can connect to the Zynq development board using ethernet.
\end{itemize}

\noindent Which connections are supported depends on the platform. The serial
port option is available in all hardware platforms except for
\code{zed-almarvi}. The PCI express connection is supported in addition to
the the serial link by \code{ml605-grlib} to allow faster memory access.
\code{zed-almarvi} only supports the memory-mapped option.

Whichever platform you use, you need to execute the following commands in a
console from the root directory of the platform you are using to set up the
debugging environment.

\begin{lstlisting}[numbers=none, language=nothing]
make debug
source debug
\end{lstlisting}

\noindent The first command generates a script called `debug' that sets up 
environment variables to allow you to use \code{rvd}. The second command runs 
that script. The next step depends on whether the FPGA board is connected to 
your machine (Section~\ref{sec:rvd-start-local}) or to another machine 
(Section~\ref{sec:rvd-start-remote}). In the latter case, you need to be able to 
\code{ssh} to that machine.

\subsection{Connecting to a remote machine}
\label{sec:rvd-start-remote}

To connect to the remote machine, we will use \code{ssh} to forward two TCP/IP 
ports. You can do this by running the following command in a second terminal 
(you will need to keep it running), obviously replacing \code{<user@host>} with 
the computer you are connecting to and your account name on that computer.

\begin{lstlisting}[numbers=none, language=nothing]
ssh -N -L 21078:localhost:21078 -L 21079:localhost:21079 <user@host>
\end{lstlisting}

\noindent Note that you will \emph{not} drop to a terminal on the remote 
computer as \code{ssh} normally does. It will appear like it is not doing 
anything after requesting your password (if required). You can test the 
connection by running \code{rvd ?} in the original terminal. If that does not 
crash with the message \code{Failed to connect to rvsrv}, you are ready to move 
on to Section~\ref{sec:rvd-run}. Otherwise, \code{ssh} is not working, or more 
likely, \code{rvsrv} is not running on the remote machine. In the latter case, 
you can try to start it yourself by \code{ssh}'ing to the machine normally and 
following the steps in \ref{sec:rvd-start-local}. If that does not work, you 
will have to ask the owner of the machine for help.

\subsection{Connecting to the FPGA}
\label{sec:rvd-start-local}

This section assumes that you are using a serial port debug link. The PCI
express connection is more complicated to set up due to the drivers required.
If you are using the Zedboard, refer to the separate documentation in the
\code{zed-almarvi} platform.

If this is the first time you are connecting to the FPGA, open the following 
file in a text editor.

\begin{lstlisting}[numbers=none, language=nothing]
<rvex-rewrite>/tools/debug-interface/configuration.cfg
\end{lstlisting}

\noindent If this file does not exist, create it by copying 
\code{default-configuration.cfg} from the \code{src} directory. This file 
describes the interfaces that the debug server (\code{rvsrv}) will connect to or 
expose. The relevant configuration key is \code{SERIAL_PORT}, which needs to be
set to the \code{tty} corresponding to the serial port.

When that has been configured, the debug server can be started in the terminal
in which we have sourced the \code{debug} script using the following command.

\begin{lstlisting}[numbers=none, language=nothing]
make server
\end{lstlisting}

\noindent You can now test the connection to the \rvex{} by running
\code{rvd ?}.

\section{Running programs}
\label{sec:rvd-run}

The procedure for uploading and running a program differs from platform to
platform, but usually, the following three commands will work.

\begin{lstlisting}[numbers=none, language=nothing]
make upload-<program>
make start-<program>
make run-<program>
\end{lstlisting}

\noindent The difference between them is that \code{upload} only uploads the 
program to the \rvex{} without starting it, \code{start} uploads and then starts 
the program, and \code{run} also waits for completion and prints the performance 
counter values. Usually, running \code{make} without parameters will (among 
other things) print a list of the available programs.

\section{Debugging programs}
\label{sec:rvd-debug}

The standard and recommended way to send debug commands to the \rvex{} is to use
\code{rvd}. All documentation for using \code{rvd} is embedded inside the
program: just run \code{rvd help}. To get command specific documentation, use
\code{rvd help <command>}.

\code{rvd} has builtin commands for halting, resuming, single stepping, 
resetting execution and printing the current state of the processor, in addition 
to the raw memory access commands. More complicated things, such as breakpoints, 
need to be set manually by accessing the control registers of the \rvex{}. You 
do not have to remember the control register addresses by heart though; you can 
use the control register names without \code{CR_} prefix directly.

\code{rvd} has a concept of contexts. By default, the debug interface for 
context 0 is used. To select a different context, you can either specify the 
context using the \code{-c} command line parameter (for example, \code{rvd -c3 
resume}) or you can set it for future commands using the \code{rvd select} 
command. In addition to specifying a single context, you can also specify a 
range of contexts (\code{<from>..<to>} or all contexts (\code{all}). When more
than one context is selected, \code{rvd} will simply execute the given command
for all selected contexts sequentially.

An alternative to \code{rvd}'s interface, the \code{gdb} port can be used. In
this case, the following command should be used.

\begin{lstlisting}[numbers=none, language=nothing]
rvd gdb -- <path_to_gdb> [parameters passed to gdb]
\end{lstlisting}

\noindent This runs \code{gdb} as a child process to \code{rvd}. The appropriate
parameters are passed to \code{gdb} to have it connect to \code{rvd} using the
remote serial protocol, in addition to the parameters specified on the command
line. A description of how to use the \rvex{} \code{gdb} port is beyond the
scope of this manual.

\section{Tracing execution}
\label{sec:rvd-trace}

The \rvex{} can be configured at design time to include a trace unit. This
allows the hardware to output a stream of data describing everything that the
processor is doing at various levels of detail. \code{rvd} supports tracing
using the following command.

\begin{lstlisting}[numbers=none, language=nothing]
rvd trace <output_file> [level_of_detail] [condition]
\end{lstlisting}

\noindent When executed, \code{rvd} writes the specified level of detail or 1
by default to the trace control byte in \creg{DCR2}. It then resumes execution
on the selected contexts and reads data from the trace buffer. Tracing stops
when the specified condition evaluates to 0, or when no more data is available
if no condition is specified.

Terminating a trace with \code{ctrl+c} is not recommended, because it prevents
\code{rvd} from resetting the trace control byte and emptying the trace buffer.
To terminate a trace gracefully when no condition is specified and the program
is stuck in a loop, run \code{rvd break} in a separate terminal. This will make
\code{rvd trace} assume that the program has finished executing.

\code{rvd trace} dumps the raw trace data to a file. This file can be converted
to a human readable format using the \code{rvtrace} tool. If a disassembly file
generated using \code{objdump -d} is specified in addition to the binary trace
file, the disassembled instructions will be included in the trace output file.

Please note that the human readable trace files are much larger than the binary
data format. It may thus take some time and a lot of disk space to generated the
human readable file. You may want to pipe the output of \code{rvtrace} to
\code{less} instead, so the output will only be saved in memory.

% \section{Under the hood}
% 
% 
% \label{sec:rvd-debuglink}
% 
% The serial port protocol implemented by the hardware debug support peripheral
% and \code{rvsrv} supports the following things.
% 
% \begin{itemize}
% \item Any single bus transaction.
% \item Bulk memory access. This is faster than single transactions, but it is not
% guaranteed that the memory is accessed only once and/or in order. That is fine
% for normal memory access and most register reads in the \rvex{}, but might not
% be good enough for memory writes.
% \item The debugged application can use the debug support peripheral as a normal
% serial port peripheral.
% \end{itemize}
% 
% \noindent All memory transactions are error-checked and corrected at the serial 
% port level. The application stream, however, is transmitted raw and is not 
% error-checked. It is expected that the software running on the \rvex{} will
% implement its own error correction protocol in applications where serial port
% errors are significant.
% 
% 
% \section{rvsrv}
% \label{sec:rvd-debuglink}
% 
% 
% \code{rvsrv} is a command line program that runs in the background to expose the
% debug link to other programs using a generic TCP/IP interface. It exists
% primarily to allow multiple programs access to the debug link simultaneously,
% but the TCP/IP interface also makes it suitable for debugging an \rvex{} running
% on an FPGA connected to another computer (such as a board server).
% 
% \code{rvsrv} exposes two TCP/IP servers on port 21078 and
% 21079 by default, for debug software to connect to. The first, called the
% application port, connects to the virtual serial port peripheral exposed to
% the debugged application. If multiple TCP clients connect to the port, data from
% the \rvex{} application is broadcasted to all clients, and any data sent by a
% client is sent to the \rvex{} application. \code{netcat} may be used to
% visualize the serial port traffic.
% 
% The second port is called the debug port. It supports a set of ASCII commands to
% request bus transactions on the \rvex{}. All commands are initiated by the
% client and responded to by \code{rvsrv}. Normally, \code{rvd} is used to connect
% to this port, but developers may also write their own interfaces. The
% \code{rvsrv} debug port protocol is described in Section~\ref{sec:rvd-dbgproto}.
% 
% \section{Debugging}
% \label{sec:rvd-rvsrv}
% 
% 
% 
% \section{rvsrv debug port protocol}
% \label{sec:rvd-dbgproto}
% 
% Kokolores.
