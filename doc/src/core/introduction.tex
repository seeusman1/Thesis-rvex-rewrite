\documentclass[main.tex]{subfiles}

\begin{document}

\section{Introduction}

The \rvex{} core is an optionally runtime-reconfigurable VLIW processor based on the HP VEX architecture. Aside from being runtime-reconfigurable, the core has several design-time configurable parameters by means of generics, and many more are available through package constants. A coarse list of configuration parameters is listed below; this list is not exhaustive.

\begin{itemize}
  
  \item Runtime-reconfigurable parameters\footnote{The degree in which these parameters are configurable depends on design-time configuration of the core.}:
  \begin{itemize}
    \item Number of lanes active.
    \item Context to lane mapping: multiple contexts can be run at once, or a single context can run on all lanes.
  \end{itemize}
  
  \item Design-time-configurable parameters through generics:
  \begin{itemize}
    \item Number of lanes: 2, 4, 8 or 16.
    \item Number of hardware contexts (to switch between and/or run in parallel): 1, 2, 4, 8 or 16.
    \item Number of lane groups; determines the degree of reconfigurability.
    \item Lane resource configuration: each lane within a group can optionally have a multiplier, and it is configurable which lane within the group has a memory interface and which can handle branch operations.
    \item Long immediate forwarding options.
    \item Whether forwarding logic is instantiated or not.
  \end{itemize}
  
  \item Design-time-configurable parameters through package constants:
  \begin{itemize}
    \item Instruction set and opcode mapping\footnote{Modifications to the datapath require changes to behavioral VHDL code.}.
    \item Pipeline and forwarding configuration, including memory bus latency: the timing of all functional units within the pipeline can be changed as much as the functional units permit, to augment the timing characteristics of the core without changing behavioral VHDL code.
  \end{itemize}

\end{itemize}

todo: short history about the core; previous versions and references

This document serves as basic documentation for the \rvex{} core files in \code{/lib/rvex/core}. More detailed (and probably more up-to-date) documentation can be found in the VHDL source code itself. This document does not provide documentation for the support packages within the \code{rvex} library.

\end{document}
