\documentclass[main.tex]{subfiles}

\begin{document}

\section{Naming conventions}

All entities, packages, labels, types and hierarchy-crossing signals use the following naming conventions.

\begin{itemize}

  \item All signals, entity names, package names and types use a combination of camelCase and underscores. Typically, underscores are used as a form of hierarchy separation, where the VHDL language does not otherwise allow it, and camelCase is used to indicate word boundaries within one level of hierarchy. For example, \code{core_ctrlRegs_bank} refers to a (register) bank for the control registers for the \rvex{} core.
  
  \item Most signal names start with an underscore-terminated abbreviation which indicates the source and destination entity. This identifier contains two entity abbreviation codes separated by a \code{2}. The entity abbreviation codes are defined at the top of \code{core.vhd} and are also listed in Section \ref{sec:overview}. Sometimes other abbreviations are used for signals local to one entity, which should be clear from context.
  
  \item All constant names are uppercase with underscores.

  \item Labels typically use underscores only, to prevent conflicts between similarly named entities and signals.

  \item Types use one of the following suffixes to indicate what kind of type they represent.
  \begin{itemize}
    \item \code{_type}: scalar type.
    \item \code{_array}: array type.
    \item \code{_ptr}: access type for a scalar.
    \item \code{_array_ptr}: access type for an array.
  \end{itemize}

  \item Enties and packages have the same name as their filename; exactly one entity or package is defined per VHDL file. Package names end in \code{_pkg}. All \rvex{} core files start with \code{core_} to keep their names unique within the \code{rvex} package, which contains a number of supporting packages as well.

\end{itemize}

\end{document}
