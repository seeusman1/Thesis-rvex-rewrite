\chapter{\rvex{} core internals}
\label{sec:core-int}

This chapter documents how the \rvex{} processor core works internally. Refer to
Chapter~\ref{sec:core-ug} instead if you are only interested in using the core
as is.

The \hyperref[sec:core-int-overview]{first section} of this chapter gives an 
architectural overview of the core, lists the VHDL files that represent the 
core and deals with code style. The \hyperref[sec:core-int-datapath]{second} and 
\hyperref[sec:core-int-flow]{third} sections together describe how instructions 
are executed; the former documents the datapath (tentatively, the lifespan of an 
instruction) while the latter documents how the next instruction is chosen. The 
\hyperref[sec:core-int-reconf]{fourth section} deals with the reconfiguration 
system of the core, and what the interconnect between the lane groups and the 
contexts looks like. Finally, the \hyperref[sec:core-int-debug]{fifth section} 
documents the external debug and trace interfaces.

\clearpage
\section{Overview}
\newcommand{\coreoverviewintro}{The last section documents the coding style
employed within all \rvex{} core files.}
\subimport{1-overview/}{core-int-overview.tex}
\subsection{Coding style}
\subimport{1-overview/}{core-int-overview-style.tex}

\clearpage
\section{Datapath}
\subimport{2-datapath/}{core-int-datapath.tex}

\clearpage
\section{Flow control}
\subimport{3-flow/}{core-int-flow.tex}

\clearpage
\section{Reconfiguration}
\subimport{4-reconf/}{core-int-reconf.tex}

\clearpage
\section{External debug and trace interface}
\subimport{5-debug/}{core-int-debug.tex}
