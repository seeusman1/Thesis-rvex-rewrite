\subsection{Registers}
\label{sec:core-ug-isa-regs}

The \rvex{} processor has five distinguishable register files. Each is described
below.

\subsubsection{General purpose registers}
\label{sec:core-ug-isa-regs-gp}

The \rvex{} core contains 64 32-bit general purpose registers for arithmetic.

Register 0 is special, as it always reads as 0 when used by the processor.
Writing to it does however work; the debug bus can read the latest value written
to it. This allows the register to be used for debugging on rare occasions.

Register 1 is intended to be used as the stack pointer. The \insn{RETURN} and
\insn{RFI} instructions can add an immediate value to it for stack adjustment,
but otherwise it behaves just as any other general purpose register.

Register 63 can optionally be mapped to the link register at design time using
generics. This allows arithmetic instructions to be performed on the link
register without needing to use \insn{MOVFL} and \insn{MOVTL}, at the cost of a
general purpose register.

There are no explicit move or load-immediate operations, as the following
syllables are already capable of these operations.

\begin{lstlisting}[numbers=none, basicstyle=\footnotesize, language=vexasm]
c0 or $r0.dest = $r0.0, $r0.src        // Move src to dest
c0 or $r0.dest = $r0.0, immediate      // Load immediate
\end{lstlisting}

\subsubsection{Branch registers}
\label{sec:core-ug-isa-regs-br}

The \rvex{} core contains 8 1-bit registers used for branch conditions, select
instructions, divisions, and additions of values wider than 32 bits.

All arithmetic operations that output a boolean value can write to either a
general purpose register (in which case they will write 0 for false and 1 for
true) or a branch register. These include all integer comparison operations and
select boolean operations.

Moving a branch register to another branch register cannot be done in a single
cycle, but loading an immediate into a branch register or moving to or from a
general purpose register can be done as follows.

\begin{lstlisting}[numbers=none, basicstyle=\footnotesize, language=vexasm]
c0 cmpeq $b0.dest = $r0.0, $r0.0       // Load true
c0 cmpne $b0.dest = $r0.0, $r0.0       // Load false
c0 cmpne $b0.dest = $r0.0, $r0.src     // Move general purpose to branch
c0 slctf $r0.dest = $b0.src, $r0.0, 1  // Move branch to general purpose
\end{lstlisting}

Branch register can also not be loaded from or stored into memory on their own.
However, to improve context switching speed slightly, the \insn{LDBR} and
\insn{STBR} instructions are available. These load or store a byte containing
all eight branch registers in a single syllable.

\subsubsection{Link register}
\label{sec:core-ug-isa-regs-lr}

The link register is a 32-bit register used to store the return address when
calling. It can also be used as the destination address for an unconditional
indirect jump or call, in cases where the branch offset field is too small or
when the jump target is determined at runtime.

When general purpose register 63 is not mapped to the link register, the 
\insn{MOVTL} and \insn{LDW} instructions can be used to load the link register 
from a general purpose register or memory respectively. \insn{MOVFL} and 
\insn{STW} perform the reverse operations.

\subsubsection{Global and context control registers}
\label{sec:core-ug-isa-regs-creg}

These two register files contain special-purpose registers. The global control
registers contain status information not specific to any context, whereas the
context control registers are context specific.

The processor can access these register files through memory operations only. 
All these accesses are single-cycle. 1 kiB of memory space has to be reserved 
for this purpose, usually mapped to \code{0xFFFFFC00..0xFFFFFFFF}. The 
location of the block is design-time configurable. Note that it is impossible 
for the processor to perform actual memory operations to this region, so the 
location of the block should be chosen wisely.

The global register file is read-only from the perspective of the program. The 
context register file is writable, but it should be noted that each program can 
only access its own hardware context register file. If an application requires 
that programs can write to the global register file or the other context 
register files, the debug bus can be made accessible for memory operations by 
the bus interconnect outside the core. In most platforms this happens 
coincidentally, as the processor can access the main bus of the platform, and 
the debug bus is wired as a slave peripheral on this bus. For more information
about the debug bus, refer to Section~\ref{sec:core-ug-cfg-inst-iface-debug}.
 
For more information about the control registers in general, refer to
Section~\ref{sec:core-ug-creg}.
