\subsection{Instruction set}
\label{sec:core-ug-isa-insn-set}

% This defines the statistic commands \instructioncount and \freeopcodecount. It
% doesn't contain any text.
\subimport{../../../generated/}{opcode-stats.generated.tex}

The \rvex{} instruction set consists of \instructioncount{} instructions. These
instructions are defined by two bitfields in the syllable, called \code{opcode}
and \code{imm_sw}. The \code{opcode} field is 8 bits in size, ranging from bit
31 to 24 inclusive, allowing for 256 different operations to be performed.
\code{imm_sw} is a single bit (bit 23) that specifies if the second operand is a
register or an immediate. This thus allows a total of 512 different
instructions in theory.

However, not all operations support both register and immediate mode. In
addition, some instructions have operand fields that extend into the
\code{opcode}, requiring a single instruction to use multiple opcodes. Taking
these things into consideration, the \rvex{} instruction set has
\freeopcodecount{} \code{opcode}s that are not yet mapped.

There are two additional fields with a fixed function within the instruction
set. The first is the stop bit, bit 1. This bit determines where the bundle
boundaries are. Refer to Section~\ref{sec:core-ug-isa-sbit} for more
information. The second field, bit 0, is reserved for cluster end bits. The
toolchain currently always outputs a 0 bit, and the processor ignores it
completely.

The following table lists all the instructions in the \rvex{} instruction set
ordered by opcode. The subsequent sections document each instruction, ordered by
function. If you are reading this document digitally, you can click any
instruction in the table to jump to its documentation.

% This includes the long instruction table.
\subimport{../../../generated/}{opcode-table.generated.tex}

% This includes the instruction documentation. \insndocsection aliases the right
% section type to use for the instruction groups.
\newcommand{\insndocsection}[1]{\subsubsection{#1}}
\subimport{../../../generated/}{opcode-docs.generated.tex}

