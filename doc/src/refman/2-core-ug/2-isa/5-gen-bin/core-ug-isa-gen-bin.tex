
\subsection{Generic binaries}
\label{sec:core-ug-isa-gen-bin}

\todo[inline]{This section might contain trace amounts of bullshit. Anthony
should probably look over this sometime, as he knows more about it than I do.
The vexparse stuff might also be wrong because I can't remember properly what
I all added to it and what it could already do and I haven't looked into it yet.
- Jeroen}

Generic binaries are binaries that can be correctly run on different core 
configurations, even if the core reconfigures during execution. They were 
introduced in Anthony's generic binary paper. \todo{Insert reference to Anthony's 
generic binary paper} Typically, a generic binary refers to a binary that can 
be run with two pipelanes (2-way), four pipelanes (4-way) or eight pipelanes 
(8-way).

A generic binary is typically compiled in the same way as a regular 8-way
binary. It is the task of the assembler to ensure that the generic binary
requirements are met. For the standard generic binary, these rules are the
following.

\begin{itemize}

\item \emph{The single branch instruction allowed per bundle must end up in the 
last execution cycle in 2-way and 4-way execution.} The \rvex{} processor
imposes the even stricter requirement that branch syllables must always be the
last syllable in a bundle.

\item \emph{RAW hazards must be avoided in all runtime configurations.} That is,
for example, a register that is written in one of the first two syllables may
not be read in subsequent slots. This is because the old value of the register
would be read in 8-way mode, but the newly written value would be read in 2-way
mode.

\end{itemize}

\noindent Extrapolating these rules to the general case should be trivial.

\subsubsection{Generating generic binaries}
\label{sec:core-ug-isa-gen-gen}

In order to generate generic binaries, the \code{-u} flag needs to be passed to
the assembler. By default, the assembler will only try to move syllables around
within bundles in order to meet the requirements imposed above. However, often
this is not possible without further processing.

There are two ways to process the assembly files to meet the requirements. The
first one can be done by the assembler as well. If the \code{--autosplit} flag
is passed, it will attempt to split bundles that it cannot schedule directly.
This solves most problems at the cost of runtime performance. Refer to Anthony's
generic binary paper\todo{Insert reference to Anthony's generic binary paper}
for more information.

The second way involves running a python script called \code{vexparse} on the 
assembly compilation output, before passing them to the assembler. Depending on 
its configuration, \code{vexparse} will extract a dependency graph of all 
syllables in a basic block\footnote{A basic block is a block of instructions 
with natural scheduling boundaries at the start and end of it. The prime example 
of such boundaries are branch instructions.} from the assembly code, and then 
completely reschedule all instructions. As a side effect, it will fix 
hand-written assembly code that failed to take multiply and load instruction 
delays into consideration.

Being a python script, \code{vexparse} is much slower than the 
\code{--autosplit} option of the assembler. However, it generates more efficient 
code, as it is not limited to merely splitting bundles.
