 
\subsection{Assembly syntax}
\label{sec:core-ug-isa-assembly}

The following listing shows the syntax for a single instruction bundle.

\begin{lstlisting}[numbers=none, basicstyle=\footnotesize, language=vexasm]
start:
  c0 stw  0x10[$r0.1] = $r0.53
  c0 add  $r0.3       = $r0.0, $r0.55
  c0 and  $b0.2       = $r0.0, $r0.10
  c0 call $l0.0       = interrupt
;;
\end{lstlisting}

The first line represents a label, as it ends in a colon. Each non-empty line 
that does not start with a semicolon and is not a label represents a syllable. 
The first part of the syllable, \code{c0}, is optional. It specifies the cluster 
that the syllable belongs to. Since the \rvex{} processor currently does not 
support clusters, only cluster zero is allowed if specified. The second part 
represents the opcode of the syllable, defining the operation to be performed. 
The third part is the parameter list. Anything that is written to is placed 
before the equals sign, anything that is read is placed after. Finally, a 
double semicolon is used to mark bundle boundaries.

The syntax for a general purpose register is \code{$r0.}\textit{index}, where 
\textit{index} is a number from 0 to 63. The first 0 is used to specify the 
cluster, which, again, is not used in the \rvex{} processor. Branch registers 
and the link register have the same syntax, substituting the `r' with a `b' or 
an `l' respectively. The \textit{index} for branch registers ranges from 0 to 7. 
For link registers only 0 is allowed.

Memory references use the following syntax:
\textit{literal}\code{[$r0.}\textit{index}\code{]}. At runtime, the literal is
added to the register value to get the address.

Any literal may be a decimal or hexadecimal number (using \code{0x} notation),
a label reference, or a basic C-like integer expression.

A port of the GNU assembler (\code{gas}) is used for assembly. Please refer to
its manual for information on target-independent directives or more information
on the expressions mentioned above.

In general, the C preprocessor is used to preprocess assembly files. This allows
usage of the usual C-style comments, includes, definitions, etc. In particular,
the control registers may be easily referenced as long as the appropriate files
are included.

