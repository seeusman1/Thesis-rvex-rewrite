
\subsubsection{Debug bus interface}
\label{sec:core-ug-cfg-inst-iface-debug}

The debug bus provides an optional slave bus interface capable of accessing most of the registers within the core.

\begin{itemize}
  
\item \code{dbg2rv_addr : in  rvex_address_type}
\item \code{dbg2rv_readEnable : in std_logic}
\item \code{dbg2rv_writeEnable : in std_logic}
\item \code{dbg2rv_writeMask : in rvex_mask_type}
\item \code{dbg2rv_writeData : in rvex_data_type}
\item \code{rv2dbg_readData : out rvex_data_type}

Debug interface bus. \code{dbg2rv_readEnable} and \code{dbg2rv_writeEnable} are 
active high and should not be active at the same time. \code{rv2dbg_readData} is 
valid one \code{clkEn}abled cycle after \code{dbg2rv_readEnable} is asserted and 
contains the data read from \code{dbg2rv_addr} as it was while 
\code{dbg2rv_readEnable} was asserted. \code{dbg2rv_writeMask}, 
\code{dbg2rv_writeData} and \code{dbg2rv_addr} define the write request when 
\code{dbg2rv_writeEnable} is asserted. All input signals are tied to \code{'0'} 
when not specified.

The debug bus can read from and write to all \rvex{} registers. 1024 bytes are 
used per context, thus the size of the debug bus control register block is $1024 
\cdot numContexts$ bytes. As the upper address bits are simply ignored, this 
block is mirrored across the full 32-bit address space.

The memory map of an \rvex{} with two contexts is shown in 
Table~\ref{tbl:core-ug-cfg-inst-iface-debug-map}. Note that the mappings per 
context equal those of direct accesses to the control registers from the \rvex{} 
memory units (Section~\ref{sec:core-ug-isa-regs-creg}), with the addition of the 
general purpose registers. Additional contexts specified at design time simply 
appear after the first two.

\begin{table}[h]
\centering
\caption{Debug bus memory map for 2 contexts.}
\label{tbl:core-ug-cfg-inst-iface-debug-map}
\begin{tabular}{| l | l |}
\hline
\textbf{Address}     & \textbf{Mapping} \\ \hline
\texttt{0x000-0x0FF} & Global control registers \\ \hline
\texttt{0x100-0x1FF} & Context 0 general purpose registers \\ \hline
\texttt{0x200-0x3FF} & Context 0 control registers \\ \hline
\texttt{0x400-0x4FF} & Mirror of global control registers \\ \hline
\texttt{0x500-0x5FF} & Context 1 general purpose registers \\ \hline
\texttt{0x600-0x7FF} & Context 1 control registers \\ \hline
\end{tabular}
\end{table}

\end{itemize}

