
\subsubsection{System control}
\label{sec:core-ug-cfg-inst-iface-syscon}

The system control signals include the clock source for the core, a synchronous 
reset signal and a global clock enable signal. \code{clk} and \code{reset} are 
required \code{std_logic} input signals. \code{clkEn} is an optional 
\code{std_logic} input signal.

The core is clocked on the rising edge of \code{clk} while \code{clkEn} is high. 
When a rising edge on \code{clk} occurs while \code{reset} is high, most
components of the core will be reset, regardless of the state of \code{clkEn}.
The only component of the core that is not reset by this is the general purpose
register file. This is because this register file is implemented using block
RAMs, which have no physical reset input in Xilinx FPGAs.

The \code{resetOut} signal is asserted high for one cycle when the debug bus 
writes a one to the reset bit in \creg{GSR}. This signal may be used to reset 
support systems as well as the core, or it may be ignored.

