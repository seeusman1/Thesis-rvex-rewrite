
\clearpage
\section{Reconfiguration and sleeping}
\label{sec:core-ug-reconf}

The process in which the \rvex{} processor switches between one large core and
more smaller cores is called reconfiguration. Reconfigurations may be requested
by the software running on the processor or the debugging interface by writing
the requested configuration to a control register. The reconfiguration
controller will then temporarily stop all contexts that will be affected by the
reconfiguration, commit the new configuration, and (re)start any contexts that
are part of the new configuration but are currently stopped.

\subsection{Configuration word encoding}
\label{sec:core-ug-reconf-word}

A configuration is described by means of a single register at most 32-bits in
size. The actual size depends on the design-time configuration of the core; in
particular, the number of lane groups and the number of contexts.

In the configuration word, each nibble (group of 4 bits, represented by a single
hexadecimal digit) maps to a lane group. The nibble signifies the context that
is to be run on that lane group. Disabling a lane group to save power is also
possible, by selecting `context' eight. This will never map to an actual
context, as the maximum amount of hardware contexts supported by the design-time
configuration system is also eight, and numbering starts at zero.

Obviously, not all 4.2 billion 32-bit values represent valid configurations.
Configuration words must adhere to the following rules.

\begin{itemize}

\item The nibbles for existing pipelane groups may be set to either zero through 
the number of hardware contexts minus one to select a context, or eight to 
disable the pipelane group. For instance, the configuration word 0x7777 is
illegal on an \rvex{} processor that does not support eight hardware contexts.
Configuration words like 0x9999 are reserved for future configurations, such as
fault tolerant duplicate and triplicate modes.

\item The nibbles for non-existant pipelane groups must be set to zero. For 
instance, the configuration word 0x88880000 is illegal for an \rvex{} processor 
that is design-time configured to only support 4 lane groups, even though it 
may make more sense than the configuration word that was probably the intention 
here, which is simply zero.

\item Any context may only be mapped to a power-of-two of contiguous pipelane
groups. For instance, configuration words 0x1118 and 0x1231 are illegal, because
the mapping for context 1 violates these rules.

\item A set of pipelane groups mapped to a single context must be aligned.
Mathematically, the index of the first pipelane group in the set must be
divisible by the cardinality of the set. For instance, the configuration word
0x0112 is illegal, because the mapping for context 1 is improperly aligned.

\end{itemize}

\noindent The reconfiguration controller will ensure that a configuration word 
is valid before committing it to the processor. If an invalid configuration is 
requested, the E flag in \creg{GSR} is set and the request is otherwise ignored.

\subsection{Requesting a reconfiguration}
\label{sec:core-ug-reconf-request}

There are three ways in which a reconfiguration can be requested.

\begin{itemize}

\item Writing to the \creg{CRR} context control register from a program running
on the core. This section primarily deals with this mechanism.

\item Writing to the \creg{BCRR} global control register from the debug bus. This
mechanism is equivalent to the first, except it is triggered from outside the
core.

\item Using the sleep and wake-up system, as described in
Section~{sec:core-ug-reconf-saw}.

\end{itemize}

\noindent Usually, when a reconfiguration is requested, the new configuration 
will be committed within something in the order of tens of cycles, depending on 
how long it takes the reconfiguration controller to pause the affected contexts. 
However, a reconfiguration may also be rejected, either another context or the 
bus is requesting a new configuration simultaneously and arbitration is lost, or 
because the requested configuration is invalid. The following C function
correctly deals with arbitration, and performs a best-effort attempt at
detecting errors without using locks implemented in software.

\begin{lstlisting}[numbers=none, basicstyle=\footnotesize, language=C]
/**
 * Requests a reconfiguration. Returns 1 if reconfiguration was successful,
 * -1 if the requested configuration is invalid or 0 if it is not known
 * whether the configuration was valid or not.
 */
int reconfigure(unsigned int newConfiguration) {
    
    // Extract our own context ID from the register file, which we will use
    // to check if we won arbitration or not.
    int ourselves = CR_CID;
    
    // Used to store the ID of the winning context after the request.
    int winner;
    
    // Retry requesting the new configuration until we win arbitration.
    do {
        
        // Request the new configuration.
        CR_CRR = newConfiguration.
        
        // Load the GSR register for state information.
        gsr = CR_GSR;
        
        // Extract the reconfiguration requester ID field from GSR.
        int winner = (gsr & CR_GSR_RID_MASK) >> CR_GSR_RID_BIT;
        
    } while (winner != ourselves);
    
    // Busy-wait for reconfiguration to complete.
    while (gsr & CR_GSR_B_MASK) {
        gsr = CR_GSR;
    }
    
    // If our context is still the one that was the last to request a
    // reconfiguration, the error flag in GSR is also meant for us. If not,
    // there is no way to tell if the configuration we requested was valid
    // or not.
    if (((gsr & CR_GSR_RID_MASK) >> CR_GSR_RID_BIT) != ourselves) {
        return 0;
    }
    
    // If the error flag is set, return -1.
    if (gsr & CR_GSR_E_MASK) {
        return -1;
    }
    
    // Reconfiguration was successful.
    return 1;
    
}
\end{lstlisting}

\subsection{Sleep and wake-up system}
\label{sec:core-ug-reconf-saw}

The sleep and wake-up system refers to two context control registers that only 
exist on context zero, through which the processor can be set up to 
automatically request a reconfiguration when the interrupt request input of 
context zero is asserted. More specifically, the wakeup system will activate
when all of the following conditions are met.

\begin{itemize}
\item The S flag in \creg{SAWC} is set.
\item An interrupt is pending on context 0.
\item Context 0 is not already active in the current configuration.
\item There is no reconfiguration in progress.
\end{itemize}

\noindent When activated, the following actions are performed.

\begin{itemize}
\item A reconfiguration to the configuration stored in \creg{WCFG} is requested.
\item \creg{WCFG} is set to the old configuration.
\item The S flag in \creg{SAWC} is cleared.
\end{itemize}

\noindent This system may be used to save power that is otherwise wasted in an 
idle loop, or to improve interrupt latency by dedicating hardware context zero 
to only handling interrupts. These use cases are described below.

\subsubsection{Power saving}
\label{sec:core-ug-reconf-saw-power}

To conserve power, the user may want to switch to a configuration where all 
pipelane groups are idle until an interrupt occurs. This is called sleeping. On 
an FPGA this is merely a proof of concept, but in an ASIC the amount of power 
that might be saved by clock gating or powering down the computational 
resources may be very significant. To go to sleep, the program should take the 
following steps.

\begin{enumerate}

\item If other hardware contexts were running other tasks in parallel to
context zero, which may be in a state in which the processor should not sleep,
first request these tasks to pause gracefully. If necessary, request a
reconfiguration to configuration zero, as described in 
Section~\ref{sec:core-ug-reconf-request}. to disable all contexts except for 
context zero.

\item Disable interrupts using the I field in \creg{CCR}.

\item If necessary, ensure that no interrupt occured before interrupts were
disabled that should cause the processor to stay awake. If this did happen,
take the appropriate actions, such as re-enabling interrupts, before attempting
to sleep again.

\item Copy \creg{CC}, the current configuration, to \creg{WCFG}, the wake-up
configuration. This is an easy way to ensure that \creg{WCFG} will not contain
an invalid configuration.

\item Set the S flag in \creg{SAWC} to enable the wake-up system.

\item Request a reconfiguration to the configuration where all pipelane groups 
are disabled, for instance 0x8888 on a core that is design-time configured to 
have four pipelane groups, as described in
Section~\ref{sec:core-ug-reconf-request}.

\item Busy-loop until the S flag in \creg{SAWC} is cleared. This ensures that
the program will not continue until after the processor has finished sleeping.

\item Enable interrupts using the I field in \creg{CCR} to service the
interrupt. The fact that this is not done automatically also allows the
interrupt request input to simply be used as a wake-up input in a simple system
where no interrupts exist.

\end{enumerate}

\subsubsection{Decreasing interrupt latency}
\label{sec:core-ug-reconf-saw-latency}

To decrease interrupt latency, context zero may be used as a dedicated context
for servicing interrupts. This prevents the context zero trap handler from
having to save and restore the state of the processor as it was before the
interrupt trap, as this information is not relevant. The other hardware contexts
may be used to run the main program; the reconfiguration system is then used for
hardware context switching.

To initialize this system, the program should do the following in context zero.

\begin{enumerate}

\item Set up links to the trap and panic handlers for context 0 in \creg{TH} and
\creg{PH}.

\item Copy \creg{CC}, the current configuration, to \creg{WCFG}, the wake-up
configuration. This is an easy way to ensure that \creg{WCFG} will not contain
an invalid configuration.

\item Set the S flag in \creg{SAWC} to enable the wake-up system.

\item Request a reconfiguration as described in 
Section~\ref{sec:core-ug-reconf-request}, to, for instance, 0x1111, if the main 
program is to run in hardware context 1.

\item Busy-loop until the S flag in \creg{SAWC} is cleared. This ensures that
the program will not continue until after the first interrupt is requested.

\item Set ready-for-trap and enable interrupts using the R and I fields in 
\creg{CCR} to service the interrupt.

\item Busy-loop forever to wait for the interrupt to be serviced.

\end{enumerate}

\noindent The other contexts can initialize in the usual manner. The context 0 
trap handler should do the following.

\begin{enumerate}

\item Perform body of the regular trap handling tasks, i.e., everything except 
for saving and restoring the context and executing \insn{RFI}.

\item Set ready-for-trap and enable interrupts using the R and I fields in 
\creg{CCR} to quickly service the next interrupt if one is already pending.
Clear ready-for-trap and disable interrupts in the next cycle again; one cycle
is enough for an interrupt to be handled.

\item Store the contents of \creg{WCFG} in a temporary register.

\item Copy \creg{CC}, the current configuration, to \creg{WCFG}, the wake-up
configuration. This is an easy way to ensure that \creg{WCFG} will not contain
an invalid configuration.

\item Set the S flag in \creg{SAWC} to enable the wake-up system.

\item Request a reconfiguration to the configuration as stored in the temporary
register, as described in Section~\ref{sec:core-ug-reconf-request}.

\item Busy-loop until the S flag in \creg{SAWC} is cleared. This ensures that
the program will not continue until after the first interrupt is requested.

\item Set ready-for-trap and enable interrupts using the R and I fields in 
\creg{CCR} to service the interrupt.

\item Busy-loop forever to wait for the interrupt to be serviced.

\end{enumerate}



\todo[inline]{Write more about the sleep and wake-up system}



