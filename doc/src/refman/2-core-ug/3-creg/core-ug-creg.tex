
\label{sec:core-ug-creg}

The \rvex{} processor has two control register files. These are the global 
control register file (\code{gbreg}) and the context control register file 
(\code{cxreg}).

The \code{gbreg} file contains mostly status information, such as a general 
purpose cycle counter, the current configuration vector and design-time 
configuration information. While the debug bus has read/write access to 
\code{gbreg}, the core can only read from it.

For more information, refer to Section~\ref{sec:core-ug-isa-regs-creg}.

\subsection{Global control registers}
\label{sec:core-ug-creg-gb}

The following table lists the global control registers of the \rvex{} processor. 
The offsets listed are with respect to the control register base address. If you 
are viewing this manual digitally, you can click the register mnemonics on the 
right to jump to their documentation.

\subimport{../../generated/}{gbreg.generated.tex}

\subsection{Context control registers}
\label{sec:core-ug-creg-cx}

The following table lists the context control registers of the \rvex{} 
processor. The offsets listed are with respect to the control register base 
address. If you are viewing this manual digitally, you can click the register 
mnemonics on the right to jump to their documentation.

\subimport{../../generated/}{cxreg.generated.tex}

\subsection{Performance counter registers}
\label{sec:core-ug-creg-perf}

All performance counters share the same nontrivial 64-bit structure,
representing up to 56 bits worth of counter data. The actual size is design-time
configurable using the \code{CFG} vector, and may be read from field P in
\creg{EXT0}.

Each performance counter may be reset independently by writing an even value to
the low register. Alternatively, all context-specific performance counters may
be reset at the same time by writing an odd number to one of the performance
counter low registers.

64-bit reads cannot be performed atomically in the \rvex{}. Therefore, reliably
reading the performance counters when they are configured to be larger than a
32-bit word is impossible to do in general, without additional hardware.

Typically, a holding register is implemented for either the low or the high
word, which is loaded at the exact same time the other word is read. While this
is fine in a single-processor environment, a multiprocessor environment would
need such a holding register for each processor separately. To make matters
worse, this holding register would also need to be saved and restored when a
software context is swapped out. This makes this solution more trouble than it's
worth.

In the \rvex{}, this problem is not avoided completely, but it is mitigated. 
Each counter is limited to seven bytes, and the middle byte is mirrored by both 
the low and high register if the counter is larger than a 32-bit word. This 
permits the following algorithm for a semi-reliable performance counter read.

\begin{lstlisting}[numbers=none, basicstyle=\footnotesize, language=C]
/**
 * Loads a 40-bit, 48-bit or 56-bit performance counter value. Do not use this
 * when the counter size is set to 32-bit!
 */
uint64_t read_counter(
    volatile uint32_t *low,
    volatile uint32_t *high
) {
    
    // Perform the read.
    uint32_t l = *low;
    uint32_t h = *high;
    
    // Check if the counters have overflowed.
    if (l >> 24) != (h & 0xFF) {
        
        // There was an overflow, so clear the low value.
        l = 0;
        
    }
    
    // Combine the values and return.
    return ((uint64_t)h << 24) | l;
    
}
\end{lstlisting}

\noindent Note that this algorithm will \emph{not} work when the counters are
configured to be 32 bits wide. In this case the high word register is
intentionally not implemented in order to save hardware, which means that the
overflow check will not work properly.

The algorithm assumes that the value is monotonously increasing. This 
is true for all performance counters as long as it is impossible for them to be 
reset during the read. As long as there was no 32-bit overflow during the read, 
the returned value will always be a counter value between what is was when 
\code{low} was read and what it was when \code{high} was read. If there 
\emph{was} such an overflow, there is a small chance (1/256 if the added value 
during the read would be uniformly distributed) that the returned value is 
slightly higher than what the counter value was when \code{high} was read.

As an example, the worst case scenario is that the counter is at 0xFFFFFF when
\code{low} is read (\code{l} = 0xFFFFFF), and at 0x100000000 when \code{high} is
read (\code{h} = (\code{l} = 0x100). This will result in 0x100FFFFFF being
returned, or about 0.4\% too much. This is, however, completely insignificant
compared to the jitter which may be expected in the value when such a delay is
possible between the two reads. It would require an extremely long interrupt
service routine or software context switch happening at exactly the wrong time,
and when such things are going on in the background.

