
\clearpage
\section{Traps and interrupts}
\label{sec:core-ug-traps}

There are many systems in a processor that need to be able to interrupt normal
program flow. For instance, an external interrupt may be requested, or a problem
occured while trying to load a word from memory, such as a page fault. The
naming conventions for such interruptions varies from processor to processor; in
the \rvex{} processor all such interruptions are called traps. The word
`interrupt' is reserved for the special trap that deals with external
interrupts, i.e. asynchronous signals from outside the core. The word `fault' is
used to refer to traps that signal that an instruction could not be executed.

\subsection{Trap sources}
\label{sec:core-ug-traps-sources}

There are roughly six sources of traps in the \rvex{} processor, which are
handled in slightly different ways.

\begin{itemize}

\item Faults. A fault signals that an instruction could not be executed for some
reason. They are always handled by the processor by jumping to the trap or panic
handler. With the exception of page faults, these traps are usually
non-recoverable, leading to abnormal termination of the executing task in an
operating system environment, or the \insn{STOP} instruction to be called in a
bare-metal environment.

\item Interrupts. The \rvex{} processor core has an interface for an interrupt
controller. When an interrupt is requested and interrupts are enabled through
the I flag in \creg{CCR}, a \trap{EXT_INTERRUPT} trap will be generated. This
trap causes the processor to jump to the trap or panic handler.

\item Context switch request. When the values in \creg{RSC} and \creg{CSC} do
not match and the context switching system is enabled by means of the the C flag
in \creg{CCR}, a \trap{CONTEXT_SWITCH} trap will be generated. This trap causes
the processor to jump to the trap or panic handler.

\item Breakpoints/debug traps. When a hardware or software breakpoint is hit 
while breakpoints are enabled through the B flag in \creg{CCR}, the processor 
will generate a debug trap. Debug traps can be handled in two ways, depending on 
the E and I flags in \creg{DCR}. When the I flag is set (this is the default), 
the traps will be handled as any other trap, i.e. by jumping to the trap or 
panic handler. However, when the E flag is set, the context will simply halt, 
and write the trap cause to the cause field in \creg{DCR}. This allows an 
external debugging system to handle the breakpoints instead of the processor 
itself. In addition, debug traps are disabled for every first instruction 
executed after returning from a trap handler or restarting the context, allowing 
either to jump over breakpoints.

\item \insn{TRAP} instructions. The \insn{TRAP} instruction can be used to
emulate any trap. If the cause maps to a debug trap, it is handled exactly as a
debug trap, allowing it to be used as a software breakpoint. Otherwise, it is
handled like a fault.

\item \insn{STOP} instructions. A \insn{STOP} instruction halts the core by
generating a \trap{STOP} trap during execution of the subsequent instruction.
The \trap{STOP} trap is always handled by stopping the hardware context. In
addition, the D flag in \creg{DCR} is set and the done output signal for the
stopped hardware context is asserted.

\end{itemize}

\subsection{Trap and panic handlers}
\label{sec:core-ug-traps-handlers}

As stated above, most traps are handled by jumping to the so-called trap or
panic handler. These handlers are simply subroutines that typically end with
either a \insn{RFI} or \insn{STOP} instruction. They should be pointed to by
the \creg{TH} and \creg{PH} control registers; it is up to the initialization
code to set up these links.

The hardware switches between the trap and panic handlers based on the R flag
in \creg{CCR}. The only hardware difference between them is that this flag
always switches to the panic handler upon servicing a trap, such that a trap
that immediately follows another trap will always be handled by the panic
handler.

The tentative difference between the two trap handlers is that one should 
attempt to jump back to the application (or alternatively, in the case of an 
operating system, kill the current process with the appropriate signal and 
context switch to another thread), and the other should not. The necessity of 
such a difference can be best illustrated with a simple example.

Consider a program that has just been trapped due to an interrupt. The first 
course of action in handling a trap must always be to save the state of the 
running program, so the trap cause and argument registers can be examined. Now 
consider that it is possible for these context-saving memory accesses to cause, 
say, a misaligned memory access, due to a programming error. A regular trap 
handler may in theory try to recover from the fault by emulating the faulting 
instruction and then jumping over it. However, if it would do so, the trap 
point, cause and argument control registers of the original interrupt trap will 
have been overwritten with the misaligned memory access. This data was simply
lost when the second trap occured; there is no way around this. Thus, the
program cannot continue.

Through the dual handler system as implemented in the \rvex{} processor, the
first trap will be handled by the regular trap handler. Upon jumping to this
trap handler, the processor will automatically clear the ready-for-trap flag,
such that the second trap will be handled by the panic handler.

What it comes down to, is that the trap handler may try to recover from a fault,
handle an interrupt or breakpoint, etc., while the panic handler should simply
display or log an error message if it can, and then stop execution or reset. The
regular trap handler may also want to jump to the panic handler if it is posed
with a fault trap that it cannot recover from.

\subsection{Trap identification}
\label{sec:core-ug-traps-ids}

In the \rvex{} processor, traps are identified not by the address that the 
processor branches to (as there are only two of these addresses, as described in 
the previous section), but by the trap cause (\creg{TC}) and trap argument 
(\creg{TA}) control registers. The former stores an 8-bit value that identifies 
the cause of the trap. The latter is a 32-bit register whose significance 
depends on the trap cause.

The list below documents the trap causes as currently defined in the processor,
and the significance of the trap argument. Note, however, that a \insn{TRAP}
instruction may emulate any of these traps with any argument.

\subimport{../../generated/}{traps.generated.tex}

\subsection{State saving and restoration}
\label{sec:core-ug-traps-save-restore}

Upon entering a trap, it is mostly up to the software to save and restore the
processor state. Specifically, the software must ensure that the state of the
general purpose registers, branch registers and the link register is as it was
when the trap handler was entered when the \insn{RFI} instruction is executed.
The hardware will handle saving and restoration of the context control flags in
\creg{CCR} and the program counter, as both of these are modified immediately
when entering the trap handler. \creg{CCR} is saved in and restored from
\creg{SCCR}, the program counter is saved in and restored from \creg{TP}.

Aside from restoring the state of the currently running task, an operating
system environment may also wish to restore the state of a different task. In
this case, the complete state of a task is defined by the contents of the
general purpose register file, the branch register file, the link register, the
program counter (to be accessed using \creg{TP}) and the context control
register (to be accessed using \creg{SCCR}).

