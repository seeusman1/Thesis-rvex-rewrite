
\subsubsection{Instruction memory interface}
\label{sec:core-ug-cfg-inst-iface-imem}

These signals interface between the \rvex{} and the instruction memory or cache. 
All signals in this section are clock gated by not only \code{clkEn}, but also 
by the respective signal in \code{rv2mem_stallOut}. They should be considered to 
be invalid when the respective \code{rv2mem_stallOut} signal is high. The number 
of enabled clock cycles without stalls after which the reply for a request is 
assumed to be valid is defined by \code{L_IF}, which is defined in 
\code{core_pipeline_pkg}. \code{L_IF} defaults to 1.

\begin{itemize}

\item \code{rv2imem_PCs : out rvex_address_array(}\textit{number of lane groups - 1}\code{ downto 0)}

Program counter outputs for each lane group. These will always be aligned to the 
size of an instruction for a full lane group. When lane groups are coupled, the 
PC for the first lane group will always be aligned to the size of the 
instruction to be executed on the set of lane groups, and the PCs for those lane 
groups will be contiguous.

\vspace{1em}
\item \code{rv2imem_fetch : out std_logic_vector(}\textit{number of lane groups - 1}\code{ downto 0)}

Read enable output. When high, the instruction memory should supply the 
instructions pointed to by \code{rv2imem_PCs} on \code{imem2rv_instr} after 
\code{L_IF} processor cycles.

\vspace{1em}
\item \code{rv2imem_cancel : out std_logic_vector(}\textit{number of lane groups - 1}\code{ downto 0)}

Cancel signal. This signal will go high combinatorially (regardless of the stall 
input from the memory) when it has been determined that the result of the most 
recently requested instruction fetch will not be used. In this case, the memory 
may cancel the request in order to be able to release the stall signal earlier. 
This signal can safely be ignored for correct operation.

\vspace{1em}
\item \code{imem2rv_instr : in rvex_syllable_array(}\textit{number of} lanes \textit{- 1}\code{ downto 0)}

Syllable input for each lane. Expected to be valid \code{L_IF} processor cycles 
after \code{rv2imem_fetch} is asserted if \code{rv2imem_cancel} and 
\code{imem2rv_fault} are low.

\vspace{1em}
\item \code{imem2rv_affinity : in std_logic_vector(}\textit{n} log(\textit{n}) \textit{- 1}\code{ downto 0)}\\
\textit{Where n = number of lane groups}

Optional block affinity input signal for reconfigurable caches. If used, it is 
expected to have the same timing as the \code{imem2rv_instr} signal. Each lane 
group has log(\textit{number of lane groups}) bits in this signal, forming an 
unsigned integer that indexes the lane group that serviced the instruction 
read. When the processor wants to reconfigure, it may use this signal as a hint 
to determine which program should be placed on which lane group next, assuming 
that there will be fewer cache misses if the currently running application is 
mapped to the lane group indexed by the affinity signal. Its value is made
available to the program using the \creg{AFF} register.

\vspace{1em}
\item \code{imem2rv_busFault : in std_logic_vector(}\textit{number of lane groups - 1}\code{ downto 0)}

Instruction fetch bus fault input signal. Expected to have the same timing as the 
\code{imem2rv_instr} signal. When high, a \trap{FETCH_FAULT} trap is generated 
and the instruction defined by \code{imem2rv_instr} will not be executed.

\end{itemize}

