
\subsubsection[.tex command reference]{\texttt{.tex} command reference}
\label{sec:core-ug-cfg-cregs-cref}

The following LaTeX-like commands are interpreted by the Python scripts to 
define the control registers. They must be the only thing on a certain line 
aside from optional LaTeX-style comments at the end of the line, otherwise they 
are interpreted as part of a documentation section.

\begin{lstlisting}[numbers=none, language=nothing]
- \contextInterface{}, \globalInterface{}
   '- \ifaceGroup{title}
       '- \ifaceSubGroup{}
           |- \ifaceIn{unit}{name}{type}
           |- \ifaceOut{unit}{name}{type}{expr}
           |- \ifaceInCtxt{unit}{name}{type}
           '- \ifaceOutCtxt{unit}{name}{type}{expr}
           
- \defineTemplate{name}{parameter list}

- \register{mnemonic}{name}{offset},
  \registergen{python range}{mnemonic}{name}{offset}{stride}
   '- \field{range}{mnemonic}
       |- \reset{bit vector}
       |- \signed{}
       |- \id{identifier}
       |- \declaration{}
       |   |- \declRegister{name}{type}{expr}
       |   |- \declVariable{name}{type}{expr}
       |   '- \declConstant{name}{type}{expr}
       |
       |- \implementation{}
       |- \resetImplementation{}
       |- \finally{}
       '- \connect{output}{expr}
       
- \perfCounter{mnemonic}{name}{offset}
   |- \declaration{}
   |   |- \declRegister{name}{type}{expr}
   |   |- \declVariable{name}{type}{expr}
   |   '- \declConstant{name}{type}{expr}
   |
   '- \implementation{}
\end{lstlisting}


\codehead{\contextInterface{}}

\codehead{\globalInterface{}}

\noindent These commands describe the port map of the context register logic and 
the global register logic respectively. They may appear more than once in the 
configuration; their contents will simply be appended.

\vskip 6 pt
\codehead{\ifaceGroup{<title>}}

\codehead{\ifaceSubGroup{}}

\noindent These commands define port groups for code readability. The toplevel 
group has a title. Both group commands will interpret the text following the 
command as comments for the code.

\vskip 6 pt
\codehead{\ifaceIn{<unit>}{<name>}{<type>}}

\codehead{\ifaceOut{<unit>}{<name>}{<type>}{<expr>}}

\codehead{\ifaceInCtxt{<unit>}{<name>}{<type>}}

\codehead{\ifaceOutCtxt{<unit>}{<name>}{<type>}{<expr>}}

\noindent These commands define ports. For inputs, the signal name will be 
\code{<unit>2cxreg_<name>} or \code{<unit>2gbreg_<name>}. For outputs it will be 
\code{cxreg2<unit>_<name>} or \code{gbreg2<unit>_<name>}. \code{<type>} is a 
type specification as defined later. \code{<expr>} is an expression using only 
predefined constants (Section~\ref{sec:core-ug-cfg-cregs-predef}), input signals 
and literals. The command name determines whether the port is per-context or 
global and whether it is an input or output.

\vskip 6 pt
\codehead{\register{<mnemonic>}{<name>}{<offset>}}

\noindent This command starts a new register description. \code{<name>} is the 
title of the section. \code{<mnemonic>} is the mnemonic of the register, 
excluding the \code{CR_} prefix. The mnemonic must be mix of up to eight 
uppercase, number or underscore characters, and must be unique. The register may 
be referenced in LaTeX as \code{\creg{<mnemonic>}}; this will generate a 
hyperlink in the PDF to the register documentation. \code{<offset>} should be a 
hex number starting with \code{0x} divisible by 4, representing the byte offset 
from the control registers base. Global registers should be within the 
\code{0x000..0x0FF} range, context registers should be within 
\code{0x200..0x3FF}. \code{0x100..0x1FF} is reserved for the general purpose 
register file.

\vskip 6 pt
\codehead{\registergen{<python range>}{<mnemonic>}{<name>}{<offset>}{<stride>}}

\noindent Same as \code{\register}, but specifies a list of registers. 
\code{<python range>} is executed as a Python expression, expected to generate 
an iterable of integers. A register is generated for each of these iterations. 
The offset for each register is computed as \code{<offset>} $+ iter *$ 
\code{<stride>}. \code{\n{}} expands to the number when used inline in 
\code{<mnemonic>} and \code{<name>}. In the documentation it expands to 
\code{$n$}.

\vskip 6 pt
\codehead{\field{<range>}{<mnemonic>}}

\noindent This command defines a field in the current register. A range 
specification is either a single bit index for a single-bit field, or of the 
form \code{<from>..<to>}, where \code{<from>} is the higher bit index, and both 
the \code{<from>} and \code{<to>} bits are included in the range. For example, 
\code{3..1} includes bits 1, 2 and 3. \code{<mnemonic>} should be a short, 
uppercase identifier for the field, which must be unique within the register. It 
should be as short as possible, in particular for single-bit fields, as it needs 
to fit in the layout of the documentation. It also needs to be a valid C and 
VHDL identifier, so for instance spaces and hyphens are not allowed.

\vskip 6 pt
\codehead{\reset{<bit vector>}}

\noindent This command sets the reset state of the previously defined field. If 
not specified, the reset state is assumed to be zero. The number of characters 
in \code{<bit vector>} must equal the number of bits in the field.

\vskip 6 pt
\codehead{\signed{}}

\noindent This command marks a field as being a signed number. The default is 
unsigned.

\vskip 6 pt
\codehead{\id{<identifier>}}

\noindent This command gives a field an alternative name for the 
C/VHDL/\code{rvd} definitions. This only works for 8-bit and 16-bit fields that 
are properly aligned.

\vskip 6 pt
\codehead{\declaration{}}

\noindent This command specifies the local register, variable and constant 
declaration section for this field implementation.

\vskip 6 pt
\codehead{\declRegister{<name>}{<type>}{<expr>}}

\codehead{\declVariable{<name>}{<type>}{<expr>}}

\codehead{\declConstant{<name>}{<type>}{<expr>}}

\noindent These commands specify registers, variables or constants respectively. 
These may be used by the implementation code. \code{<name>} must start with an 
underscore, and will expand to 
\code{cr_<register-mnemonic>_<field-mnemonic><name>}. It must be a valid C and 
VHDL identifier. \code{<type>} is a type name, as defined in 
Section~\ref{sec:core-ug-cfg-cregs-types}. \code{<expr>} is an expression which 
defines the constant, initial value or reset value, using only predefined 
constants (Section~\ref{sec:core-ug-cfg-cregs-predef}) for a constant value, and 
only inputs or predefined constants for variables and registers.

\vskip 6 pt
\codehead{\implementation{}}

\noindent This command starts a language-agnostic code section as defined in 
Section~\ref{sec:core-ug-cfg-cregs-code}, executed every rising clock edge with 
\code{clkEn} high and \code{reset} inactive. The following variables are 
predefined to interface with the core and debug busses for regular register
fields.

\begin{itemize}

\item \code{_write}: the value being written if the corresponding \code{_wmask}
bits are set.

\item \code{_wmask}: write mask for each bit in the field, honoring both writes
from the core directly and writes from the debug bus.

\item \code{_wmask_dbg}: same as \code{_wmask}, but only honors debug bus
accesses.

\item \code{_wmask_core}: same as \code{_wmask}, but only honors accesses made
by the core directly.

\item \code{_read}: this must be written to in the \code{\implementation{}}
section to specify the read value for the field.

\end{itemize}

\noindent The types of these variables are \code{bitvec}s with the same width as 
the field. As an example of how to use these, a simple register may be created 
as follows. This requires \code{_reg} to be declared using \code{\declRegister} 
as a \code{bitvec} of the field size.

\begin{lstlisting}[numbers=none, language=C]
_reg = (_reg & ~_wmask) | (_write & _wmask);
_read = _reg;
\end{lstlisting}

\noindent Performance counter implementations do not have these variables. They
have \code{_add} instead. This is a \code{byte}-typed variable which specifies
how much should be added to the performance counter in this cycle. The bus
interfacing logic is generated to conform to Section~\ref{sec:core-ug-creg-perf}.

\vskip 6 pt
\codehead{\resetImplementation{}}

\noindent This command starts a language-agnostic code section as defined in 
Section~\ref{sec:core-ug-cfg-cregs-code}, executed every rising clock edge with 
\code{clkEn} high, while the global \code{reset} signal is inactive but the 
context-specific reset signal is active. This allows register implementations to 
override a soft context reset, for instance to make register values persistent 
in this case. This is necessary for, for instance, the B flag in \creg{DCR}, to 
allow the debugger to reset the core without immediately starting execution. 
This command is only allowed for context-specific registers.

\vskip 6 pt
\codehead{\finally{}}

\noindent This command starts a language-agnostic code section as defined in 
Section~\ref{sec:core-ug-cfg-cregs-code}, executed every rising clock edge with 
\code{clkEn} high and \code{reset} inactive, after all \code{\implementation{}} 
sections have been processed. Variables from other fields and registers may be 
read in this section, in addition to all the objects which are accessible from 
\code{\implementation{}}. This allows registers to be written combinatorially
from multiple field implementations, by having the regular field implementations
prepare variables that describe the new value, and subsequently combining the
variable values in a \code{\finally{}} section.

\vskip 6 pt
\codehead{\connect{<output>}{<expr>}}

\noindent This command combinatorially connects the specified output port with 
the specified expression. This expression may use registers and predefined 
constants (Section~\ref{sec:core-ug-cfg-cregs-predef}). This may be used to 
easily connect an output port to an internal register. Note however, that since 
inputs cannot be used in the expression, it still cannot make a combinatorial 
path from an input to an output. This is illegal specifically because such 
combinatorial paths would be needlessly difficult to model with a simulator.

\vskip 6 pt
\codehead{\perfCounter{<mnemonic>}{<name>}{<offset>}}

\noindent This command generates a performance counter register conforming to 
Section~\ref{sec:core-ug-creg-perf}. The counter will occupy two 32-bit register 
slots from \code{<offset>} onwards, holding up to 7 bytes worth of counter data. 
The implementation expects \code{_add} to be set to the value which is to be 
added to the counter; all the bus interfacing logic is generated. The counter 
value register is accessible from other implementations as 
\code{CR_<mnemonic>_<mnemonic>0_r}.
   
