
\subsubsection{LAC object objects and references}
\label{sec:core-ug-cfg-cregs-refs}

All objects except for predefined constants
(Section~\ref{sec:core-ug-cfg-cregs-predef}) need to be declared using the
\code{\iface*} and \code{\decl*} commands. These objects can then be referenced
as follows, for both reads and writes.

\begin{itemize}

\item The most basic way to reference an object is to use its full name. For 
inputs, this takes the form \code{<unit>2cxreg_<name>} or 
\code{<unit>2gbreg_<name>}, depending on whether it is part of the context or 
global register file interface. Likewise, outputs take the form 
\code{cxreg2<unit>_<name>} or \code{gbreg2<unit>_<name>}. Finally, objects
declared using the \code{\decl*} commands take the form
\code{cr_<register-mnemonic>_<field-mnemonic>_<name>}. The register and field
mnemonics are included to make them unique within global scope.

\item In the LAC sections for a certain field, objects that are declared in the
same field can be referenced using just \code{_<name>}. That is, the
\code{cr_<register-mnemonic>_<field-mnemonic>} part of the object name is
implicit in the reference.

\item To explicitly specify a context for context-specific objects, an \code{@} 
symbol and a natural literal determining the context may be appended. For 
instance, \code{_pc@2} references the value of \code{_pc} for context 2. Note 
that the context \emph{must} be a literal, not even constant objects are 
permitted. The only thing that is permitted is \code{\n{}} in generated 
registers, as this expands to a natural literal before parsing, similar to how a 
C macro would behave. If the context is not explicitly specified, the current 
context is used. If there is no current context, the context must be explicitly 
specified.

\item Member access for aggregate types is done by appending a \code{.} (decimal 
point) followed by the member name. Rudimentary support is provided for array 
members using \code. As with explicitly specified contexts, only 
natural literals and \code{\n{}} (in a \code{\registergen} environment) are 
allowed for specifying the index.

\item VHDL-like bit indexes and slices are supported for \code{unsigned} and 
\code{bitvec} types. Indexing a single bit is done by appending 
\code{[<position>]} to the end of the reference, where \code{<position>} may be 
any natural-typed expression. The result of the indexing operation is a 
\code{bit}. Slicing is done using \code{[<position>, <length>]}, where 
\code{<position>} still accepts any natural-typed expression, but 
\code{<length>} only supports natural literals. This is necessary to determine 
the resulting type at compile time, which is an \code{unsigned} or \code{bitvec} 
of size \code{<length>}. \code{<position>} specifies the lower bit index of the
slice range.

\end{itemize}
