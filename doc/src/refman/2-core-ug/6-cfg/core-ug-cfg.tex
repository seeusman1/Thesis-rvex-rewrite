
\clearpage
\section{Design-time configuration}
\label{sec:core-ug-cfg}

\todo[inline]{The design-time configuration section is an incomplete mess.}

The \rvex{} platform is highly design-time configurable. The important metrics
of the core, such as the issue width and the degree of reconfigurability, are
controlled by VHDL generics. This allows for heterogeneous multicore systems,
where \rvex{} processors of different sizes are combined. Capability registers
in the global control registers of each \rvex{} core may be used by the software
running on the core to determine which features it can use. These registers can
also be read through the debug bus, allowing debug support software to detect
the configuration as well. The available generics are listed in detail in
Section~\ref{sec:core-ug-cfg-inst-iface-generics}.

In addition to the VHDL generics based configuration system, a second system is
in place. This system resides in the \code{config} directory in the root of the
\rvex{} git repository. It consists of a set of LaTeX-formatted configuration
files, interpreted by a set of python scripts. These scripts generate various
configuration and source files in the repository, including some of the LaTeX
sources for this very document, related to the following things.

\begin{itemize}

\item Global and context control register file functionality, memory map and
      documentation (Section~\ref{sec:core-ug-cfg-cregs}).

\item Instruction set decoding and documentation, as well as assembly syntax
      (Section~\ref{sec:core-ug-cfg-opcodes}).

\item Pipeline configuration of the \rvex{} core
      (Section~\ref{sec:core-ug-cfg-pipeline}).

\item Trap decoding and documentation (Section~\ref{sec:core-ug-cfg-traps}).

\end{itemize}

\noindent The purpose of this configuration system is primarily to have a 
centralized set of files which control the interfaces between the VHDL files of 
the processor, the assembly and compilation toolchain, the debug system and this 
documentation. Thus, if it is necessary to change something, it only needs to be 
changed in one place, so everything necessarily remains synchronized.

However, as this system directly affects the source files of the core, it is 
impossible to have different core configurations in the same design. In 
addition, it would be impractical to encode all this information in the 
capability registers of the core. Differences in configuration can be detected 
as a change of the core version tag, however. The automated versioning system is 
described in Section~\ref{sec:core-ug-version}.

All configuration files share a LaTeX-like syntax. This allows for comfortable
editing of the LaTeX documentation sections using text editors which support
LaTeX syntax highlighting. It is, however, not possible or intended to supply
the files directly to the LaTeX toolchain. They are only intended to be read by
the configuration scripts.

