
\subsubsection{Run control}
\label{sec:core-ug-cfg-inst-port-rctrl}

The run control signals provide an interface between the core and an interrupt 
controller or a master processor if the \rvex{} is used as a coprocessor. All 
signals are optional. All signals are arrays of some sort, indexed by hardware 
context IDs in descending order.

\begin{itemize}

\item \code{rctrl2rv_irq : in std_logic_vector(}\textit{number of contexts - 1}\code{ downto 0)}
\item \code{rctrl2rv_irqID : in rvex_address_array(}\textit{number of contexts - 1}\code{ downto 0)}
\item \code{rv2rctrl_irqAck : out std_logic_vector(}\textit{number of contexts - 1}\code{ downto 0)}

When \code{rctrl2rv_irq} is high, an interrupt trap will be generated within the 
indexed context as soon as possible, if the interrupt enable flag in the context 
control register is set. Interrupt entry is acknowledged by 
\code{rv2rctrl_irqAck} being asserted high for one \code{clkEn}abled cycle. 
\code{rctrl2rv_irqID} is sampled in exactly that cycle and is made available to 
the trap handler through the trap argument register. When not specified, 
\code{rctrl2rv_irq} is tied to \code{'0'} and \code{rctrl2rv_irqID} is tied to 
\code{X"00000000"}.

When \code{rv2rctrl_irqAck} is high, an interrupt controller would typically 
release \code{rctrl2rv_irq} and set \code{rctrl2rv_irqID} to a value signalling 
that no interrupt is active on the subsequent clock edge. Alternatively, if more
interrupts are pending,  \code{rctrl2rv_irq} may remain high and
\code{rctrl2rv_irqID} may be set to the  code identifying the next interrupt.

Releasing \code{rctrl2rv_irq} before an interrupt is acknowledged may still 
cause an interrupt trap to be caused. This is due to the fact that traps take 
time to propagate through the pipeline. The core will still assert 
\code{rv2rctrl_irqAck} upon entry of the trap service routine in this case. In 
order to properly account for this behavior, interrupt controllers should ignore 
\code{rv2rctrl_irqAck} if no interrupt is active, and there should be a special 
\code{rctrl2rv_irqID} value that signals `no interrupt'. The trap service 
routine should return to application code as soon as possible in this case.

\vspace{1em}
\item \code{rctrl2rv_run : in std_logic_vector(}\textit{number of contexts - 1}\code{ downto 0)}
\item \code{rv2rctrl_idle : out std_logic_vector(}\textit{number of contexts - 1}\code{ downto 0)}

When \code{rctrl2rv_run} is asserted low, the indexed context will stop 
executing instructions as soon as possible. It will finish instructions that 
were already in the pipeline and have already committed data, and set the 
program counter to point to the next instruction that should be issued for the 
program to resume correctly later. As soon as \code{rctrl2rv_run} is asserted
high again, the context will resume, assuming there is nothing else preventing
it from running. When \code{rctrl2rv_run} is not specified, it is tied to
\code{'1'}.

Only when the context has completely stopped, i.e., there are no instructions in 
the pipeline, will \code{rv2rctrl_idle} be asserted high. This may also happen 
while \code{rctrl2rv_run} is high, when the core is being halted for a different 
reason. Such reasons include preparing for reconfiguration, the context not 
having lane groups assigned to it, and the B flag in \creg{DCR}. 
\code{rv2rctrl_idle} remains high until the next instruction is fetched.

\vspace{1em}
\item \code{rctrl2rv_reset : in std_logic_vector(}\textit{number of contexts - 1}\code{ downto 0)}
\item \code{rctrl2rv_resetVect : in rvex_address_array(}\textit{number of contexts - 1}\code{ downto 0)}
\item \code{rv2rctrl_done : out std_logic_vector(}\textit{number of contexts - 1}\code{ downto 0)}

When \code{rctrl2rv_reset} is asserted high, the context control registers for 
the indexed context are synchronously reset in the next \code{clkEn}abled cycle. 
Note that this behavior is different from the master \code{reset} signal, which 
ignores \code{clkEn}. When it is not specified, it is tied to \code{'0'}.

\code{rctrl2rv_resetVect} determines the reset vector for each context, i.e. the
initial program counter. When it is not specified, it is tied to the reset
vector specified by the \code{CFG} generic.

\code{rv2rctrl_done} is connected to the D flag in \creg{DCR}, which is set when 
the processor executes a \insn{STOP} instruction. The only way to clear this 
signal without debug bus accesses is to assert \code{reset} or 
\code{rctrl2rv_reset}.

When the \rvex{} is running as a co-processor, \code{rctrl2rv_reset} could be 
used as an active low flag indicating that the currently loaded kernel needs to 
be executed, in which case \code{rv2rctrl_done} signals completion.
\code{rctrl2rv_resetVect} marks the entry point for the kernel.

\end{itemize}

