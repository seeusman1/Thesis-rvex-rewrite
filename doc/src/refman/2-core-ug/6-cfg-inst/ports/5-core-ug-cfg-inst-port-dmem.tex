
\subsubsection{Data memory interface}
\label{sec:core-ug-cfg-inst-port-dmem}

These signals interface between the \rvex{} and the data memory or cache. All 
signals in this section are clock gated by not only \code{clkEn}, but also by 
the respective signal in \code{rv2mem_stallOut}. They should be considered to be 
invalid when the respective \code{rv2mem_stallOut} signal is high. The number of 
enabled clock cycles after which the reply for a request is assumed to be valid 
is defined by \code{L_MEM}, which is defined in \code{core_pipeline_pkg}. 
\code{L_MEM} defaults to 1.

\begin{itemize}
  
\item \code{rv2dmem_addr : out rvex_address_array(}\textit{number of lane groups - 1}\code{ downto 0)}

Memory address that is to be accessed if \code{rv2dmem_readEnable} or 
\code{rv2dmem_writeEnable} is high. The two least significant bits of the 
address will always be \code{"00"} and may be ignored. Note that a configurable 
1 kiB block within this 4 GiB memory space is inaccessible, because it is 
replaced by the core control registers. This is configurable through the 
\code{cregStartAddress} entry in \code{CFG}, which defaults to 0xFFFFFC00, 
meaning that addresses 0xFFFFFC00 through 0xFFFFFFFF are inaccessible.

\vspace{1em}
\item \code{rv2dmem_readEnable : out std_logic_vector(}\textit{number of lane groups - 1}\code{ downto 0)}

Active high read enable signal from the core for each memory unit. When high 
during an enabled rising clock edge, the \rvex{} expects the access result to be 
valid \code{L_MEM} enabled cycles later.

\vspace{1em}
\item \code{rv2dmem_writeData : out rvex_data_array(}\textit{number of lane groups - 1}\code{ downto 0)}
\item \code{rv2dmem_writeMask : out rvex_mask_array(}\textit{number of lane groups - 1}\code{ downto 0)}

These signals define the write operation to be performed when 
\code{rv2dmem_writeEnable} is high. \code{rv2dmem_writeMask} contains a bit for 
each byte in \code{rv2dmem_writeData}, which determines whether the byte should 
be written or not: when high, the respective byte should be written; when low, 
the byte should not be affected. Mask bit $i$ governs data bits $i*8+7$ downto 
$i*8$. This corresponds to byte address $a + 3 - i$, where $a$ is the word
address specified by \code{rv2dmem_addr}, because the \rvex{} is big endian.

\vspace{1em}
\item \code{rv2dmem_writeEnable : out std_logic_vector(}\textit{number of lane groups - 1}\code{ downto 0)}

Active high write enable signal from the core for each memory unit. When high 
during an enabled rising clock edge, the \rvex{} expects either that the write 
request defined by \code{rv2dmem_addr}, \code{rv2dmem_writeData} and 
\code{rv2dmem_writeMask} will be performed, or that \code{dmem2rv_ifaceFault} or 
\code{dmem2rv_busFault} is asserted high \code{L_MEM} cycles later.

\vspace{1em}
\item \code{dmem2rv_readData : in rvex_data_array(}\textit{number of lane groups - 1}\code{ downto 0)}

This is expected to contain the read data for read requested by 
\code{rv2dmem_readEnable} and \code{rv2dmem_addr} \code{L_MEM} enabled cycles 
earlier, unless \code{dmem2rv_ifaceFault} or \code{dmem2rv_busFault} are high.

\vspace{1em}
\item \code{dmem2rv_ifaceFault : in std_logic_vector(}\textit{number of lane groups - 1}\code{ downto 0)}

These signals are expected to be valid \code{L_MEM} enabled cycles after a read 
or write request. \code{dmem2rv_ifaceFault} being high indicates that the read 
or write could not be performed because the memory system is incapable of 
servicing the specific type of memory access. For instance, the reconfigurable 
cache asserts this signal if more than one request is made at a time by coupled 
lane groups. \code{dmem2rv_busFault} being high indicates that some kind of bus 
fault occured, for example if a memory access was made to unmapped memory.

In either case, a \code{DMEM_FAULT} trap will be issued. The trap argument will 
be set to the address that was requested.

\todo[inline]{There is currently no way to distinguish between a data memory
interface fault and a bus fault. A new trap should probably be added to the core
for this sometime.}

\end{itemize}

