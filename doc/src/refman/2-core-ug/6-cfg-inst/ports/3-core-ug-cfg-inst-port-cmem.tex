
\subsubsection{Common memory interface}
\label{sec:core-ug-cfg-inst-port-cmem}

These control signals are common to both the data and instruction memory 
interface.

\begin{itemize}

\item \code{rv2mem_decouple : out std_logic_vector(}\textit{number of lane groups - 1}\code{ downto 0)}

This vector represents the current runtime configuration of the core. In 
particular, it specifies which lane groups are working together to execute code 
within a single context. When a bit in this vector is high, the indexed lane 
group is `decoupled' from the next lane group, i.e., is operating within a 
different context. When a bit is low, the indexed lane group is working as a 
slave to the next higher indexed lane group for which the bit is set.

Due to constraints in the core, the indices of pipelane groups working together 
are always aligned to the number of pipelane groups in the group. As an example, 
if pipelane groups 0 and 1 are working together, group 2 cannot join them 
without group 3 also joining them. This allows binary tree structures to be used 
in the coupling logic. This means that, in the default core configuration, only 
the following decouple vectors are legal: \code{"1111"}, \code{"1110"}, 
\code{"1011"}, \code{"1010"} and \code{"1000"}.

The state of the \code{rv2mem_decouple} signal has several implications on the 
behavior of the memory ports on the \rvex{}.

\begin{itemize}

\item The PCs presented by the instruction memory ports will always be 
contiguous and aligned for groups which are working together. The fetch and 
cancel signals will always be equal.

\item The \rvex{} assumes that the \code{mem2rv_blockReconfig} and 
\code{mem2rv_stallIn} signals are equal for coupled pipelane groups. 
Behavior is completely undefined if these assumptions are violated.

\end{itemize}

\vspace{1em}
\item \code{mem2rv_blockReconfig : in std_logic_vector(}\textit{number of lane groups - 1}\code{ downto 0)}

This signal can be used by the memories to block reconfiguration due to ongoing 
operations. When a bit in this vector is high, the context associated with the 
indexed group is guaranteed to not reconfigure. The \rvex{} will assume that the 
associated bits in the \code{mem2rv_blockReconfig} signal will always be 
released eventually when no operations are requested by those pipelane groups, 
otherwise the system may deadlock. When pipelane groups are coupled, their 
respective \code{mem2rv_blockReconfig} signals must be equal. When this 
signal is not specified, it is tied to all zeros.

\vspace{1em}
\item \code{mem2rv_stallIn : in std_logic_vector(}\textit{number of lane groups - 1}\code{ downto 0)}

Stall input signals for each pipelane group. When the stall signal for a 
pipelane group is high, the next rising edge of the clock signal will be 
ignored. When pipelane groups are coupled, their respective 
\code{mem2rv_stallIn} signals must be equal. When this signal is not specified, 
it is tied to all zeros.

\vspace{1em}
\item \code{rv2mem_stallOut : out std_logic_vector(}\textit{number of lane groups - 1}\code{ downto 0)}

Stall output signals for each pipelane group. This serves as a combined stall 
signal from all stall sources, indicating whether a pipelane group is actually 
stalled or not. When \code{rv2mem_stallOut} is high, all memory request signals 
from the associated pipelane group should be considered to be undefined. Memory 
access requests should thus be initiated (and registered) only at the rising 
edge of the \code{clk} signal when \code{clkEn} is high and the associated 
\code{rv2mem_stallOut} signal is low. In addition, the result of a previously
requested memory operation should remain valid until the next \code{clkEn}abled
cycle where the \code{rv2mem_stallOut} signal is low, as this is when the core
will sample the signal.

When pipelane groups are coupled, their respective \code{rv2mem_stallOut} 
signals will be equal. In addition, the \code{unifiedStall} configuration 
parameter in the \code{CFG} record may be set to \code{true} to enforce equal 
stall signals for all pipelane groups at all times, should this be desirable for 
the memory implementation.

\vspace{1em}
\item \code{mem2rv_cacheStatus : in rvex_cacheStatus_array(}\textit{number of lane groups - 1}\code{ downto 0)}

This signal may be driven with cache status information. This is used by the 
trace unit only. The data type is a record defined in \code{core_pkg} as 
follows.

\begin{lstlisting}[numbers=none]
type rvex_cacheStatus_type is record
  instr_access      : std_logic;
  instr_miss        : std_logic;
  data_accessType   : std_logic_vector(1 downto 0);
  data_bypass       : std_logic;
  data_miss         : std_logic;
  data_writePending : std_logic;
end record;
\end{lstlisting}

All signals must be externally gated by the stall signals of the core for 
compatibility with performance counters in the future. Otherwise, the 
\code{instr_} prefixed signals share the timing of the instruction fetch result, 
and \code{data_} prefixed signals share the timing of the data memory access 
result.

\code{instr_access} should be high when an instruction fetch was performed. In
this case, \code{instr_miss} may also be high to signal that the fetch caused a
cache miss.

\code{data_access} should be set to \code{01} if a read access was performed,
to \code{10} if a 32-bit write access was performed and to \code{11} if a
partial write was performed. \code{00} logically means no operation. If an
access was performed which bypassed the cache, \code{data_bypass} should be set.
If an access was performed which caused a cache miss, \code{data_miss} should be
set. If an access was performed by a cache block which had a nonempty write
buffer when the request was made, \code{data_writePending} should be set.

Note that these signals are very cache implementation dependent. They were
designed to work with the reconfigurable cache described in
Section~\ref{sec:cache} specifically. It may, of course, be abused by other
cache implementations, as long as the people working with the resulting traces
are adequatly informed.

\end{itemize}

